\chapter{Specificity and Transparency}

\chapterprecishere{We discuss two important aspects of the interpretation of DPs
  in intensional contexts.}

\minitoc

\section{Predictions of our framework}
\label{sec:predictions}

When a DP occurs in the scope of an intensional operator, our framework makes
clear predictions. Consider, for example:

\ex
Chris wanted Dana to buy a book about soccer.
\xe
%
Imagine that we give the following meaning to \emph{want}:

\ex
$\sv{want}^{w,g} = \lambda p_{\type{s,t}}.\lambda x_e.\ \forall w'$ such that
all of $x$'s wants in $w$ are satisfied in $w'\co p(w')=1$. 
\xe
%
In other words, \emph{$x$ wants $p$} is true iff $p$ is true in all worlds where
$x$'s wants are satisfied. Further, assume that the DP \emph{a book about
  soccer} is interpreted within the embedded clause. Then, we claim, (\blastx)
will be true iff in all of the worlds that satisfy all of Chris' wants, there is
a book about soccer that Dana buys. (You prove this claim in this exercise:)

\begin{exercise}
  Draw the obvious, if simplified, LF for (\blastx) and calculate its
  truth-conditions. \eex
\end{exercise}
%
Now, consider what happens if the object of the lower verb QRs and adjoins to
the matrix clause:

\ex
[a book about soccer] (1 [Chris wanted Dana to buy $t_1$])
\xe
%
When you calculate the truth-conditions of (\lastx) [please do so], you will get
a result that is very different from the previous exercise. Now what is claimed
is that there is a book about soccer, call it $x$, such that in all of the
worlds satisfying all of Chris' wants Dana buys $x$.

There are two important differences between the truth-conditions that our framework
assigns to these two LFs.

\emph{Quantifier scope}: Since \emph{want} is a universal quantifier over
worlds and \emph{a book about soccer} is an existential quantifier over
individuals, there's a question about the relative scope of the two quantifiers.
In the first truth-conditions we sketched, the existential quantifier scopes
under the universal quantifier: for every world there is an individual such that
bla-bla. In the second LF, the existential quantifier scopes over the universal
one: there is an individual such that in every world yadda-yadda. The most
common terminology for this difference involves the pair
\emph{specific/non-specific}: the sentence (or the object DP) is interpreted
specifically if the DP takes scope over the intensional operator, and it is
interpreted non-specifically if the DP takes scope under the intensional operator.

\emph{Predicate evaluation}: When the existential quantifier scopes over the
intensional operator, this also has the effect that the predicate contained in
it, \emph{book about soccer} is evaluated in the matrix evaluation world. And
when the quantifier takes lower scope, its predicate is evaluated in the worlds
that the intensional operator shifts to. One evocative terminology for whether
the predicate is evaluated with respect to the matrix evaluation world or the
worlds shifted to by the intensional operator is \emph{transparent/opaque}. A
predicate evaluated relative to the matrix world is called \emph{transparent}. A
predicate evaluated in the worlds shifted to by an intensional operator is
receiving an \emph{opaque} interpretation.

There's another terminological pair that is very common: \emph{de re}/\emph{de
  dicto}. One way to conceive of that distinction in our framework is that it
stands for a particular combination of the two distinctions we just introduced:
\emph{de re} means specific and transparent, and \emph{de dicto} means
non-specific and opaque.

Caution about terminology: terminological confusion and exuberance is rampant in
this area (and many others). In a way, terminology is just a short-hand way to
pick out salient properties of LFs (or their denotation). It's the latter that
truly matters. One particular problematic aspect of the terminology is its
binary nature, while the relevant distinctions are actually more complex,
especially as soon as we are dealing with nested intensionality.

\begin{exercise}
  Consider the sentence \emph{Chris must want Dana to buy a book about soccer}.
  One can imagine using this to describe a scenario where we are seeing Dana
  enter a bookstore known to cater to soccer aficionados. For some reason we
  won't go into, we come to the conclusion that there is a specific book about
  soccer that Chris must have asked Dana to buy. But at the same time, we have
  no idea what that book might be, so there's not a specific book about which we
  made our deduction. This suggest that we may want to give the object DP
  intermediate scope. So, draw an LF that corresponds to this idea and calculate
  its truth-conditions.
\end{exercise}

Our recommendation is to use the terms \emph{specific/non-specific},
\emph{transparent/opaque}, \emph{de re/de dicto} only with extreme caution. They
are sometimes useful shorthands, but unless it is crystal-clear what properties
of LFs/denotations you are using them to pick out, they are more likely to be a
source of obfuscation and confusion.


