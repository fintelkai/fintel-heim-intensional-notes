\excnt=1
\chapter{The argument structure of modals}
\label{chap:arg}

\chapterprecishere{Are modals really raising predicates? Are they at least
  sometimes control predicates? If they are ``raising'' predicates, is that
  syntactic raising or some kind of ``semantic'' raising?}

\minitoc

\section{Raising vs. Control}
\label{sec:raising-control}

\marginnote{This section is based on lecture notes by Kai von Fintel and Sabine
  Iatridou for a class on Morphology, syntax, and semantics of modals at the
  2009 LSA Summer Institute at the University of California at Berkeley.}


% In a tradition going back to Jackendoff (1972), it has been widely assumed that deontic and epistemic modals differ in certain basic syntactic properties.
% One should keep in mind that this is an important issue, especially if one wants to adopt or develop a system like Kratzer’s (1978, 1981, 1991), in which modals differ only in contextual parameters. Under an idealized such account, there should be no syntactic differences between epistemic and deontic modals, for example.
% Specifically, the idea is/was that deontics are instances of a control structure, while epistemics are raising predicates:
%  (15) a. John must be there at 5pm (Deontic) b. [John must [PRO be there at 5pm]]
% 5
% (16) John must be there already (Epistemic) [ ec must [John be there already]]
% [Johnk must [tk be there already]]
% The idea behind this intuition was pretty straightforward: Deontics assert that the subject has a certain property, namely the property of having a particular obligation (or permission). Therefore, the modal assigns a theta-role to the subject. The lower thematic subject, of course, also exists; in the absence of Case, it is instantiated as PRO1.
% Epistemics, on the other hand, are propositional predicates. They say nothing about the subject or any particular argument inside the clause. At most, they mediate a relationship between a proposition and the belief system of an individual (the speaker). Therefore, the subject does not receive a theta-role from the epistemic modal. Rather, the modal has a thematic relationship with the entire CP/IP.
% Linguists subsequently further refined their understanding of deontics, taking from the philosophers the distinction between ought to do and ought to be deontic modality. Brennan (1993), probably the first linguist to make this connection, based an interesting syntactic proposal on this distinction2.
% The idea here is that in ought-to-do modality, a particular individual has a certain obligation. For Brennan, ought-to-do modals are control predicates, as they assign a theta-role to their subject (and are infinitive-embedding).
% (17) Ought-to-do:
% John ought to/ has to/should hand this in before 5pm. John should PRO hand this in before 5pm.
% On the other hand, ought-to-be modality does not assert obligation of any particular person.
% For Brennan, ought-to-be modals are raising predicates, as evidenced also by the fact that
%  1
% Alternatively, in a theory without PRO, (15b) would be as follows:
% (15) b'. [John must [ be there at 5pm]]
% Brennan (1993) cites Feldman (1986) as having influenced her work but the idea started earlier.
% 2
% However, Castañeda (1970) gives an early description of the notion, citing even earlier sources:
% “ Deontic concepts like ought, right, obligation, forbidden, and permissible have benefited from the philosophically exciting work in the semantics of modal concepts done by Kanger, Hintikka, Kripke, Montague and others. Their semantics illuminates both the topic and the contribution of the standard axiomatic approach to deontic logic: the topic is what philosophers used to call the Ought-to-be. On the other hand, the non- standard approach represented by early axiomatic deontic systems of ours deals with the Ought-to-do."
% 6
% they can support expletive subjects.
% (18) Ought-to-be:
% There ought to/have to/should be laws against things like this.
% In short, for Brennan, epistemic modals are raising predicates but deontic modals come in two varieties: Control (17) and Raising (18).
% In effect, the difference between ought-to-do and ought-to-be modals is whether the obligation is understood to be borne by somebody. The idea was that when there is a bearer of the obligation, and hence an ought-to-do modal, the carrier of the obligation would be the thematic subject of the ought-to-do modal. With ought-to-do modality, there is no bearer of obligation. In other words, either the bearer of obligation is the theta- marked subject, or there is no bearer of obligation.
% However, it turns out that there are deontic sentences that have a bearer of obligation that is not the syntactic subject.
% From Bhatt (1998):
% (19) John has to eat an apple today.
% (said as an instruction to John’s caretaker at the day-care, who is therefore
% the carrier of the obligation)
% (20) Bill has to be consulted by John on every decision.
% (John is the bearer of the obligation) From Wurmbrand (1999):
% (21) The traitor must die.
% (22) The old man must fall down the stairs and it must look like an accident.
% From Claire Halpert (p.c.):
% (23) The security guard must not see you as you break into the museum.
% Given such data, the idea that the bearer of the obligation is a theta-role assigned under structural conditions to the subject becomes difficult to maintain. An overt non-subject can be the bearer of the obligation, as in (20) (argument of by-phrase) and (23) (object of the verb). In addition, the bearer of the obligation can also be absent syntactically, as in (19), (21), and (22).
% 7
% Furthermore, as Bhatt points out, (24) is structurally like an example of Brennan’s ought- to-be modality and thus is not expected to have a carrier of the obligation at all. However, when said to the caterer, it becomes an ought-to-do modality, with the caterer being the carrier of the obligation.
% (24) We are expecting fifty guests tonight. There have to be 50 chairs in the living room room by 5p.m.
% What both Bhatt and Wurmbrand conclude is that there is no structurally assigned theta- role ‘carrier of the obligation’. It is claimed that there are no syntactic differences in the representations of ought-to-do and ought-to-be deontic modals; all deontic modals are raising constructions. That is, deontic modals never come with a theta-role of obligation (or permission). However, there is an inference mechanism that can identify the carrier of the obligation, who can appear in various syntactic positions (see (20) and (23)), or absent altogether, as in (19), (21-22) and (24).
% In addition, both Bhatt and Wurmbrand bring to the fore several syntactic arguments that deontic modals pattern like raising predicates and not like control.
% One of their arguments comes from Case. Bhatt discusses Hindi, but the argument there is a bit more complicated than we can do justice here. Wurmbrand’s argument from Icelandic Case is easier to convey in a few words.
% In Icelandic, while most verbs take nominative subjects, there are verbs that take accusative subjects (the verb meaning ‘lack’) and verbs that take dative subjects (the verb meaning ‘like’ ). When these verbs are embedded under a control predicate, the higher subject gets the Case associated with the higher verb. When they are embedded under a raising predicate, the subject appears in the Case associated with the embedded predicate:
% (25) NPnom VControl Pred PRO lack/like DP (26) NPacc/dat VRaising Pred t lack/like DP
% When verbs with quirky subjects are embedded under a modal, the case of the subject of the modal depends on the embedded predicate.3
% (27) NPacc/dat Modal lack/like DP Here are Wurmbrand’s actual Icelandic data:
% 3
% Unlike Thráinsson and Vikner (1995), who say that such sentences receive epistemic readings, Wurmbrand says that deontic modals have the same pattern.
% 8
 
% (28)
% (29)
% a.
% b.
% a.
% b.
% c.
% Harald / *Haraldur vantar peninga. Harold-ACC / *Harold-NOM lacks money ‘Harold lacks money’
% Haraldi / *Haraldur líkar vel í Stuttgart. Harold-DAT / *Harold-NOM likes well in Stuttgart ‘Harold likes it in Stuttgart’
% Haraldur / *Harald vonast til a! vanta ekki peninga. Harold-NOM / *Harold-ACC hopes for to lack not money ‘Harold hopes not to lack money’
% Haraldur / *Haraldi vonast til a! líka vel i Stuttgart. Harold-NOM / *Harold-DAT hopes for to like well in Stuttg. ‘Harold hopes to like it in Stuttgart’
% Harald vir!ist vanta ekki peninga.
% Harold-ACC seems lack not money ‘Harold seems not to lack money’
% Haraldi / *Haraldur verour a! líka hamborgarar. Harold-DAT / *Harold-NOM must to like hamburgers ‘Harold must like hamburgers’ (in order to be accepted by his new American in-laws)
% Umsaekjandann veraur a! vanta peninga. The-applicant-ACC must to lack money
% ‘The applicant must lack money’ (in order to apply for this grant)
% (30)
% a.
% b.
% Note that the fact that nominative is impossible shows that raising is not just an option with these Icelandic modals, but is the only choice.
% These and other syntactic arguments convinced a fair amount of people that all deontic modals are raising predicates, just like epistemic modals4.
% This conclusion also fits Kratzer’s proposal, since a common syntactic representation is more easily compatible with the view that epistemics and deontics differ only in conversational backgrounds and ordering source.
% However, the conclusion that deontic modals are uniformly raising predicates may be too
% 4
% Though we do not have time to go into this issue here, the same question arises for other modals, e.g, dynamic and ability ones. For these, see Hacquard (2006) and Hackl (1998).
% 9
 
% strong for the general case. Bhatt and Wurmbrand have shown for sure that some deontic modals must be raising predicates. However, there is no apparent conceptual reason why there couldn’t be deontic modals that are control predicates.
% Recently, Nauze (2008) revives the position that English deontic modals are control predicates. However, his position is that the syntactic and semantic properties of modals differ significantly crosslinguistically. (Interestingly, this is in combination with a rejection of Kratzer’s proposal that deontic/epistemic, etc., distinctions among modals are only contextually determined.)
% Nauze puts faith in what he considers Brennan’s strongest argument for the status of English deontic modals as theta-role assigners5 (Nauze p. 148, (16a,b), repeated below as (31)).
% Sentences (31a) and (31b) are equivalent:
% (31) a. The president shook hands with John. b. John shook hands with the president.
% Epistemic predicates, which are propositional arguments for Brennan, retain this equivalence:
% (32) a. The president may/must have shaken hands with John. b. John may/must have shaken hands with the president.
% However, deontics do not (Nauze 149, (20a,b)):
% (33) a. The president must shake hands with John. b. John must shake hands with the president.
% Brennan/Nauze say that as ought-to-be modality (e.g. as said to the president’s campaign director or John’s secretary) the two sentences are equivalent. However, as ought-to-do modality (33a,b) are not equivalent. For Brennan and Nauze this is the result of the deontic modal assigning a theta-role to ‘the president’ in (a) but to ‘John’ in (b).
% Is this a necessary conclusion? Once we have our inference mechanism (which we need anyway for sentences like (19-24), we can use it to choose either ‘the president or ‘John’ as the bearer of the obligation to shake hands with the other. Nothing would have to follow about the syntax of the two modals6.
%  5
% He does not provide arguments against the position that all English deontics are raising predicates
% but does acknowledge the existence of that position in fn 78 on page 126, where he cites Wurmbrand
% (1999) and Barbiers (2006).
% 6
% To be fair to Nauze, he does seem to acknowledge a path to this possibility in fn 27 on p. 149:
% 10
% However, as we said earlier, even if we manage to show that all deontic modals in English and in Icelandic are raising predicates, there is no conceptual reason given as to why deontics could only be raising –unlike epistemics, whose status as propositional predicates seems quite clear.
% To conclude this section:
% • There has been a debate about the syntactic status of deontic and epistemic modals: do they assign a theta-role to their syntactic subjects (making them control predicates) or do they not (making them raising predicates)? Epistemics are uncontroversially taken to be raising predicates. The actual debate mostly centers on deontics. While there may be reason to expect that all epistemics will be raising predicates crosslinguistically, it may well turn out to be the case that deontics can go either way.
% • These questions should be extended to other modals (ability, dynamic etc), as well as to other languages.
% • In addition, one should keep in mind that if one wants a theory like Kratzer’s, where modals differ only in contextual parameters, possible syntactic differences like raising versus control will have to be addressed seriously.



