%!TEX root = IntensionalSemantics.tex
\chapter{Conditionals}\label{cha:conditionals} % (fold)
\excnt=1

{\setlength{\epigraphwidth}{44ex}\epigraph{The word ``if'', just two tiny letters \\
    Says so much for something so small \\
    The biggest little word in existence; \\
    Never answers, just questions us all\\[6pt]

    If regrets were gold, I'd be rich as a queen \\
    If teardrops were diamonds, how my face would gleam \\
    If I'd loved you better, I wouldn't be lonely \\
    If only, if only, if only}{Dolly Parton, \emph{If Only}}}

\chapterprecishere{We develop a possible worlds semantics for conditionals that
  treats the \emph{if}-clause as an intensional operator, with a bit of
  context-dependency thrown in.}

\minitoc

\section{\expression{If} \dots\ the biggest little word}
\label{sec:if}

In many ways, conditionals are the archetypal construction of displacement: the
consequent is evaluated not against the actual here and now but against the
scenario conjured up by the antecedent. Consider a few conditional sentences:

\pex
\a If Kim left before 6am, she got here in time.
\a If there's an earthquake tomorrow, this house will collapse.\label{ex:earthquake}
\a If there had been a massive snowstorm last night, Kai would have stayed home.
\xe
%
These represent the three main subtypes of conditionals (there are more):
(\lastx a) is an ``indicative'' conditional about the past, (\lastx b) is an
indicative conditional about the future, and (\lastx c) is a ``subjunctive''
conditional. For the moment, the differences will be left aside.

The basic idea of how conditionals work is this: the \expression{if}-clause
takes us to a particular possible world (or maybe a set thereof) and the
consequent clause is asserted to be true of that world (or those worlds). But
what world(s) are we being taken to? The most obvious requirement is that the
antecedent of the conditional needs to be true of the world(s). But there's
more.

Given our discussion of how the semantics of fiction operators anchors them in
facts about the actual world (the content of the relevant body of fiction), it
shouldn't come as a surprise that conditionals are similarly anchored. So, look
at the examples in (\lastx): what in the actual world are they about?

\sidepar{It is perhaps unfortunate that David Lewis used a rather whimsical
  example to start off his seminal book on counterfactuals (``If kangaroos had
  not tails, they would topple over''). Or consider this scene from the TV show
  ``Big Bang Theory'': \url{https://www.youtube.com/watch?v=0lpY0Kt4bn8}. As the
  examples in the text make clear, conditionals can be very down-to-earth.}%
Here's a sketch: (\lastx a) is about the local transportation system, the
weather, the traffic, and so on. (\lastx b) is about the sturdiness of this
house, facts of geology, laws of physics, and so on. (\lastx c) is about Kai's
proclivities (such as avoiding traffic snarls), the local climate, and so on.
Since the conditionals are anchored in real world facts, they are no mere
flights of fancy and whether they are true depends on those facts. If today's
traffic was particularly bad, it may be false that Kim's leaving before 6am
would have got her here in time. If the architects went to great lengths to make
the house earthquake-safe, (\lastx b) may well be false. And if there was an
attendance-mandatory faculty meeting, Kai may well have come in in spite of a
massive snowstorm.

So, the outlines of the semantics of conditionals are clear: \expression{if}
takes us to worlds where the antecedent is true but that match the actual world
in certain relevant features. And the consequent then is evaluated in those
worlds. There are many details to work out and we'll return to that task in Part
III of these notes. But for now, we put forward a placeholder analysis.

\section{The Restricted Strict Implication Analysis}

We will treat \expression{if} as a higher-order operator that together with the
antecedent creates an intensional operator with a semantics very similar to the
final analysis we gave to \expression{in the world of Sherlock Holmes} in the
previous chapter. But where the fiction operator directly encoded what features
of the actual world it's sensitive to (the Sherlock Holmes fiction),
conditionals rely on context for this job. Here's the proposal:

\ex\label{ex:if-strict-context}%
$\svt{if}^{w,g} = \lambda p \in D_{\angles{s,t}}.\ \lambda q \in
D_{\angles{s,t}}.$ \\ \null\hfill $\forall w'\co p(w')=1\ \&\ w' \text{ is
  relevantly like } w \rightarrow q(w')=1.$ \xe
%
The contextual anchoring to features of the evaluation world $w$ is here
effected by the placeholder ``relevantly like $w$''. This is crucial because
otherwise the conditional would talk about any world whatsoever where the
antecedent is true. This would make the truth-conditions not just not contingent
on the actual world but also far too strong to allow most sensible conditionals
to be true ever.

Think about the earthquake conditional \refx{ex:earthquake}: we would derive the
absurdly strong truth-conditions that the conditional is true iff \emph{all} of
the worlds where there is a major earthquake in Cambridge tomorrow are worlds
where my house collapses.

There are some obvious and immediate problems with this analysis. For one, while
it's easy to imagine circumstances where the conditional \refx{ex:earthquake} is
judged to be true, there surely are possible worlds where there's an earthquake
but my house does not collapse: perhaps, the builders in that world used all the
recommended best practices to make the building earthquake-safe, perhaps it's a
world where I'm simply unreasonably lucky, or the house is immediately adjacent
to much sturdier neighboring buildings which keep it propped up, or Harry Potter
flies by and protects the house at the last minute (he owes me a favor, after
all). This problem (that the house doesn't in fact collapse in \emph{all}
possible worlds where there's an earthquake but that the conditional can still
be judged true in some worlds) is accompanied with another problem: whether the
conditional is true depends on what the world is like. Was the house built to
exacting standards? Is it propped up by its neighbors? Does Harry Potter owe me
a favor? That is the problem solved by restricting the quantifier over worlds to
world ``relevantly like $w$''.

Obviously, this is a semantics with a ``placeholder'', because what does
``relevantly like'' mean precisely? Now, just because the semantics is therefore
rather vague and context-dependent doesn't mean it is wrong. As
\textcite{lewis:1973:counterfactuals} writes:

\begin{quote} Counterfactuals are notoriously vague. That does not mean that we
  cannot give a clear account of their truth conditions. It does mean that such
  an account must either be stated in vague terms \dash which does not mean
  ill-understood terms \dash or be made relative to some parameter that is fixed
  only within rough limits on any given occasion of language use.
\end{quote}
%
In a later chapter, we will revisit the formulation in
\refx{ex:if-strict-context} and fill it out more explicitly. Before we go on to
consider other intensional constructions, we'll briefly talk about the
astonishingly silly idea that conditionals are not intensional constructions.
 
\section{A red herring: Material implication}

Consider the following example:

\ex If I am healthy, I will come to class. \xe

The simplest analysis of such conditional constructions is the so-called
\term{material implication} analysis,\footnote{Quoth the Stoic philosopher Philo
  of Megara: ``a true conditional is one which does not have a true antecedent
  and a false consequent'' (according to \citet[II,
  110--112]{sextus-empiricus:200:outlines}).} which treats \expression{if} as
contributing a truth-function operating on the truth-values of the two component
sentences (which are called the \term{antecedent} and \term{consequent} \dash
from Latin \dash or \term{protasis} and \term{apodosis} \dash from Greek). The
lexical entry for \expression{if} would look as follows:

\ex\label{ex:material} \marginnote{Note that as a truth-functional connective,
  this \expression{if} does not vary its denotation depending on the evaluation
  world. It's its arguments that vary with the evaluation world.}$\sv{\mbox{if}}
= \lambda u \in D_t.\ \lambda v \in D_t.\ u=0 \mbox{ or } v=1.$ \xe
%
Applied to example in (\blastx), this semantics would predict that the example
is false just in case the antecedent is true, I am healthy, but the consequent
false, I do not come to class. Otherwise, the sentence is true. We will see that
there is much to complain about here. But one should realize that under the
assumption that \expression{if} denotes a truth-function, \emph{this one} is the
most plausible candidate.

\citet{suber:1997:material} does a good job of persuading (or at least trying to
persuade) recalcitrant logic students:

\begin{quote}
	
	After saying all this, it is important to note that material implication does
  conform to some of our ordinary intuitions about implication. For example,
  take the conditional statement, \expression{If I am healthy, I will come to
    class.} We can symbolize it: H $\supset$ C.\footnote{The symbol $\supset$
    which Suber uses here is called the ``horseshoe''. We have been using the
    right arrow $\rightarrow$ as the symbol for implication. We think that this
    is much preferable to the confusing horseshoe symbol. There is an intimate
    connection between universal quantification, material implication, and the
    subset relation, usually symbolized as $\subset$, which is the other way
    round from the horseshoe. The horseshoe can be traced back to the notation
    introduced by \citet{peano:1889:nova}, a capital C standing for
    `conseguenza' facing backwards. The C facing in the other (more ``logical'')
    direction was actually introduced first by \citet{gergonne:1817:essai}, but
    didn't catch on.}
	
	The question is: when is this statement false? When will I have broken my
  promise? There are only four possibilities:
	
	\begin{center}
		\begin{tabular}
			{c|c||c} H & C & H$\supset$ C\\
			\hline T & T & ?\\
			T & F & ?\\
			F & T & ?\\
			F & F & ? 
		\end{tabular}
	\end{center}

\begin{itemize}
		
\item In case \#1, I am healthy and I come to class. I have clearly kept my
  promise; the conditional is true.
\item In case \#2, I am healthy, but I have decided to stay home and read
  magazines. I have broken my promise; the conditional is false.
\item In case \#3, I am not healthy, but I have come to class anyway. I am
  sneezing all over you, and you're not happy about it, but I did not violate my
  promise; the conditional is true.
\item In case \#4, I am not healthy, and I did not come to class. I did not
  violate my promise; the conditional is true.

\end{itemize}
%	
But this is exactly the outcome required by the material implication. The
compound is only false when the antecedent is true and the consequence is false
(case \#2); it is true every other time.

\end{quote}
%
Despite the initial plausibility of the analysis, it cannot be maintained.
Consider again our earthquake example:

\ex[exno=\ref{ex:earthquake}] If there is a major earthquake in Cambridge
tomorrow, my house will collapse. \xe
%
If we adopt the material implication analysis, we predict that
\refx{ex:earthquake} will be false just in case there is indeed a major
earthquake in Cambridge tomorrow but my house fails to collapse. This makes a
direct prediction about when the negation of \refx{ex:earthquake} should be
true. A false prediction, if ever there was one:

\pex \a\label{neg-earthquake} It's not true that if there is a major earthquake
in Cambridge tomorrow, my house will collapse. \a $\not\equiv$ There will be a
major earthquake in Cambridge tomorrow, and my house will fail to collapse. \xe
%
Clearly, one might think that (\lastx a) is true without at all being committed
to what the material implication analysis predicts to be the equivalent
statement in (\lastx b). This is one of the inadequacies of the material
implication analysis.

These inadequacies are sometimes referred to as the ``paradoxes of material
implication''. But that is misleading. As far as logic is concerned, there is
nothing wrong with the truth-function of material implication. It is
well-behaved and quite useful in logical systems. What is arguable is that it is
not to be used as a reconstruction of what conditionals mean in natural
language.

\begin{exercise}
	What prediction does the restricted strict implication analysis in
  \refx{ex:if-strict-context} make about the negated conditional in
  \refx{neg-earthquake}? \eex
\end{exercise}

\begin{exercise}
  Under the assumption that \emph{if} has the meaning in \refx{ex:material},
  calculate the truth-conditions predicted for (\nextx):

  \pex
  \a No student will succeed if he goofs off.
  \a No student $\lambda x$ (if $x$ goofs off, $x$ will succeed)
  \xe

State the predicted truth-conditions in words and evaluate whether they correspond to the actual meaning of (\lastx).
\eex
\end{exercise}

\section{Supplemental Readings}
\label{sec:supplemental}

{\setlength{\parindent}{0pt}\nonzeroparskip

We'll come back to conditionals soon enough, but here are two introductory
readings:

\begin{bibentrylist}
\item \fullcite{fintel-2011-hsk-conditionals}.
\item \fullcite{fintel-2012-subjunctives}.
\end{bibentrylist}

Here is a not so introductory reading:

\begin{bibentrylist}
\item \fullcite{gillies-2016-conditionals}.
\end{bibentrylist}

A couple of defenses of the material conditional:

\begin{bibentrylist}
\item \fullcite{abbott-2010-fopl}.
  \item \fullcite{rieger-2015-simple}.
\end{bibentrylist}

}



% chapter conditionals-first (end)

%%% Local Variables:
%%% mode:latex 
%%% TeX-master: "IntensionalSemantics.tex" 
%%% End: 