%!TEX root = IntensionalSemantics.tex
\chapter{Propositional Attitudes}\label{cha:propositional_attitudes} % (fold)

\chapterprecishere{With the basic framework in place, we now proceed to analyze a number of intensional constructions. We start with the basic possible worlds semantics for propositional attitude ascriptions. We talk briefly about the formal properties of accessibility relations.}

\minitoc

\section{Hintikka's Idea} \label{sec:hintikkas-idea}

Expressions\marginnote{According to \citet{hintikka:1969:attitudes}, the term \term{propositional attitude} goes back to \citet{russell:1940:inquiry}.} like \expression{believe}, \expression{know}, \expression{doubt}, \expression{expect}, \expression{regret}, and so on are usually said to describe \term{propositional attitudes}, expressing relations between individuals (the attitude holder) and propositions (intensions of sentences). 

The simple idea is that \expression{George believes that Henry is a spy} claims that George believes of the proposition that Henry is a spy that it is true.\marginnote{Of course, the possible worlds semantics for propositional attitudes was in place long before the extension to fiction contexts was proposed. Our discussion here has inverted the historical sequence for pedagogical purposes.} Note that for the attitude ascription to be true it does not have to hold that Henry is actually a spy. But where \dash in which world(s) \dash does Henry have to be a spy for it be true that George believes that Henry is a spy? We might want to be inspired by the colloquial phrase ``in the world according to George'' and say that \expression{George believes that Henry is a spy} is true iff in the world according to George's beliefs, Henry is a spy. We immediately recall from the previous chapter that we need to fix this idea up by making space for multiple worlds compatible with George's beliefs and by tying the truth-conditions to contingent facts about the evaluation world. That is, what George believes is different in different possible worlds.

The following lexical entry thus offers itself:

\ex. $\sv{\mbox{believe}}^{w,g} =\\
\null\hfill\lambda p_{\angles{s,t}}.\ \lambda x. \ \forall w' \mbox{ compatible with } x's \mbox{ beliefs in } w\!: p(w') = 1.$

What\marginnote{It is important to realize the modesty of this semantics: we are not trying to figure out what belief systems are and particularly not what their internal workings are like. That is the job of psychologists (and philosophers of mind, perhaps). For our semantics, we treat the belief system as a black box that determines for each possible world whether it considers it possible that it is the world it is located in.} is going on in this semantics? We conceive of George's beliefs as a state of his mind about whose internal structure we will remain agnostic, a matter left to other cognitive scientists. What we require of it is that it embody opinions about what the world he is located in looks like. In other words, if his beliefs are confronted with a particular possible world $w'$, they will determine whether that world may or may not be the world as they think it is. What we are asking of George's mental state is whether any state of affairs, any event, anything in $w'$ is in contradiction with anything that George believes. If not, then $w'$ is compatible with George's beliefs. For all George believes, $w'$ may well be the world where he lives. Many worlds will pass this criterion, just consider as one factor that George is unlikely to have any precise opinions about the number of leaves on the tree in front of my house. George's belief system determines a set of worlds compatible with his beliefs: those worlds that are viable candidates for being the actual world, as far as his belief system is concerned.

Now, George believes a proposition iff that proposition is true in all of the worlds compatible with his beliefs. If there is just one world compatible with his beliefs where the proposition is not true, that means that he considers it possible that the proposition is not true. In such a case, we can't say that he believes the proposition. Here is the same story in the words of \citet{hintikka:1969:attitudes}, the source for this semantics for propositional attitudes:

% The picture of Hintikka here is from his Helsinki homepage: http://www.helsinki.fi/filosofia/filo/henk/hintikka.htm.

\begin{quotation}
	
	\noindent My\marginpar{\centering\includegraphics[height=1in]{hintikka1.JPG}\\ {\tiny \href{http://www.helsinki.fi/filosofia/filo/henk/hintikka.htm}{Jaakko Hintikka}}} basic assumption (slightly simplified) is that an attribution of any propositional attitude to the person in question involves a division of all the possible worlds (\dots) into two classes: into those possible worlds which are in accordance with the attitude in question and into those which are incompatible with it. The meaning of the division in the case of such attitudes as knowledge, belief, memory, perception, hope, wish, striving, desire, etc. is clear enough. For instance, if what we are speaking of are (say) $a$'s memories, then these possible worlds are all the possible worlds compatible with everything he remembers. [\dots]
	
	How are these informal observations to be incorporated into a more explicit semantical theory? According to what I have said, understanding attributions of the propositional attitude in question (\dots) means being able to make a distinction between two kinds of possible worlds, according to whether they are compatible with the relevant attitudes of the person in question. The semantical counterpart to this is of course a function which to a given individual person assigns a set of possible worlds.
	
	However, a minor complication is in order here. Of course, the person in question may himself have different attitudes in the different worlds we are considering. Hence this function in effect becomes a relation which to a given individual and to a given possible world $\mu$ associates a number of possible worlds which we shall call the \term{alternatives} to $\mu$. The relation will be called the alternativeness relation. (For different propositional attitudes, we have to consider different alternativeness relations.)
\end{quotation}

\begin{exercise}
	
	Let's adopt Hintikka's idea that we can use a function that maps $x$ and $w$ into the set of worlds $w'$ compatible with what $x$ believes in $w$. Call this function $\mathcal{B}$. That is,
	
	\ex. $\mathcal{B} = \lambda x.\ \lambda w.\ \{w': w' \mbox{ is compatible with what } x \mbox{ believes in } w\}.$
	
	Using this notation, our lexical entry for \expression{believe} would look as follows:
	
	\ex. $\sv{\mbox{believe}}^{w,g} = \lambda p_{\angles{s,t}}.\ \lambda x.\ \mathcal{B}(x)(w) \subseteq p.$
	
	We are here indulging in the usual sloppiness in treating $p$ both as a function from worlds to truth-values and as the set characterized by that function.
	
	Here now are two ``alternatives'' for the semantics of \expression{believe}:
	
	\ex. \extitle{Attempt 1 (very wrong)}\\[3pt]
	$\sv{\mbox{believe}}^{w,g} = \lambda p \in D_{\angles{s,t}}. \big[ \lambda x \in D.\ p = \mathcal{B}(x)(w) \big]$.
	
	\ex. \extitle{Attempt 2 (also very wrong)}\\[3pt]
	$\sv{\mbox{believe}}^{w,g} = \lambda p \in D_{\angles{s,t}}. \big[ \lambda x \in D.\ p \cap \mathcal{B}(x)(w) \neq \emptyset \big]$.
	
	Explain why these do not adequately capture the meaning of \expression{believe}. \eex
\end{exercise}
%
\begin{exercise}
  Follow-up: The semantics in \Last would have made \emph{believe} into an existential quantifier of sorts: it would say that \emph{some} of the worlds compatible with what the subject believes are such-and-such. You have argued (successfully, of course) that such an analysis is wrong for \emph{believe}. But \emph{are} there attitude predicates with such an ``existential'' meaning? Discuss some candidates. If you can't find any candidates that survive scrutiny, can you speculate why there might be no existential attitude predicates? [Warning: this is unexplored territory!]\eex
\end{exercise}
%
We can also think of belief states as being represented by a function $\mathcal{BS}$\marginnote{$\mathcal{BS}$ is meant to stand for `\underline{\bfseries b}elief \underline{\bfseries s}tate', not for what you might have thought!}, which maps an individual and a world into a set of propositions: those that the individual believes. From there, we could calculate the set of worlds compatible with an individual $x$'s beliefs in world $w$ by retrieving the set of those possible worlds in which all of the propositions in $\mathcal{BS}(x)(w)$ are true: $\{w': \forall p \in \mathcal{BS}(x)(w): p(w') = 1\}$, which in set talk is simply the big intersection of all the propositions in the set: $\cap \mathcal{BS}(x)(w)$. Our lexical entry then would be:

\ex. $\sv{\mbox{believe}}^{w,g} = \lambda p_{\angles{s,t}}.\ \lambda x. \cap \mathcal{BS}(x)(w) \subseteq p.$

\begin{exercise}
	
	Imagine that our individual $x$ forms a new opinion. Imagine that we model this by adding a new proposition $p$ to the pool of opinions. So, $\mathcal{BS}(x)(w)$ now contains one further element. There are now more opinions. What happens to the set of worlds compatible with $x$'s beliefs? Does it get bigger or smaller? Is the new set a subset or superset of the previous set of compatible worlds? \eex
\end{exercise}

\section{Accessibility Relations} \label{sec:access-relat}

Another way of reformulating Hintikka's semantics for propositional attitudes is via the notion of an \term{accessibility relation}. We talk of a world $w'$ being accessible from $w$. Each attitude can be associated with such an accessibility relation. For example, we can introduce the relation $w\mathcal{R}^{\mathcal{B}}_{a}w'$ which holds iff $w'$ is compatible with $a$'s belief state in $w$. We have then yet another equivalent way of specifying the lexical entry for \expression{believe}:

\ex. $\sv{\mbox{believe}}^{w,g} = \lambda p_{\angles{s,t}}.\ \lambda x. \ \forall w': w\mathcal{R}^{\mathcal{B}}_{x}w' \rightarrow p(w') = 1.$

It is profitable to think of different attitudes (belief, knowledge, hope, regret, memory, \dots) as corresponding to different accessibility relations. Recall now\marginnote{Kirill Shklovsky (in class) asked why we call reflexivity, transitivity, and symmetry ``formal'' properties of relations. The idea is that certain properties are ``formal'' or ``logical'', while others are more substantial. So, the fact that the relation ``have the same birthday as'' is symmetric seems a more formal fact about it than the fact that the relation holds between my daughter and my brother-in-law. Nevertheless, one of the most common ways of characterizing formal/logical notions (permutation-invariance, if you're curious) does not in fact make symmetry etc. a formal/logical notion. So, while intuitively these do seem to be formal/logical properties, we do not know how to substantiate that intuition. See \citet{macfarlane:2005:logical-constants} for discussion.} that the linguistic study of determiners benefitted quite a bit from an investigation of the formal properties of the relations between sets of individuals that determiners express. We can do the same thing here and ask about the formal properties of the accessibility relation associated with belief versus the one associated with knowledge, etc. The obvious properties to think about are reflexivity, transitivity, and symmetry.

\subsection{Reflexivity} \label{sec:reflexivity}

A relation is reflexive iff for any object in the domain of the relation we know that the relation holds between that object and itself. Which accessibility relations are reflexive? Take knowledge:

\ex. $w\mathcal{R}^{\mathcal{K}}_{x}w'$ iff $w'$ is compatible with what $x$ knows in $w$.

We are asking whether for any given possible world $w$, we know that $\mathcal{R}^{\mathcal{K}}_{x}$ holds between $w$ and $w$ itself. It will hold if $w$ is a world that is compatible with what we know in $w$. And clearly that must be so. Take our body of knowledge in $w$.\marginnote{We talk here about knowledge entailing (or even presupposing) truth but we do not mean to say that knowledge simply equals true belief. Professors Socrates and Gettier  and their exegetes have further considerations.} The concept of knowledge crucially contains the concept of truth: what we know must be true. So if in $w$ we know that something is the case then it must be the case in $w$. So, $w$ must be compatible with all we know in $w$. $\mathcal{R}^{\mathcal{K}}_{x}$ is reflexive.

Now, if an attitude $X$ corresponds to a reflexive accessibility relation, then we can conclude from \expression{$a$ $Xs$ that $p$} being true in $w$ that $p$ is true in $w$.\marginnote{In modal logic notation: $\square p \rightarrow p $. This pattern is sometimes called \textbf{T} or \textbf{M}, as is the corresponding system of modal logic.} This property of an attitude predicate is often called \term{veridicality}. It is to be distinguished from \term{factivity}, which is a property of attitudes which \emph{presuppose} -- rather than (merely) entail -- the truth of their complement.

If we consider the relation $\mathcal{R}^\mathcal{B}_{x}$ pairing with a world $w$ those worlds $w'$ which are compatible with what $x$ \emph{believes} in $w$, we no longer have reflexivity: belief is not a veridical attitude.\marginnote{The difference between \emph{believe} and \emph{know} in natural discourse is quite delicate, especially when one considers first person uses (\emph{I believe the earth is flat} vs. \emph{I know the earth is flat}).} It is easy to have false beliefs, which means that the actual world is not in fact compatible with one's beliefs, which contradicts reflexivity. And many other attitudes as well do not involve veridicality/reflexivity: what we hope may not come true, what we remember may not be what actually happened, etc.

In modal logic, the correspondence between formal properties of the accessibility relation and the validity of inference patterns is well-studied. What we have just seen is that reflexivity of the accessibility relation corresponds to the validity of $\square p \rightarrow p$. Other properties correspond to other characteristic patterns. Let's see this for transitivity and symmetry.

\subsection{*Transitivity} \label{sec:transitivity}

Transitivity\marginnote{Starred sections are optional.} of the accessibility relation corresponds to the inference $\square p \rightarrow \square \square p$.\marginnote{In the literature on epistemic modal logic, the pattern is known as the \term{KK Thesis} or \term{Positive Introspection}. In general modal logic, it is the characteristic axiom \textbf{4} of the modal logic system \textbf{S4}, which is a system that adds \textbf{4} to the previous axiom \textbf{M}/\textbf{T}. Thus, \textbf{S4} is the logic of accessibility relations that are both reflexive and transitive.} The pattern seems not obviously wrong for knowledge: if one knows that $p$, doesn't one thereby know that one knows that $p$? But before we comment on that, let's establish the formal correspondence between transitivity and that inference pattern. This needs to go in both directions.
  
\begin{figure}[htbp]
  \centering
    \includegraphics[height=1.5in]{transitivity.pdf}
  \caption{Transitivity}
  \label{fig:transitivity}
\end{figure}

\noindent What does it take for the pattern to be valid? Assume that $\square p$ holds for an arbitrary world $w$, i.e. that $p$ is true in all worlds $w'$ accessible from $w$. Now, the inference is to the fact that $\square p$ again holds in any world $w''$ accessible from any of those worlds $w'$ accessible from $w$. But what would prevent $p$ from being false in some $w''$ accessible from some $w'$ accessible from $w$? That could only be prevented from happening if we knew that $w''$ itself is accessible from $w$ as well, because then we would know from the premiss that $p$ is true in it (since $p$ is true in \emph{all} worlds accessible from $w$). Ah, but $w''$ (some world accessible from a world $w'$ accessible from $w$) is only guaranteed to be accessible from $w$ if the accessibility relation is transitive (if $w'$ is accessible from $w$ and $w''$ is accessible from $w'$, then transitivity ensures that $w''$ is accessible from $w$). This reasoning has shown that validity of the pattern requires transitivity. The other half of proving the correspondence is to show that transitivity entails that the pattern is valid.

The proof proceeds by reductio. Assume that the accessibility relation is transitive. Assume that (i) $\square p$ holds for some world $w$ but that (ii) $\square \square p$ doesn't hold in $w$. We will show that this situation cannot obtain. By (i), $p$ is true in all worlds $w'$ accessible from $w$. By (ii), there is some non-$p$ world $w''$ accessible from some world $w'$ accessible from $w$. But by transitivity of the accessibility relation, that non-$p$ world $w''$ must be accessible from $w$. And since \emph{all} worlds accessible from $w$ are $p$ worlds, $w''$ must be a $p$ world, in contradiction to (ii). So, as soon as we assume transitivity, there is no way for the inference not to go through.

Now, do any of the attitudes have the transitivity property? It seems rather obvious that as soon as you believe something, you thereby believe that you believe it (and so it seems that belief involves a transitive accessibility relation). And in fact, as soon as you believe something, you believe that you \emph{know} it. But one might shy away from saying that knowing something automatically amounts to knowing that you know it. For example, many are attracted to the idea that to know something requires that (i) that it is true, (ii) that you believe it, and (iii) that you are justified in believing it: the justified true belief analysis of knowledge. So, now couldn't it be that you know something, and thus (?) that you believe you know it, and thus that you believe that you are justified in believing it, but that you are not justified in believing that you are \emph{justified} in believing it? After all, one's source of knowledge, one's reliable means of acquiring knowledge, might be a mechanism that one has no insight into. So, while one can implicitly trust (believe) in its reliability, and while it is in fact reliable, one might not have any means to have trustworthy beliefs about it. [Further worries about the KK Thesis are discussed by \citet{williamson:2000:limits}.]

\subsection{*Symmetry}

What would the consequences be if the accessibility relation were symmetric? Symmetry of the accessibility relation $\mathcal{R}$ corresponds to the validity of the following principle:
\newpage
\ex. Brouwer's Axiom\marginnote{In modal logic notation: $p \rightarrow \square\diamondsuit p$, known simply as B in modal logic. The system that combines \textbf{T}/\textbf{M} with B is often called Brouwer's System (\textbf{B}), after the mathematician L.E.J. Brouwer, not because he proposed it but because it was thought that it had some connections to his doctrines.}:\\
$\forall p \forall w:\ w\in p \rightarrow \Bigl[\forall w' \bigl[ w\mathcal{R}w' \rightarrow \exists w'' \left[ w'\mathcal{R}w'' \&\ w''\in p\right]\bigr]\Bigr]$
%
% Brouwer photo from http://www-gap.dcs.st-and.ac.uk/~history/Mathematicians/Brouwer.html
% O. Becker in "Zur Logik der Modalitäten" (Jahrbuch für Philosophie und Phänomenologische Forschung, vol. 11 (1930), 497-548, called this Brouwer's axiom.)

\begin{figure}[htbp]
  \centering
    \includegraphics[height=1.5in]{symmetry.pdf}
  \caption{Symmetry}
  \label{fig:symmetry}
\end{figure}

\noindent Here's\marginpar{\centering\includegraphics[height=1in]{Brouwer.jpg}\\ {\tiny \href{http://plato.stanford.edu/entries/brouwer/}{L.E.J. Brouwer}}} the reasoning: Assume that $R$ is in fact symmetric. Pick a world $w$ in which $p$ is true. Now, could it be that the right hand side of the inference fails to hold in $w$? Assume that it does fail. Then, there must be some world $w'$ accessible from $w$ in which $\diamondsuit p$ is false. In other words, from that world $w'$ there is no accessible world $w''$ in which $p$ is true. But since $R$ is assumed to be symmetric, one of the worlds accessible from $w'$ is $w$ and in $w$, $p$ is true, which contradicts the assumption that the inference doesn't go through. So, symmetry ensures the validity of the inference.

The other way (validity of the inference requires symmetry): the inference says that from any $p$ world we only have worlds accessible from which there is at least one accessible $p$ world. But imagine that $p$ is true in $w$ but not true in any other world. So, the only way for the conclusion of the inference to hold automatically is to have a guarantee that $w$ (the only $p$ world) is accessible from any world accessible from it. That is, we need to have symmetry. QED.

To see whether a particular kind of attitude is based on a symmetric accessibility relation, we can ask whether Brouwer's Axiom is intuitively valid with respect to this attitude. If it is not valid, this shows that the accessibility relation can't be symmetric. In the case of a knowledge-based accessibility relation (epistemic accessibility), one can argue that \emph{symmetry does not hold}:\footnote{Thanks to Bob Stalnaker (pc to Kai von Fintel) for help with the following reasoning.}
\begin{quote}
  
  The symmetry condition would imply that if something happens to be true in the actual world, then you know that it is compatible with your knowledge (Brouwer's Axiom). This will be violated by any case in which your beliefs are consistent, but mistaken. Suppose that while $p$ is in fact true, you feel certain that it is false, and so think that you know that it is false. Since you think you know this, it is compatible with your knowledge that you know it. (Since we are assuming you are consistent, you can't both believe that you know it, and know that you do not). So it is compatible with your knowledge that you know that \expression{not} $p$. Equivalently\footnote{This and the following step rely on the duality of necessity and possibility: $q$ is compatible with your knowledge iff you don't know that \expression{not} $q$.}: you don't know that you don't know that \expression{not} $p$. Equivalently: you don't know that it's compatible with your knowledge that $p$. But by Brouwer's Axiom, since $p$ is true, you would have to know that it's compatible with your knowledge that $p$. So if Brouwer's Axiom held, there would be a contradiction. So Brouwer's Axiom doesn't hold here, which shows that epistemic accessibility is not symmetric.
\end{quote}

\noindent Game theorists and theoretical computer scientists who traffic in logics of knowledge often assume that the accessibility relation for knowledge is an equivalence relation (reflexive, symmetric, and transitive). But this is appropriate only if one abstracts away from any error, in effect assuming that belief and knowledge coincide. One\marginnote{All one really needs to make \textbf{NI} valid is to have a \term{Euclidean} accessibility relation: any two worlds accessible from the same world are accessible from each other. It is a nice little exercise to prove this, if you have become interested in this sort of thing. Note that all reflexive and Euclidean accessibility relations are transitive and symmetric as well \dash another nice little thing to prove.} striking consequence of working with an equivalence relation as the accessibility relation for knowledge is that one predicts the principle of \term{Negative Introspection} to hold:

\ex. \extitle{Negative Introspection (\textbf{ni})}\\
If one doesn't know that $p$, then one knows that one doesn't know that $p$. ($\neg\square p \rightarrow \square\neg\square p$).

This surely seems rather dubious: imagine that one strongly believes that $p$ but that nevertheless $p$ is false, then one doesn't know that $p$, but one doesn't seem to believe that one doesn't know that $p$, in fact one believes that one does know that $p$.

% \section{A Note on Shortcomings} \label{sec:note-shortcomings}
% 
% [ \dots to be written \dots stuff about \term{hyperintensionality} ]

%POSSIBLE EXERCISE: SHOW THAT NEGATION CANNOT BE TREATED AS SHIFTING US TO THE SET OF WORLDS WHERE THINGS ARE THE OPPOSITE OF THE WAY THEY ARE IN THE EVALUATION WORLD. OR IS THAT TOO CRAZY?

\section{Supplemental Readings}

{\setlength{\parindent}{0pt}\setlength{\parskip}{6pt}

%We will come back to propositional attitudes and especially the scope of noun phrases with respect to them, including the infamous \term{de dicto}-\term{de re} distinction.

A recent survey on attitudes:
\begin{bibentrylist}
	\item \bibentry{swanson:2009:hsk-attitudes}.
\end{bibentrylist}

Further connections between mathematical properties of accessibility relations and logical properties of various notions of necessity and possibility are studied extensively in modal logic:
\begin{bibentrylist}
  \item \bibentry{hughes-cresswell:1996:new}. 
  \item \bibentry{garson:2003:logic-modal}, especially section 7 and 8, ``Modal Axioms and Conditions on Frames'', ``Map of the Relationships between Modal Logics''. 
\end{bibentrylist}

A thorough discussion of the possible worlds theory of attitudes, and some of its potential shortcomings, can be found in Bob Stalnaker's work:
\begin{bibentrylist}
	\item \bibentry{stalnaker:1984:inquiry}. 
	\item \bibentry{stalnaker:1999:contextandcontent}. 
\end{bibentrylist}

A quick and informative surveys about the notion of knowledge:
\begin{bibentrylist}
	\item \bibentry{steup:2008:knowledge}. 
\end{bibentrylist}

Linguistic work on attitudes has often been concerned with various co-occurrence patterns, particularly which moods (indicative or subjunctive or infinitive) occur in the complement and whether negative polarity items are licensed in the complement.

Mood licensing:
\begin{bibentrylist}
	\item \bibentry{portner:1997:mood}. 
\end{bibentrylist}

NPI-Licensing:
\begin{bibentrylist}
	\item \bibentry{kadmon-landman:1993:any}. 
	\item \bibentry{fintel:1999:npi}.
  \item \bibentry{giannakidou:1999:affective}. 
\end{bibentrylist}

There is some interesting work out of Amherst rethinking the way attitude predicates take their complements:

\begin{bibentrylist}
  \item \bibentry{kratzer:2006:decomposing}.
  \item \bibentry{moulton:2008:wager}. \url{http://people.umass.edu/keir/Wager.pdf}.
  \item \bibentry{moulton:2009:thesis}.
\end{bibentrylist}

Tamina Stephenson in her MIT dissertation and related work explores the way attitude predicates interact with epistemic modals and taste predicates in their complements:

\begin{bibentrylist}
  \item \bibentry{stephenson:2007:judge-lp}.
  \item\bibentry{stephenson:2007:thesis}.
\end{bibentrylist}

Jon Gajewski in his MIT dissertation and subsequent work explores the distribution of the \term{neg-raising} property among attitude predicates and traces it back to presuppositional components of the meaning of the predicates:

\begin{bibentrylist}
	\item \bibentry{gajewski:2005:thesis}.
	\item \bibentry{gajewski:2007:neg-raising}.
\end{bibentrylist}

Interesting work has also been done on presupposition projection in attitude contexts:
\begin{bibentrylist}
	\item \bibentry{asher:1987:attitudes}.
	\item \bibentry{heim:1992:attitude}. 
	\item \bibentry{geurts:1998:attitudes}. 
\end{bibentrylist}

}

% chapter propositional_attitudes (end)

