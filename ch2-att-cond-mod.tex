\chapter{Attitudes, conditionals, modals}
\label{cha:att-cond-mod}

\minitoc

\section*{Introduction}

\note{%
  \(M\ [f(a)]\ (\phi)\)
  \begin{description}
  \item[\(M\):] a quantificational/modal relation between two sets of worlds
    (propositions)
  \item[\(a\):] the anchor of the modal claim
  \item[\(f\):] the flavor function that projects a set of worlds from the
    anchor
  \item[\(\phi\):] the prejacent set of worlds (proposition)
  \end{description}
} %
Towards the end of the first chapter, we identified a general schema for modal
displacement operators. It begins with a ``flavor function'' that ``projects'' a
set of relevant worlds from an ``anchor'', and then a quantificational claim is
made about those worlds and their relation to the prejacent. We will now see
this pattern at work in three kinds of constructions: propositional attitudes,
conditionals, and modals. These look superficially quite dissimilar:

\pex
\a Charlotte believes that Lucy is smart.
\a If Lucy is smart, she will cancel the meeting.
\a Lucy might cancel the meeting.
\xe
%
The intensional operator in \Last[a] is the lexical verb \expression{believe},
in \Last[b] it is the subordinating complementizer \expression{if}, and in
\Last[c] it is the auxiliary verb \expression{might}. Despite this surface
variety, the core semantic contributions are very similar.

Propositional attitude predicates like \expression{believe} project a set of
worlds from the mental state of their subject and relate those worlds to the
worlds described by the prejacent proposition. Conditionals select a relevant
subset of the worlds described by their antecedent and relate them to the worlds
described by their consequent. And the modal \expression{might} says that
some worlds in a relevant set make the prejacent proposition true.

We will now look at these ideas in more detail and build some nimbleness in
deploying the technical notions here.

\section{Attitudes}
\label{sec:attitudes}

We move from the contents of works of fiction to the contents of the minds of
people. Instead of worlds compatible with the Sherlock Holmes stories, we look
at the worlds compatible with the beliefs or other mental states of a person.

\subsection{Hintikka's idea}
\label{subsec:hintikka}

\note{According to \cite{hintikka-1969-attitudes}, the term \term{propositional
    attitude} goes back to \cite{russell-1940-inquiry}.}%
Expressions like \expression{believe}, \expression{know}, \expression{doubt},
\expression{expect}, \expression{want}, \expression{regret}, and so on are
usually said to describe \term{propositional attitudes}, expressing relations
between individuals (the attitude holder) and propositions (intensions of
sentences).

\note{The possible worlds semantics for propositional attitudes was in place
  long before the extension to fiction contexts was proposed. Our discussion
  here has inverted the historical sequence for pedagogical purposes.}%
The simple idea is that \expression{Amandine believes that Letícia is a spy}
claims that Amandine believes of the proposition that Letícia is a spy that it
is true. Note that for the attitude ascription to be true it does not have to
hold that Letícia is actually a spy. But where \dash in which world(s) \dash
does Letícia have to be a spy for it be true that Amandine believes that Letícia
is a spy?

\marginfig{accordingtoMrRogers.jpg}%
We might want to be inspired by the colloquial phrase ``the world according to
\dots'' and say that \expression{Amandine believes that Letícia is a spy} is
true iff in the world according to Amandine, Letícia is a spy. We immediately
recall from Chapter 1 that we need to fix this idea up by making space for
multiple worlds compatible with Amandine's beliefs and by tying the
truth-conditions to contingent facts about the evaluation world. That is, what
Amandine believes is different in different possible worlds.

\kwn The following lexical entry thus offers itself:

\ex $\svt{believe}^{w,g} =$\\\ \null\hfill$\lambda p_{\type{s,t}}.\ \lambda x. \
\forall w' \mbox{ compatible with } x's \mbox{ beliefs in } w\co p(w') = 1.$ \xe

\note{It is important to realize the modesty of this semantics: we are not
  trying to figure out what belief systems are and particularly not what their
  internal workings are. That is the job of psychologists (and philosophers of
  mind, perhaps). For our semantics, we treat the belief system as a black box
  that determines for each possible world whether it considers it possible that
  it is the world it is located in.}%
What is going on in this semantics? We conceive of Amandine's beliefs as a state
of her mind about whose internal structure we will remain agnostic, a matter
left to other cognitive scientists. What we require of it is that it embody
opinions about what the world she is located in looks like. In other words, if
her beliefs are confronted with a particular possible world $w'$, they will
determine whether that world may or may not be the world as they think it is.
What we are asking of Amandine's mental state is whether any state of affairs,
any event, anything in $w'$ is in contradiction with anything that Amandine
believes. If not, then $w'$ is compatible with Amandine's beliefs. For all
Amandine believes, $w'$ may well be the world where she lives. Many worlds will
pass this criterion, just consider as one factor that Amandine is unlikely to
have any precise opinions about the number of leaves on the tree in front of my
house. Amandine's belief system determines a set of worlds compatible with her
beliefs: those worlds that are viable candidates for being the actual world, as
far as her belief system is concerned.

Now, Amandine believes a proposition iff that proposition is true in all of the
worlds compatible with her beliefs. If there is just one world compatible with
her beliefs where the proposition is not true, that means that she considers it
possible that the proposition is not true. In such a case, we can't say that she
believes the proposition.

\kwn 
Here is the same story in the words of \citet{hintikka-1969-attitudes}, the
source for this semantics for propositional attitudes:

\begin{quotation}
  \marginfig{hintikka.jpg}%
  My basic assumption (slightly simplified) is that an attribution of any
  propositional attitude to the person in question involves a division of all
  the possible worlds (\dots) into two classes: into those possible worlds which
  are in accordance with the attitude in question and into those which are
  incompatible with it. The meaning of the division in the case of such
  attitudes as knowledge, belief, memory, perception, hope, wish, striving,
  desire, etc. is clear enough. For instance, if what we are speaking of are
  (say) $a$'s memories, then these possible worlds are all the possible worlds
  compatible with everything he remembers. [\dots]
	
	How are these informal observations to be incorporated into a more explicit
  semantical theory? According to what I have said, understanding attributions
  of the propositional attitude in question (\dots) means being able to make a
  distinction between two kinds of possible worlds, according to whether they
  are compatible with the relevant attitudes of the person in question. The
  semantical counterpart to this is of course a function which to a given
  individual person assigns a set of possible worlds.
	
	However, a minor complication is in order here. Of course, the person in
  question may himself have different attitudes in the different worlds we are
  considering. Hence %
  \note{We recognize our schema: the anchor is the pair of an individual and a
    world, the flavor function projects from the anchor a set of worlds
    compatible with the relevant attitude of the anchor individual in the anchor
    world.}%
  \note{Note that what Hintikka calls ``alternativeness relations'' are now
    more commonly known as \term{accessibility relations}. We will encounter
    such relations again and again, and we will explore their formal properties
    later on.}%
  this function in effect becomes a relation which to a given individual and to
  a given possible world $\mu$ associates a number of possible worlds which we
  shall call the \term{alternatives} to $\mu$. The relation will be called the
  alternativeness relation. (For different propositional attitudes, we have to
  consider different alternativeness relations.)
\end{quotation}

\begin{exercise}
	
	Let's adopt Hintikka's idea that we can use a function that maps $x$ and $w$
  into the set of worlds $w'$ compatible with what $x$ believes in $w$. Call
  this function $\mathcal{B}$. That is,
	
	\ex $\mathcal{B} = \lambda x.\ \lambda w.\ \{w'\co w' \mbox{ is compatible
    with } x \mbox{'s beliefs in } w\}.$ \xe
%	
	Using this notation, our lexical entry for \expression{believe} could look as
  follows:
	
	\ex $\svt{believe}^{w,g} = \lambda p_{\type{s,t}}.\ \lambda x.\
  \mathcal{B}(x)(w) \subseteq p.$ \xe
%	
	We are here indulging in the usual sloppiness in treating $p$ both as a
  function from worlds to truth-values and as the set characterized by that
  function.
	
	Here now are two ``alternatives'' for the semantics of \expression{believe}:
	
	\ex \extitle{Attempt 1 (very wrong)}\\[3pt]
	$\svt{believe}^{w,g} = \lambda p \in D_{\type{s,t}}. \big[ \lambda x
  \in D.\ p = \mathcal{B}(x)(w) \big]$. \xe
	
	\ex~ \extitle{Attempt 2 (also very wrong)}\\[3pt]
	$\svt{believe}^{w,g} = \lambda p \in D_{\type{s,t}}. \big[ \lambda x
  \in D.\ p \cap \mathcal{B}(x)(w) \neq \emptyset \big]$. \xe
%	
	Explain why these do not adequately capture the meaning of
  \expression{believe}. \qed
\end{exercise}
%
\begin{exercise}\label{exercise:existential-attitudes}
  Follow-up: The semantics in \Last would have made \emph{believe} into an
  existential quantifier of sorts: it would say that \emph{some} of the worlds
  compatible with what the subject believes are such-and-such. You have argued
  (successfully, of course) that such an analysis is wrong for \emph{believe}. %
  \note{If you can't find any candidates that survive scrutiny, can you
    speculate why there might be no existential attitude predicates? [Warning:
    this is underexplored territory!]}%
  But \emph{are} there attitude predicates with such an ``existential'' meaning?
  Discuss some candidates. \qed
\end{exercise}

\begin{exercise}%
  \note{After this little exercise, you might be interested in some really tough
    questions about intensionality inside noun phrases:
    \cite{bogal-albritten-2013-epistemic-adverbs,
      bogal-albritten-weir-2017-collins} and \cite[especially Chapter
    4]{hirsch-2017-thesis}.}%
  Propose a semantics for the adjective \emph{alleged} as in \emph{Vera is an
    alleged kleptomaniac}. Do not assume any hidden structure. Try to relate
  your semantics to the verb \emph{allege} as in \emph{Romelu alleged that Vera
    is a kleptomaniac}. \qed
\end{exercise}

\subsection{Iterated attitudes}
\label{subsec:iterated-attitudes}

\marginfig{madscientist.jpeg}%
Semantics is a lab science in several ways. The most crucial way is that we
learn a lot about our objects of study when we put them together and see how
they react to each other.

We expect attitudes to be able to \emph{iterate}: an attitude claim is a
sentence with contingent truth-conditions and thus provides a proposition that
in turn could be the complement of another attitude claim. In fact, one suspects
that much of the fabric of human life involves iterated attitudes: we wonder
whether Emma realizes that Caroline believes that Janet has invited Preston for
dinner without telling Abby.

\begin{exercise}
  Derive the truth-conditions of \emph{Emma believes that Caroline believes that
    Preston is from Texas}. [Follow-up question: if Emma's belief is correct,
  does that mean that Preston is from Texas?] \qed
\end{exercise}

\clearpage
\section{Conditionals}
\label{sec:conditionals}

In many ways, conditionals are the archetypal construction of displacement: the
consequent is evaluated not against the actual here and now but against the
scenario conjured up by the antecedent. Consider a few conditional sentences:

\pex
\a If Kim left before 6am, she got there in time.
\a If there's an earthquake tomorrow, this house will collapse.\label{ex:earthquake}
\a If there had been a massive snowstorm last night, Kai would have stayed home.
\xe
%
These represent the three main subtypes of conditionals (there are more):
(\lastx a) is an ``indicative'' conditional about the past, (\lastx b) is an
indicative conditional about the future, and (\lastx c) is a ``subjunctive''
conditional. For the moment, the differences will be left aside.

The basic idea of how conditionals work is this: the \expression{if}-clause
whisks us away to a particular possible world (or maybe a set thereof) and the
consequent clause is asserted to be true of that world (or those worlds). But
what world(s) are we being taken to? The most obvious requirement is that the
antecedent of the conditional needs to be true of the world(s). But there's
more.

\note{Lewis used a rather whimsical example to start off his seminal
  \citeyear{lewis-1973-counterfactuals} book on counterfactuals: ``If kangaroos
  had not tails, they would topple over''. For another example of counterfactual
  whimsy, consider this scene from the TV show ``Big Bang Theory'':
  \url{https://www.youtube.com/watch?v=0lpY0Kt4bn8}. As the examples in the text
  make clear, conditionals are actually very down-to-earth in real life.}%
\marginfig{kangaroo-upright.jpg}%
\marginfig{kangaroo-toppled.jpg}%
Given our discussions of how the semantics of fiction operators anchors them in
facts about the actual world (the content of the relevant body of fiction) and
how the semantics of attitude predicates is anchored in the mental states of an
individual in the actual world, it shouldn't come as a surprise that
conditionals are similarly anchored. So, look at the examples in (\lastx): what
in the actual world are they about?

Here's a first attempt of an answer: (\lastx a) is about the local
transportation system, the weather, the traffic, and so on. (\lastx b) is about
the sturdiness of this house, facts of geology, laws of physics, and so on.
(\lastx c) is about Kai's proclivities (such as avoiding traffic snarls), the
local climate, and so on. Since the conditionals are anchored in real world
facts, they are no mere flights of fancy and whether they are true depends on
those facts. If today's traffic was particularly bad, it may be false that Kim's
leaving before 6am would have got her there in time. If the architects went to
great lengths to make the house earthquake-safe, (\lastx b) may well be false.
And if there was an attendance-mandatory faculty meeting, Kai may well have come
in in spite of a massive snowstorm.

So, the outlines of the semantics of conditionals are clear: \expression{if}
takes us to worlds where the antecedent is true but that match the actual world
in certain relevant features. And the consequent then is evaluated in those
worlds. There are many details to work out and we'll keep returning to that
task. But for now, we put forward a placeholder analysis.

We will treat \expression{if} as a higher-order operator that together with the
antecedent creates an intensional operator with a semantics very similar to the
final analysis we gave to \expression{in the world of Sherlock Holmes}. But
where the fiction operator directly encoded what features of the actual world
it's sensitive to (the Sherlock Holmes fiction), conditionals rely on context
for this job. Here's a first draft of the proposal:

\ex\label{ex:if-strict-context}%
$\svt{if}^{w,g} = \lambda p \in D_{\type{s,t}}.\ \lambda q \in
D_{\type{s,t}}.$ \\
\hfill $\forall w'\co p(w')=1\ \&\ w' \text{ is relevantly like } w \rightarrow
q(w')=1.$ \xe
%
The contextual anchoring to features of the evaluation world $w$ is here
effected by the placeholder ``relevantly like $w$''. This is crucial because
otherwise the conditional would talk about any world whatsoever where the
antecedent is true. This would make the truth-conditions not just not contingent
on the actual world but also far too strong to allow most sensible conditionals
to be true ever.

Think about the earthquake conditional \refx{ex:earthquake}: we would derive the
absurdly strong truth-conditions that the conditional is true iff \emph{all} of
the worlds where there is a major earthquake in Cambridge tomorrow are worlds
where my house collapses.

There are some obvious and immediate problems with this analysis. For one, while
it's easy to imagine circumstances where the conditional \refx{ex:earthquake} is
judged to be true, there surely are possible worlds where there's an earthquake
but my house does not collapse: perhaps, the builders in that world used all the
recommended best practices to make the building earthquake-safe, perhaps it's a
world where I'm simply unreasonably lucky, or the house is immediately adjacent
to much sturdier neighboring buildings which keep it propped up, or Harry Potter
flies by and protects the house at the last minute (he owes me a favor, after
all). This problem (that the house doesn't in fact collapse in \emph{all}
possible worlds where there's an earthquake but that the conditional can still
be judged true in some worlds) is accompanied with another problem: whether the
conditional is true depends on what the world is like. Was the house built to
exacting standards? Is it propped up by its neighbors? Does Harry Potter owe me
a favor? That is the problem solved by restricting the quantifier over worlds to
world ``relevantly like $w$''.

\enlargethispage{24pt}
Obviously, this is a semantics with a ``placeholder'', because what does
``relevantly like'' mean precisely? Now, just because the semantics is therefore
rather vague and context-dependent doesn't mean it is wrong. As
\cite[p.1]{lewis-1973-counterfactuals} writes:

\begin{quote} Counterfactuals are notoriously vague. That does not mean that we
  cannot give a clear account of their truth conditions. It does mean that such
  an account must either be stated in vague terms \dash which does not mean
  ill-understood terms \dash or be made relative to some parameter that is fixed
  only within rough limits on any given occasion of language use.
\end{quote}
%
\clearpage
\note{\cite{strawson-1950-referring} famously wrote: ``Neither Aristotelian nor
  Russellian rales give the exact logic of any expression of ordinary language;
  for ordinary language has no exact logic.''}%
The insight articulated by Lewis here is very important. Applying mathematical
or logical methods to analyzing natural language meaning often arouses severe
skepticism, precisely because natural language is often vague and
context-dependent. But that just means that an adequate analysis needs to not
ignore vagueness and context-dependence and rather be clear about where they
enter.

All the more reason to refine our initial draft of the proposal. We put a
placeholder for context-dependence in the meta-language (``worlds relevantly
like the evaluation world'') but that is not really sufficient. We would like to
embed the analysis in a general framework for how context enters the semantics.
For the purposes of this book, we will adopt an approach that generalizes from
the analysis of ``free'' pronouns in the Heim \&\ Kratzer textbook.

In H\&K, chapters 9--11, a technical implementation of context-dependency is
developed for pronouns and their referential (and E-Type) readings. Referential
pronouns are analyzed there as free variables, appealing to a general principle
that free variables in an LF need to be supplied with values from the utterance
context. If we want to describe the context-dependency of conditionals (and as
we'll soon see, modals) in a technically analogous fashion, we can think of
their LF-representations as incorporating or subcategorizing for a kind of
invisible pronoun, a free variable that effects the anchoring of the conditional
claim to relevant features of the evaluation world.

\note{We are using the notation for variables of types other than $e$ introduced
  by Heim \&\ Kratzer, p. 213. An index on a variable now is an ordered pair of
  a natural number and a type. The variable assignments relative to which we
  calculate semantic values now are functions from ordered pairs of a natural
  number and a type to elements of the domain of objects of that type.}%
Concretely, we posit LF-structures where \emph{if} doesn't just take two
propositions as its arguments but also an object language variable of type
\type{s,\type{s,t}}:

\ex
\begin{forest}
baseline,
sn edges,
for tree={s sep=10mm, inner sep=0, l=0}
[{}, my pretty nice empty nodes
[[[if][$f_{\type{3,\type{s,st}}}$]][$\phi_{\text{antecedent}}$]]
[$\psi_{\text{consequent}}$]]
\end{forest}
\xe
%
We have written the silent pronoun as ``$f$'' to remind you of ``flavor'', the
evocative term we used to describe the anchoring of intensional operators. So,
we have a variable over flavor functions that will return a set of worlds when
applied to a given world. We will give \emph{if} the job of telling $f$ that
what we need is the set of worlds that $f$ assigns to the evaluation world:

\ex\label{ex:if-strict-context-formal}%
$\svt{if}^{w,g} = \lambda f \in D_{\type{s,\type{s,t}}}.\ \lambda p \in
D_{\type{s,t}}.\ \lambda q \in D_{\type{s,t}}.$ \\
\hfill $\forall w'\co f(w)(w') = 1\ \&\ p(w')=1 \rightarrow
q(w')=1.$
\xe
%
Together this means that a conditional says about the evaluation world $w_{0}$
that among the worlds that are $f$-related to $w_{0}$, the ones where the
antecedent is true are all worlds where the consequent is also true.

We get different flavors of conditionals from different contextual resolutions
of $f$. Consider for example the earthquake conditional in \refx{ex:earthquake}.
Context might assign to $f$ the function that when given an evaluation world $w$
returns the set of worlds that ``agree'' with $w$ on how sturdy this house is,
what the local geology is like, and what the laws of physics are. Then, the
conditional claims that the actual world is such that all the worlds that agree
with it via $f$ and where there is an earthquake are worlds where this house
collapses. 

With context-dependency comes the spectre of indeterminacy. A famous example is
a pair attributed by \cite[221]{quine-1960-word} to Nelson Goodman (imagine
these being said while the Korean War was going on):

\marginfig{mushroom.jpg}%
\marginfig{catapult.png}%
\pex
\a If Caesar were in command, he would use the atom bomb.
\a If Caesar were in command, he would use catapults.
\xe
%
Both versions seem possible: saying \Last[a] would talk about worlds where
Caesar with all his ruthlessness is in command, while \Last[b] would talk about
worlds where Caesar's own arsenal comes with him.

We will return to conditionals in the next chapter when some additional
complications become necessary.

\begin{exercise}
  We continue our experiments and embed a conditional under an attitude. Compute
  the truth-conditions of the following sentence:

  \ex\label{ex:embedded-if}
  Allie believes that if there's an earthquake, this house will collapse.
  \xe
\end{exercise}

\begin{exercise}
  Here's a potentially simpler analysis of conditionals. What if the silent
  pronoun were of type \type{s,t}, a set of worlds? \emph{If}
  would then take the contextually assigned set of worlds and say that all
  the worlds in that set in which the antecedent is true are worlds where the
  consequent is true. Spell out the details of this idea. How would it fare when
  confronted with the embedded conditional \refx{ex:embedded-if} from the
  previous exercise? \qed 
\end{exercise}


\clearpage
\section{Modals}
\label{sec:modals}

The final empirical addition of this chapter are modal auxiliaries like
\expression{may, must, can, have to}, etc. Most of what we say here should carry
over straightforwardly to modal adverbs like \expression{maybe, possibly,
  certainly}, etc. We will make certain syntactic assumptions, which make our
work easier but which leave aside many questions that at some point deserve to
be addressed.

\subsection{Syntactic assumptions} \label{sec:synt-assumpt-1}

% \note{The issue of raising vs. control will probably be taken up later. If you
% are eager to get started on it and other questions of the morphosyntax of
% modals, read the handout from an LSA class Sabine and Kai taught some years
% ago: \url{http://web.mit.edu/fintel/lsa220-class-2-handout.pdf}.}
We will assume, at least for the time being, that a modal like \expression{may}
is a \term{raising} predicate (rather than a \term{control} predicate), i.e.,
its subject is not its own argument, but has been moved from the
subject-position of its infinitival complement. So, we are dealing with the
following kind of structure:

\pex
\a Ann may be smart. 
\a\null [ Ann [ $\lambda_1$ [ may [ t$_1$ be smart ]]]]
\xe

Actually, we will be working here with the even simpler structure below, in
which the subject has been reconstructed to its lowest trace position. (E.g.,
these could be generated by deleting all but the lowest copy in the movement
chain.\note{We will talk about reconstruction in more detail later.}) We will be
able to prove that movement of a name or pronoun never affects truth-conditions,
so at any rate the interpretation of the structure in (\lastx b) would be the
same as that of (\nextx). As a matter of convenience, %
\note{We will assume that even though \expression{Ann be smart} is a non-finite
  sentence, this will not have any effect on its semantic type, which is that of
  a sentence, which in turn means that its semantic value is a truth-value. This
  is hopefully independent of the (interesting) fact that \expression{Ann be
    smart} on its own cannot be used to make a truth-evaluable assertion.}%
then, we will take the reconstructed structures, which allow us to abstract away
from the (here irrelevant) mechanics of variable binding.

\ex may [ Ann be smart ] \xe

So, for now at least, we are assuming that modals are expressions that take a
full sentence as their semantic argument. Now then, what do modals mean?

\subsection{Quantification over possible worlds} \label{sec:quant-over-poss}

\note{This idea goes back a long time. It was famously held by Leibniz,
  but there are precedents in the medieval literature; see
  \cite{knuuttila-2003-modality-medieval}. See \cite{copeland-2002-genesis}
  for the modern history of the possible worlds analysis of modal expressions.}%
The basic idea of the possible worlds semantics for modal expressions is that they
are quantifiers over possible worlds. Toy lexical entries for \expression{must}
and \expression{may}, for example, would look like this:

\ex $\sv{\mbox{must}}^{w,g} = \lambda p_{\type{s,t}}.\ \forall w'\co p(w') =
1$. \xe

\ex~ $\sv{\mbox{may}}^{w,g} = \lambda p_{\type{s,t}}.\ \exists w'\co p(w') =
1$. \xe
%
\note{Sometimes, people call necessity modals ``universal modals'' and
  possibility modals ``existential modals'', which obviously presupposes this
  quantificational analysis.}%
A necessity modal like \emph{must} says that all worlds make its prejacent true,
while a possibility modal like \emph{may} says that some worlds make its
prejacent true. Note that our previous intensional operators were all universal
quantifiers (unless you found some existential attitudes in
Exercise~\ref{exercise:existential-attitudes}), so the existential force of
\emph{may} is a new frontier for us.

The analysis in \LLast/\Last is too crude (in particular, notice that it would
make modal sentences non-contingent \dash there is no occurrence of the
evaluation world on the right hand side!). But it does already have some
desirable consequences that we will seek to preserve through all subsequent
refinements. It correctly predicts a number of intuitive judgments about the
logical relations between \expression{must} and \expression{may} and among
various combinations of these items and negations. To start with some elementary
facts, we feel that \expression{must} $\phi$ entails \expression{may} $\phi$,
but not vice versa:

\ex You must stay.\\
Therefore, you may stay. \hfill\textsc{valid} \xe

\ex~ You may stay.\\
Therefore, you must stay. \hfill\textsc{invalid} \xe

\note{The somewhat stilted \expression{it is not the case}-construction is used
  in (\ref{ex:stilted}a) to make certain that negation takes scope over
  \expression{must}. When modal auxiliaries and negation are together in the
  auxiliary complex of the same clause, their relative scope seems not to be
  transparently encoded in the surface order; specifically, the scope order is
  not reliably negation $\succ$ modal. (Think about examples with
  \expression{mustn't}, \expression{can't, shouldn't, may not} etc. What's going
  on here? This is an interesting topic which we must set aside for now. See the
  references at the end of the chapter for relevant work.) With modal
  \emph{main} verbs (such as \expression{have to}), this complication doesn't
  arise; they are consistently inside the scope of clause-mate auxiliary
  negation. Therefore we can use (\ref{ex:stilted}b) to (unambiguously) express
  the same scope order as (\ref{ex:stilted}a), without having to resort to a
  biclausal structure.}%
\pex~\label{ex:stilted} \a You may stay, but it is not the case that you must
stay. \a You may stay, but you don't have to stay. \hfill\textsc{consistent} \xe

We judge \expression{must} $\phi$ incompatible with its ``inner negation''
\expression{must} [\expression{not} $\phi$ ], but find \expression{may} $\phi$
and \expression{may} [\expression{not} $\phi$ ] entirely compatible:

\ex You must stay, and/but also, you must leave. (leave = not stay).\\
\hfill\textsc{contradictory} \xe

\ex~ You may stay, but also, you may leave. \hfill\textsc{consistent} \xe

We also judge that in each pair below, the (a)-sentence and the (b)-sentences
say the same thing.

\pex\label{ex:must} \a You must stay. \a It is not the case that you may leave.\\
You aren't allowed to leave.\\
(You may not leave.)%
\note{The parenthesized variants of the (b)-sentences in \refx{ex:must} are
  pertinent here only to the extent that we can be certain that negation scopes
  over the modal. In these examples, apparently it does, but as we remarked
  above, this cannot be taken for granted in all structures of this form.}
\\
(You can't leave.) \xe

\pex~ \a You may stay. \a It is not the case that you must leave.\\
You don't have to leave.\\
You don't need to leave.\\
(You needn't leave.)
\xe

\note{In logicians' jargon, \expression{must} and
  \expression{may} behave as \term{duals} of each other. For definitions of
  ``dual'', see \cite[197]{barwise-cooper-1981-generalized} or
  \cite[vol.2,238]{gamut:91}.}%
Given that \expression{stay} and \expression{leave} are each other's negations
(i.e. \exts{leave} = \exts{not stay}, and \exts{stay} = \exts{not leave}), the
LF-structures of these equivalent pairs of sentences can be seen to instantiate
the following schemata:

\pex \a \expression{must} $\phi$ $\equiv$ \expression{not} [\expression{may}
[\expression{not} $\phi$]] \a \expression{must} [\expression{not}
\ensuremath{\psi}] $\equiv$ \expression{not} [\expression{may}
\ensuremath{\psi}] \xe

\pex~ \a \expression{may} $\phi$ $\equiv$ \expression{not} [\expression{must}
[\expression{not} $\phi$]] \a \expression{may} [\expression{not}
\ensuremath{\psi}] $\equiv$ \expression{not} [\expression{must}
\ensuremath{\psi}] \xe

\note{More linguistic data regarding the ``parallel logic'' of modals and
  quantifiers can be found in Horn's dissertation
  \parencite{horn-1972-dissertation}.}%
Our present analysis of \expression{must}, \expression{have-to}, \dots{} as
universal quantifiers and of \expression{may}, \expression{can}, \dots{} as
existential quantifiers straightforwardly predicts all of the above judgments,
as you can easily prove.

% \pex \a $\forall x \phi \equiv \neg \exists x \neg \phi$ \a $\forall x \neg
% \phi \equiv \neg \exists x \phi$ \xe

% \pex \a $\exists x \phi \equiv \neg \forall x \neg \phi$ \a $\exists x \neg \phi
% \equiv \neg \forall x \phi$ \xe

\subsection{Contingency, flavors, context-dependency} \label{sec:contingency}

We already said that the semantics we started with is too simple-minded. In
particular, we have no dependency on the evaluation world, which would make
modal statements non-contingent. This is not correct.

If one says \expression{It may be snowing in Cambridge}, that may well be part
of useful, practical advice about what to wear on your upcoming trip to
Cambridge. It may be true or it may be false. The sentence seems true if said in
the dead of winter when we have already heard about a Nor'Easter that is
sweeping across New England. The sentence seems false if said by a clueless
Australian acquaintance of ours in July.

The contingency of modal claims is not captured by our current semantics. All
the \expression{may}-sentence would claim under that semantics is that there is
some possible world where it is snowing in Cambridge. And surely, once you have
read Lewis' quote in Chapter 1, where he asserts the existence of possible
worlds with different physical constants than we enjoy here, you must admit that
there have to be such worlds even if it is July. The problem is that in our
semantics, repeated here:

\ex $\sv{\mbox{may}}^{w,g} = \lambda p_{\type{s,t}}.\ \exists w'\co p(w') =
1$, \xe
%
\note{Conversely, the plenitude of possible worlds would make
  \expression{must}-claims very likely false if they are not reigned in or
  anchored somehow.}%
there is no occurrence of $w$ on the right hand side. This means that the
truth-conditions for \expression{may}-sentences are world-independent. In other
words, they make non-contingent claims that are either true whatever or false
whatever, and because of the plenitude of possible worlds they are more likely
to be true than false. This needs to be fixed. But how?

Well, what makes \expression{it may be snowing in Cambridge} seem true when we
know about a Nor'Easter over New England? What makes it seem false when we know
that it is summer in New England? The idea is that we only consider possible
worlds \term{compatible with the evidence available to us}. And since what
evidence is available to us differs from world to world, so will the truth of a
\expression{may}-statement.

\ex \note{From now on, we will leave off type-specifications such as that $p$
  has to be of type $\type{s,t}$, whenever it is obvious what they should be and
  when saving space is aesthetically called for.}%
$\sv{\mbox{may}}^{w,g} = \lambda p.\ \exists w' \mbox{ compatible w/ the
  evidence in } w\co p(w') = 1$. \xe

\ex~ $\sv{\mbox{must}}^{w,g} = \lambda p.\ \forall w' \mbox{ compatible w/ the
  evidence in } w\co p(w') = 1$. \xe

\kwn
Let us consider a different example:

\ex You have to be quiet. \xe
%
Imagine this sentence being said based on the house rules of the particular
dormitory you live in. Again, this is a sentence that could be true or could be
false. Why do we feel that this is a contingent assertion? Well, the house rules
can be different from one world to the next, and so we might be unsure or
mistaken about what they are. In one possible world, they say that all noise
must stop at 11pm, in another world they say that all noise must stop at 10pm.
Suppose we know that it is 10:30 now, and that the dorm we are in has either one
or the other of these two rules, but we have forgotten which. Then, for all we
know, \expression{you have to be quiet} may be true or it may be false. This
suggests a lexical entry along these lines:

\ex $\sv{\mbox{have-to}}^{w,g} = \lambda p.\ \forall w' \mbox{ compatible with
  the rules in } w\co p(w') = 1$. \xe

Again, we are tying the modal statement about other worlds down to certain
worlds that stand in a certain relation to the actual world: those worlds where
the rules as they are here are obeyed.

A note of caution: it is very important to realize that the worlds compatible
with the rules as they are in $w$ are those worlds where nothing happens that
violates any of the $w$-rules. This is not at all the same as saying that the
worlds compatible with the rules in $w$ are those worlds where the same rules
are in force. Usually, the rules do not care what the rules are, unless the
rules contain some kind of meta-statement to the effect that the rules have to
be the way they are, i.e. that the rules cannot be changed. So, in fact, a world
$w'$ in which nothing happens that violates the rules as they are in $w$ but
where the rules are quite different and in fact what happens violates the rules
as they are in $w'$ is nevertheless a world compatible with the rules in $w$.
For example, imagine that the only relevant rule in $w$ is that students go to
bed before midnight. Take a world $w'$ where a particular student goes to bed at
11:30 pm but where the rules are different and say that students have to go to
bed before 11 pm. Such a world $w'$ is compatible with the rules in $w$ (but of
course not with the rules in $w'$).

Apparently, there are different flavors of modality, varying in what kind of
facts in the evaluation world they are sensitive to. The semantics we gave for
\expression{must} and \expression{may} above makes them talk about evidence,
while the semantics we gave for \expression{have-to} made it talk about rules.
But that was just because the examples were hand-picked. In fact, in the dorm
scenario we could just as well have said \expression{You must be quiet}. And,
vice versa, there is nothing wrong with using \expression{it has to be snowing
  in Cambridge} based on the evidence we have. In fact, many modal expressions
seem to be multiply ambiguous. The English modal \emph{have to} is probably the
world champion in this regard:

\pex\label{ex:poly-have}
\a It has to be raining.
\a Visitors have to leave by six pm.
\a You have to go to bed in ten minutes.
\a I have to sneeze.
\a To get home in time, you have to take a taxi.
\xe

\note{Beyond ``epistemic'' and ``deontic,'' there is a great deal of
  terminological exuberance. Sometimes all non-epistemic readings are grouped
  together under the term \term{root modality} (nobody knows why).}%
Traditional descriptions of modals often distinguish a number of ``readings'':
\term{epistemic, deontic, ability, circumstantial, dynamic,} \dots. Here are
some initial illustrations.

\ex \label{epist}\extitle{Epistemic Modality}\\[6pt]
A: Where is John?\\
B: I don't know. He \expression{may} be at home. \xe

\ex \extitle{Deontic Modality}\\[6pt]
A: Am I allowed to stay over at Janet's house?\\
B: No, but you \expression{may} bring her here for dinner. \xe

\ex \extitle{Circumstantial/Dynamic Modality}\\[6pt]
A: I will plant the rhododendron here.\\
B: That's not a good idea. It \expression{can} grow very tall. \xe

How are \expression{may} and \expression{can} interpreted in each of these
examples? What do the interpretations have in common, and where do they differ?

In all three examples, the modal makes an existentially quantified claim about
possible worlds. This is usually called the \term{modal force} of the claim.
What differs is what worlds are quantified over, sometimes called the
\term{modal flavor}. In \term{epistemic} modal sentences, we quantify over
worlds compatible with the available evidence. In \term{deontic} modal
sentences, we quantify over worlds compatible with the rules and/or regulations.
And in the \term{circumstantial} modal sentence, we quantify over the set of
worlds which conform to the laws of nature (in particular, plant biology). What
speaker B in (\lastx) is saying, then, is that there are some worlds conforming
to the laws of nature in which this rhododendron grows very tall.

\subsection{Epistemic vs. Circumstantial}

\note{In the earlier \cite{kratzer-1981-notional}, the hydrangeas were
    \expression{Zwetschgenbäume} `plum trees'. The German word
    \expression{Zwetschge}, by the way, is etymologically derived from the name
    of the city Damascus (Syria), the center of the ancient plum trade.}%
Do flavors of modality correspond to some sorts of signals in the structure of
sentences? Read the following famous passage from \cite{kratzer-1991-modality}
and think about how the two sentences with their very different modal meanings
differ in structure:

\begin{quotation}
	
	\noindent Consider sentences (\nextx) and (\anextx):%
  
  \ex[numoffset=\leftmargin] Hydrangeas can grow here. \xe
	\ex~[numoffset=\leftmargin] There might be hydrangeas growing here. \xe
	
	\noindent The two sentences differ in meaning in a way which is illustrated by
  the following scenario.
	
	\medskip Suppose I acquire a piece of land in a far away country and discover
  that soil and climate are very much like at home, where hydrangeas prosper
  everywhere. Since hydrangeas are my favorite plants, I wonder whether they
  would grow in this place and inquire about it. The answer is (\blastx). In
  such a situation, the proposition expressed by (\blastx) is true. It is true
  regardless of whether it is or isn't likely that there are already hydrangeas
  in the country we are considering. All that matters is climate, soil, the
  special properties of hydrangeas, and the like. Suppose now that the country
  we are in has never had any contacts whatsoever with Asia or America, and the
  vegetation is altogether different from ours. Given this evidence, my
  utterance of (\lastx) would express a false proposition. What counts here is
  the complete evidence available. And this evidence is not compatible with the
  existence of hydrangeas.
	
	\medskip (\blastx) together with our scenario illustrates the pure
  \term{circumstantial} reading of the modal \expression{can}. [\dots ].
  (\lastx) together with our scenario illustrates the epistemic reading of
  modals. [\dots] circumstantial and epistemic conversational backgrounds
  involve different kinds of facts. In using an epistemic modal, we are
  interested in what else may or must be the case in our world given all the
  evidence available. Using a circumstantial modal, we are interested in the
  necessities implied by or the possibilities opened up by certain sorts of
  facts. Epistemic modality is the modality of curious people like historians,
  detectives, and futurologists. Circumstantial modality is the modality of
  rational agents like gardeners, architects, and engineers. A historian asks
  what might have been the case, given all the available facts. An engineer asks
  what can be done given certain relevant facts.
\end{quotation}

\noindent Consider also the very different prominent meanings of the following
two sentences, taken from Kratzer as well:

\pex \a Cathy can make a pound of cheese out of this can of milk. 
\a Cathy might make a pound of cheese out of this can of milk. \xe

\subsection{Toward an analysis}
\label{sec:modal-analysis}

How can we account for this variety of readings? One way would be to write a
host of lexical entries, basically treating this as a kind of (more or less
principled) ambiguity/polysemy. Another way, which is preferred by many people,
is to treat this as a case of context-dependency, as argued in seminal work by
\citet{kratzer-1977-must-can, kratzer-1978-dissertation,
  kratzer-1981-notional,kratzer-1991-modality}.

\note{It is well-known that natural language quantification is in general
  subject to contextual restriction. See for example
  \cite[Ch.2]{fintel-1994-thesis} and \cite{stanley-szabo-2000-restriction}.}%
According to Kratzer, what a modal brings with it intrinsically is just a modal
force, that is, whether it is an existential (possibility) modal or a universal
(necessity) modal. What worlds it quantifies over is determined by context. In
essence, the context has to supply a restriction to the quantifier. How can we
implement this idea? Well, we just have to transpose the setup we put in place
for conditionals.

We give modals a flavor argument, a silent pronoun of type \type{s,st}, just
like we did with \emph{if}. So we posit LF-structures like this:

\ex \label{newlf} [ [must $f_{\type{n,\type{s,st}}}$ ] [you quiet] ] \xe
%
$f_{\type{n,\type{s,st}}}$, again, is a variable over functions from worlds to
(characteristic functions of) sets of worlds, which \dash like all free
variables \dash needs to receive a value from the utterance context. For
example, it may be supplied with the function which, for any world $w$, yields
the set \{$w'$: the house rules that are in force in $w$ are obeyed in $w'$\}.
If we apply this function to a world $w_{1}$, in which the house rules read ``no
noise after 10 pm'', it will yield a set of worlds in which nobody makes noise
after 10 pm. If we apply the same function to a world $w_{2}$, in which the
house rules read ``no noise after 11 pm'', it will yield a set of worlds in
which nobody makes noise after 11 pm.

\note{Compare these entries for the modals to the entry for \emph{if} in
  \refx{ex:if-strict-context-formal}. What are the differences?}%
The new lexical entries for \expression{must} and \expression{may} that will fit
this new structure are these:

\pex\label{ex:modal-sst}
\a \exts{must} = \\
$\lambda f\in D_{\type{s,st}}.\ \lambda q\in D_{\type{s,t}}.\ \forall w'\in W\
[f(w)(w') =1 \rightarrow q(w')=1]$%
\note{in set talk: $f(w)\subseteq q$}
\a \exts{may} = \\
$\lambda f\in D_{\type{s,st}}.\ \lambda q\in D_{\type{s,t}}.\ \exists w'\in
W\ [f(w)(w')=1\ \&\ q(w')=1]$%
\note{in set talk: $f(w)\cap q\neq\emptyset$}
\xe

\begin{exercise}
  Let $w$ be a world, and assume that the context supplies an assignment $g$
  such that:
  %
  \ex $g(f_{\type{17,\type{s,st}}}) = \lambda w.\ \lambda w'.$ the rules
  in force in $w$ are obeyed in $w'$.\label{ex:deontic-f} \xe
  %
  Compute the truth-conditions for the LF in \Next:

  \ex\ [ must $f_{\type{17,\type{s,st}}}$  [you quiet] ] \xe
  %
  Does truth-value of \Last correctly depend on the evaluation world $w$? \qed
\end{exercise}


On this approach, the epistemic, deontic, etc. ``readings'' of individual
occurrences of modal verbs come about by a combination of two separate things.
The lexical semantics of the modal itself encodes just a quantificational force,
a \emph{relation} between sets of worlds. This is either the subset-relation
(universal quantification; necessity) or the relation of non-disjointness
(existential quantification; possibility). The covert variable next to the modal
when applied to the evaluation world yields a set of worlds, and this functions
as the quantifier's restrictor. The labels ``epistemic'', ``deontic'',
``circumstantial'' etc. group together certain conceptually natural classes of
possible values for this covert restrictor.

Notice that, strictly speaking, there is not just one deontic reading (for
example), but many. A speaker who utters

\ex You have to be quiet. \xe
%
might mean: `I want you to be quiet,' (i.e., you are quiet in all those worlds
that conform to my preferences). Or she might mean: `unless you are quiet, you
won't succeed in what you are trying to do,' (i.e., you are quiet in all those
worlds in which you succeed at your current task). Or she might mean: `the house
rules of this dormitory here demand that you be quiet,' (i.e., you are quiet in
all those worlds in which the house rules aren't violated). And so on. %
\note{Proponents of polysemy accounts of the variety of modal flavors will
  presumably have to tackle the apparent limitlessness of variation in some
  principle way. See \cite{viebahn-vetter-2016-modal-polysemy} for a polysemy
  account.}%
So the label ``deontic'' appears to cover a whole open-ended set of imaginable
``readings'', and which one is intended and understood on a particular utterance
occasion may depend on all sorts of things in the interlocutors' previous
conversation and tacit shared assumptions. (And the same goes for the other
traditional labels.)

A disappointing feature of our analysis in \refx{ex:modal-sst} is that a lot of
the work is being done by the modals: they don't just take a restriction as
their argument but they have to enforce that this restriction is evaluated in
the evaluation world. This is a departure from the ideal that modals are simply
quantifiers over possible worlds. It would be preferable to merely present them
with a set of worlds to quantify over rather than giving them the responsibility
of obtaining this set by applying an accessibility relation to the evaluation
world. We will later put in place a different framework (for unrelated reasons)
that will make good on this vision.

\begin{exercise}	
	Describe two worlds $w_{1}$ and $w_{2}$ so that\\
	\exts[w_1,g]{must $f_{\type{17,\type{s,st}}}$ you quiet} = 1\\
  and \exts[w_2,g]{must $f_{\type{17,\type{s,st}}}$ you quiet} = 0. \qed
\end{exercise}
\begin{exercise}
	
	In analogy to the deontic relation
  $g(f_{\type{17,\type{s,st}}})$ defined in \refx{ex:deontic-f}, define an
  appropriate relation that yields an epistemic reading for a sentence
  like \expression{You may be quiet}. \qed
\end{exercise}

\begin{exercise}
  Describe the set of worlds that constitutes the understood restrictor of
  \emph{have to} in each of the examples in \refx{ex:poly-have}. \qed
\end{exercise}

\begin{exercise}
	
	Describe values for the covert \type{s,st}-variable that are
  intuitively suitable for the interpretation of the modals in the
  following sentences:
	
	\ex As far as John's preferences are concerned, you
  \expression{may} stay with us. \xe
	
	\ex~ According to the guidelines of the graduate school, every PhD
  candidate \expression{must} take 9 credit hours outside his/her
  department. \xe
	
	\ex~ John \expression{can} run a mile in 5 minutes. \xe
	
	\ex~ This \expression{has} to be the White House. \xe
	
	\ex~ This elevator \expression{can} carry up to 3000 pounds. \xe
	%
	For some of the sentences, different interpretations are conceivable
  depending on the circumstances in which they are uttered. You may
  therefore have to sketch the utterance context you have in mind
  before describing the accessibility relation. \qed
\end{exercise}
\begin{exercise}
	
	Collect two naturally occurring examples of modalized sentences
  (e.g., sentences that you overhear in conversation, or read in a
  newspaper or novel -- not ones that are being used as examples in a
  linguistics or philosophy paper!), and give definitions of values
  for the covert \type{s,st}-variable which account for the way in
  which you actually understood these sentences when you encountered
  them. (If the appropriate interpretation is not salient for the
  sentence out of context, include information about the relevant
  preceding text or non-linguistic background.) \qed
\end{exercise}

\begin{exercise}
  Modals can be iterated like other intensional operators:

  \ex You might have to walk. \xe
  %
  Describe a context in which \Last would be an appropriate utterance. State
  plausible values for the \type{s,st} flavor functions restricting the two
  modals. Posit an LF for \Last. Derive the truth-conditions compositionally.
  \qed
  
\end{exercise}

\section{Explorations and variations}
\label{sec:variations}

With the basic analyses of attitudes, conditionals, and modals in place, we turn
to some technical explorations.

\subsection{Accessibility relations}
\label{sec:accessibility}

In all of the cases we have looked at the intensional operator ranges over
worlds that are relevant related to the evaluation world (via some kind of
anchor). In the case of conditional and modals, in fact, we made the operator
take a relation between worlds as its contextually provided first
argument:

\ex%
$\svt{if}^{w,g} = \lambda f_{\type{s,st}}.\ \lambda p_{\type{s,t}}.\
\lambda q_{\type{s,t}}.$ \\
\hfill $\forall w'\co f(w)(w') = 1\ \&\ p(w')=1 \rightarrow q(w')=1.$ \xe

\ex \exts{must} =
$\lambda f_{\type{s,st}}.\ \lambda q_{\type{s,t}}.\ \forall w'\co f(w)(w')
=1 \rightarrow q(w')=1$. \xe

\enlargethispage{24pt}
In the Hintikka-semantics for attitudes as well, we think of the worlds
quantified over as those that stand in a particular relation to the evaluation
world. Which relation that is is lexically specified by the attitude predicate.
So, \expression{believe} ranges over worlds compatible with the subject's
beliefs in the evaluation world.

It might be useful to rehearse some mathematical basics. A relation is something
that two elements stand in. A relation might not be symmetrical (or as we say
outside of math, it might not be reciprocated), so the order of elements is part
of the notion. The formal model of a relation $\mathcal{R}$ therefore is a set
of ordered pairs drawn from two sets $A$ and $B$:

\[\mathcal{R} = \{\pair{x,y}\co x\in A\ \&\ y\in B\}\]

In the function-centric and Schönfinkeled world we inherit from Heim \& Kratzer,
this notion of a set of ordered pairs becomes a staggered function that takes
first an element from $A$, then an element from $B$, and then yields a
truth-value:

\[f_{\mathcal{R}} = \lambda x\co x\in A.\ \left( \lambda y\co y\in B.\ \pair{x,y}\in
\mathcal{R}\right)\]

Since we are used to thinking of a function from some type to truth-values as
characterizing a set of things of that type, we can reconceive a relation as a
function from elements of the first set to a set of elements from the second set
(the ones they stand in the relevant relation to).

\note{One might call this an ``endorelation''.}%
Note that in the case of accessibility relations, we are dealing with a relation
between worlds. That is, the elements that stand in the relation come from the
same set. 

\note{Why do we call reflexivity, transitivity, and symmetry ``formal''
  properties of relations? The idea is that certain properties are ``formal'' or
  ``logical'', while others are more substantial. So, the fact that the relation
  ``have the same birthday as'' is symmetric seems a more formal fact about it
  than the fact that the relation holds between my daughter and my
  brother-in-law. Nevertheless, one of the most common ways of characterizing
  formal/logical notions (permutation-invariance, if you're curious) does not in
  fact make symmetry etc. a formal/logical notion. So, while intuitively these
  do seem to be formal/logical properties, we do not know how to substantiate
  that intuition. See \cite{macfarlane-2005-logical-constants} for discussion.}%
Recall now (for example, from Section 6.6 of H\&K) that the linguistic study of
determiners benefitted quite a bit from an investigation of the formal
properties of the relations between sets of individuals that determiners
express. We can do the same thing here and ask about the formal properties of
the accessibility relation associated with belief versus the one associated with
knowledge, etc. The obvious properties to think about are reflexivity,
transitivity, and symmetry.

A relation is \term{reflexive} iff for any object in the domain of the relation
we know that the relation holds between that object and itself. Which
accessibility relations are reflexive? Take a knowledge-induced accessibility
relation (let's call it $\mathcal{K}_{x}$):

\ex $w\mathcal{K}_{x}w'$ iff $w'$ is compatible with what $x$
knows in $w$. \xe

\note{We talk here about knowledge entailing (or even presupposing) truth but we
  do not mean to say that knowledge simply equals true belief. For initial
  overview of the issues, see the Stanford Encyclopedia entry:
  \cite{sep-knowledge-analysis}.}%
We are asking whether for any given possible world $w$, we know that
$\mathcal{K}_{x}$ holds between $w$ and $w$ itself. It will hold
if $w$ is a world that is compatible with what we know in $w$. And clearly that
must be so. Take our body of knowledge in $w$. The concept of knowledge
crucially contains the concept of truth: what we know must be true. So if in
$w$, we know that something is the case then it must be the case in $w$. So, $w$
must be compatible with all we know in $w$. $\mathcal{K}_{x}$ is
reflexive.

\clearpage
Now, if an attitude $X$ corresponds to a reflexive accessibility relation, then
we can conclude from \expression{$a$ $Xs$ that $p$} being true in $w$ that $p$
is true in $w$. %
\note{In modal logic notation: $\Box p \rightarrow p $. This pattern is
  sometimes called \textbf{T} or \textbf{M}, as is the corresponding system of
  modal logic.}%
This property of an attitude predicate is often called \term{veridicality}. It
is to be distinguished from \term{factivity}, which is a property of attitudes
which \emph{presuppose} -- rather than (merely) entail -- the truth of their
complement.

If we consider the relation $\mathcal{B}_{x}$ pairing with a world
$w$ those worlds $w'$ which are compatible with what $x$ \emph{believes} in $w$,
we no longer have reflexivity: belief is not a veridical attitude. %
\note{The difference between \emph{believe} and \emph{know} in natural
  discourse is quite delicate, especially when one considers first person uses
  (\emph{I believe the earth is flat} vs. \emph{I know the earth is flat}).}%
It is easy to have false beliefs, which means that the actual world is not in
fact compatible with one's beliefs, which contradicts reflexivity. And many
other attitudes as well do not involve veridicality/reflexivity: what we hope
may not come true, what we remember may not be what actually happened, etc.

In modal logic, the correspondence between formal properties of the
accessibility relation and the validity of inference patterns is well-studied.
What we have just seen is that reflexivity of the accessibility relation
corresponds to the validity of $\Box p \rightarrow p$. Other properties
correspond to other characteristic patterns. Let's see this for transitivity and
symmetry.

\note{In the literature on epistemic modal logic, the pattern is known as the
  \term{KK Thesis} or \term{Positive Introspection}. In general modal logic, it
  is the characteristic axiom \textbf{4} of the modal logic system \textbf{S4},
  which is a system that adds \textbf{4} to the previous axiom
  \textbf{M}/\textbf{T}. Thus, \textbf{S4} is the logic of accessibility
  relations that are both reflexive and transitive.}%
\term{Transitivity} of the accessibility relation corresponds to the inference
$\Box p \rightarrow \Box \Box p$. The pattern seems not obviously wrong
for knowledge: if one knows that $p$, doesn't one thereby know that one knows
that $p$? But before we comment on that, let's establish the formal
correspondence between transitivity and that inference pattern. This needs to go
in both directions.
  
\marginfig{transitivity.pdf}%
What does it take for the pattern to be valid? Assume that $\Box p$
holds for an arbitrary world $w$, i.e. that $p$ is true in all worlds $w'$
accessible from $w$. Now, the inference is to the fact that $p$ again holds in
any world $w''$ accessible from any of those worlds $w'$ accessible from $w$.
But what would prevent $p$ from being false in some $w''$ accessible from some
$w'$ accessible from $w$? That could only be prevented from happening if we knew
that $w''$ itself is accessible from $w$ as well, because then we would know
from the premiss that $p$ is true in it (since $p$ is true in \emph{all} worlds
accessible from $w$). Ah, but $w''$ (some world accessible from a world $w'$
accessible from $w$) is only guaranteed to be accessible from $w$ if the
accessibility relation is transitive (if $w'$ is accessible from $w$ and $w''$
is accessible from $w'$, then transitivity ensures that $w''$ is accessible from
$w$). This reasoning has shown that validity of the pattern requires
transitivity. The other half of proving the correspondence is to show that
transitivity entails that the pattern is valid.

The proof proceeds by reductio. Assume that the accessibility relation is
transitive. Assume that (i) $\Box p$ holds for some world $w$ but that (ii)
$\Box \Box p$ doesn't hold in $w$. We will show that this situation cannot
obtain. By (i), $p$ is true in all worlds $w'$ accessible from $w$. By (ii),
there is some non-$p$ world $w''$ accessible from some world $w'$ accessible
from $w$. But by transitivity of the accessibility relation, that non-$p$ world
$w''$ must be accessible from $w$. And since \emph{all} worlds accessible from
$w$ are $p$ worlds, $w''$ must be a $p$ world, in contradiction to (ii). So, as
soon as we assume transitivity, there is no way for the inference not to go
through.

Now, do any of the attitudes have the transitivity property? It seems rather
obvious that as soon as you believe something, you thereby believe that you
believe it (and so it seems that belief involves a transitive accessibility
relation). And in fact, as soon as you believe something, you believe that you
\emph{know} it. But one might shy away from saying that knowing something
automatically amounts to knowing that you know it. For example, many are
attracted to the idea that to know something requires that (i) that it is true,
(ii) that you believe it, and (iii) that you are justified in believing it: the
justified true belief analysis of knowledge. So, now couldn't it be that you
know something, and thus (?) that you believe you know it, and thus that you
believe that you are justified in believing it, but that you are not justified
in believing that you are \emph{justified} in believing it? After all, one's
source of knowledge, one's reliable means of acquiring knowledge, might be a
mechanism that one has no insight into. So, while one can implicitly trust
(believe) in its reliability, and while it is in fact reliable, one might not
have any means to have trustworthy beliefs about it. [Further worries about the
KK Thesis are discussed in \cite{williamson-2000-limits}.]

\kwn
\note{In modal logic notation, \Next looks as follows:
  $p \rightarrow \Box\Diamond p$, known simply as B in modal logic. The
  system that combines \textbf{T}/\textbf{M} with B is often called Brouwer's
  System (\textbf{B}), after the mathematician L.E.J. Brouwer, not because he
  proposed it but because it was thought that it had some connections to his
  doctrines. Brouwer got his own commemorative stamp from the Netherlands:}%
\marginfig[0.5]{brouwer.jpg}%
What would the consequences be if the accessibility relation were
\term{symmetric}? Symmetry of the accessibility relation $\mathcal{R}$
corresponds to the validity of the following principle:
%
\ex Brouwer's Axiom :\\
$\forall p \forall w:\ w\in p \rightarrow \Bigl[\forall w' \bigl[ w\mathcal{R}w'
\rightarrow \exists w'' \left[ w'\mathcal{R}w'' \&\ w''\in p\right]\bigr]\Bigr]$
\xe

Here's the reasoning: Assume that $R$ is in fact symmetric. Pick a world $w$ in
which $p$ is true. Now, could it be that the right hand side of the inference
fails to hold in $w$? Assume that it does fail. Then, there must be some world
$w'$ accessible from $w$ in which $\Diamond p$ is false. In other words,
from that world $w'$ there is no accessible world $w''$ in which $p$ is true.
But since $R$ is assumed to be symmetric, one of the worlds accessible from $w'$
is $w$ and in $w$, $p$ is true, which contradicts the assumption that the
inference doesn't go through. So, symmetry ensures the validity of the
inference.

\marginfig{symmetry.pdf}%
The other way (validity of the inference requires symmetry): the inference says
that from any $p$-world (``$p$-world'' is a very common way of saying ``world of
which $p$ is true''), we can only access worlds from which, in turn, there is at
least one accessible $p$-world. But imagine that $p$ is true in $w$ but not true
in any other world. So, the only way for the conclusion of the inference to hold
automatically is to have a guarantee that $w$ (the only $p$-world) is accessible
from any world accessible from it. That is, we need to have symmetry. QED.

To see whether a particular kind of attitude is based on a symmetric
accessibility relation, we can ask whether Brouwer's Axiom is intuitively valid
with respect to this attitude. If it is not valid, this shows that the
accessibility relation can't be symmetric. In the case of a knowledge-based
accessibility relation (epistemic accessibility), one can argue that
\emph{symmetry does not hold} (thanks to Bob Stalnaker (pc to Kai von
  Fintel) for help with the following reasoning):
\begin{quote}
  
  The symmetry condition would imply that if something happens to be true in the
  actual world, then you know that it is compatible with your knowledge
  (Brouwer's Axiom). This will be violated by any case in which your beliefs are
  consistent, but mistaken. Suppose that while $p$ is in fact true, you feel
  certain that it is false, and so think that you know that it is false. Since
  you think you know this, it is compatible with your knowledge that you know
  it. (Since we are assuming you are consistent, you can't both believe that you
  know it, and know that you do not). So it is compatible with your knowledge
  that you know that \expression{not} $p$. Equivalently\note{This and the
    following step rely on the duality of necessity and possibility: $q$ is
    compatible with your knowledge iff you don't know that \expression{not}
    $q$.}: you don't know that you don't know that \expression{not} $p$.
  Equivalently: you don't know that it's compatible with your knowledge that
  $p$. But by Brouwer's Axiom, since $p$ is true, you would have to know that
  it's compatible with your knowledge that $p$. So if Brouwer's Axiom held,
  there would be a contradiction. So Brouwer's Axiom doesn't hold here, which
  shows that epistemic accessibility is not symmetric.
\end{quote}

\note{All one really needs to make \textbf{NI} valid is to have a
  \term{Euclidean} accessibility relation: any two worlds accessible from the
  same world are accessible from each other. It is a nice little exercise to
  prove this, if you have become interested in this sort of thing. Note that all
  reflexive and Euclidean accessibility relations are transitive and symmetric
  as well \dash another nice little thing to prove.}%
Game theorists and theoretical computer scientists who traffic in logics of
knowledge often assume that the accessibility relation for knowledge is an
equivalence relation (reflexive, symmetric, and transitive). But this is
appropriate only if one abstracts away from any error, in effect assuming that
belief and knowledge coincide. One striking consequence of working with an
equivalence relation as the accessibility relation for knowledge is that one
predicts the principle of \term{Negative Introspection} to hold:

\ex \extitle{Negative Introspection (\textbf{ni})}\\
If one doesn't know that $p$, then one knows that one doesn't know that $p$.
($\neg\Box p \rightarrow \Box\neg\Box p$). \xe

This surely seems rather dubious: imagine that one strongly believes that $p$
but that nevertheless $p$ is false, then one doesn't know that $p$, but one
doesn't seem to believe that one doesn't know that $p$, in fact one believes
that one does know that $p$.

\clearpage
\subsection{Conversational backgrounds}
\label{sec:conversational-backgrounds}

\marginfig{angelika.jpg}%
There is a well-known technical variant on relativizing the semantics of
intensional operators to accessibility relations. The basic idea, due to
Kratzer, is to make intensional operators work with a set of propositions rather
than a set of worlds. For example, \expression{must} would say that a certain
set of propositions together entail the prejacent. A conditional \expression{if
  $p$, $q$} would say that a certain set of propositions when augmented with
the antecedent proposition $p$ will entail the consequent proposition $q$.
And one could conceive of a belief state as providing a set of propositions
(each of which is a belief of the agent), so that the intensional operator
\expression{believe} would claim that this set of propositions entails the
prejacent.

We need to add the usual layer of contingency, which in this case means that the
relevant ingredient is actually a function from evaluation worlds to sets of
propositions.

At this point, we will assume that the resulting set of propositions is always
consistent, by which we mean that there's at least one world where all of the
propositions in the set are true:

\ex A set of propositions $\mathcal{P}$ is \term{consistent} relative to a set
of worlds $W$ iff $\exists w\in W\co \forall p\in\mathcal{P}\co p(w)=1)$. \xe

From any consistent set of propositions, we can retrieve the set of worlds
characterized by it: those worlds such that each proposition in the set is true
in them. If we think of propositions as sets of worlds, this corresponds to the
grand intersection of the set of propositions:

\ex For any consistent set of propositions $\mathcal{P}$,\\
\(char({\mathcal{P}}) = \{w\co \forall p\in\mathcal{P}\co p(w)=1\} =
\bigcap(\mathcal{P})\) \xe

The semantics of the modal \expression{must}, for example, can now be rewritten:

\ex \(\exts{must} = \lambda \mathcal{M}_{\type{s,\type{\type{st,t}}}}.\ \lambda
p.\ \bigcap(\mathcal{M}(w)) \subseteq p\) \xe
%
The context supplies a function $\mathcal{M}$ from worlds to sets of
propositions and \expression{must} claims that when applied to the evaluation
world, $\mathcal{M}$ yields a set of propositions that jointly entail the
prejacent $p$, or in other words: all the worlds where all the propositions in
$\mathcal{M}(w)$ are true are worlds where the prejacent $p$ is true.

Kratzer called functions of type $\type{s,\type{\type{st,t}}}$
\term{conversational backgrounds} and used the term \term{modal base} for the
conversational backgrounds that restrict modal operators. She also sometimes
calls the resulting sets of propositions \term{premise sets}.

\kwn
Note that we can retrieve accessibility relations from conversational
backgrounds:

\ex \label{convers}For any conversational background $\mathcal{M}$ of type
\type{s,\type{st,t}}, we define the corresponding accessibility relation $f_{\mathcal{M}}$
of type \type{s,st} as follows:

$f_{\mathcal{M}} := \lambda w.\ \lambda w'.\ \forall p\ [ \mathcal{M}(w)(p)=1\ \rightarrow\ p(w')=1
]$. \xe
%
In words, $w'$ is $f_{\mathcal{M}}$-accessible from $w$ iff all propositions $p$
that are assigned by $\mathcal{M}$ to $w$ are true in $w'$.

What motivates the complication from accessibility relations to conversational
backgrounds? One consideration that we will turn to in the next chapter concerns
inconsistent premise sets. But there is also another, perhaps more intuitive,
motivation. A conversational background is the sort of thing that is identified
by phrases like \expression{what the law provides, what we know}, etc. Take the
phrase \expression{what the law provides}. What the law provides is different
from one possible world to another. And what the law provides in a particular
world is a \emph{set of propositions}. Likewise, what we know differs from world
to world. And what we know in a particular world is a set of propositions. The
intension of \expression{what the law provides} is then that function which
assigns to every possible world the set of propositions $p$ such that the law
provides in that world that $p$. Of course, that doesn't mean that $p$ holds in
that world itself: the law can be broken. And the intension of \expression{what
  we know} will be that function which assigns to every possible world the set
of propositions we know in that world. Now, consider:

\ex (In view of what we know,) Brown must have murdered Smith. \xe

The \expression{in view of}-phrase may explicitly signal the intended
conversational background. Or, if the phrase is omitted, we can just infer from
other clues in the discourse that such an epistemic conversational background is
intended. 

What follows are some (increasingly technical exercises) on conversational
backgrounds.

\begin{exercise}
  \note{``$\mathcal{BS}$'' stands for ``belief state''.}%
  Imagine that we model individual $x$'s belief state with a set of
  propositions $\mathcal{BS}_{x}$. Now, when $x$ forms a new opinion, we could model
  this by adding a new proposition $p$ to $\mathcal{BS}_x$. So,
  $\mathcal{BS}_x$ now contains one further element. There are now more
  opinions. What happens to the set of worlds compatible with $x$'s beliefs?
  Does it get bigger or smaller? Is the new set a subset or superset of the
  previous set of compatible worlds? \qed
\end{exercise}

\begin{exercise}
  Kratzer calls a conversational background (modal base) \term{realistic} iff it
  assigns to \emph{any} world a set of propositions that are all true in that
  world. The modal base \expression{what we know} is realistic, the modal bases
  \expression{what we believe} and \expression{what we want} are not.

  Show that a conversational background $\mathcal{M}$ is realistic iff the
  corresponding accessibility relation $f_{\mathcal{M}}$ (defined as in
  \refx{convers}) is reflexive. \qed
\end{exercise}

\begin{exercise}
	
	Let us call an accessibility relation \term{trivial} if it makes
  every world accessible from every world. $f$ is \term{trivial} iff
  $\forall w\forall w'\co w'\in f(w)$. What would the conversational
  background $\mathcal{M}$ have to be like for the accessibility relation
  $f_{\mathcal{M}}$ to be trivial in this sense? \qed
\end{exercise}

\begin{exercise}\label{closure}
	
	[For the intrepid only!] The definition in \refx{convers} specifies, in
  effect, a function from $D_{\type{s,\type{st,t}}}$ to $D_{\type{s,st}}$. It
  maps each function $\mathcal{M}$ of type \type{s,\type{st,t}} to a unique
  function $f_{\mathcal{M}}$ of type \type{s,st}. This mapping is not
  one-to-one, however. Different elements of $D_{\type{s,\type{st,t}}}$ may be
  mapped to the same value in $D_{\type{s,st}}$.

  Prove this claim. I.e., give an example of two functions $\mathcal{M}$ and
  $\mathcal{M}'$ in $D_{\type{s,\type{st,t}}}$ for which \refx{convers}
  determines $f_{\mathcal{M}}$ = $f_{\mathcal{M}'}$ .
		
  As you have just proved, if every function of type \type{s,\type{st,t}}
  qualifies as a `conversational background', then two different conversational
  backgrounds can collapse into the same accessibility relation. Conceivably,
  however, if we imposed further restrictions on conversational backgrounds
  (i.e., conditions by which only a proper subset of the functions in
  $D_{\type{s,\type{st,t}}}$ would qualify as conversational backgrounds), then
  the mapping between conversational backgrounds and accessibility relations
  might become one-to-one after all. In this light, consider the following
  potential restriction:

	\ex Every conversational background $\mathcal{M}$ must be ``closed under
  entailment''; i.e., it must meet this condition:\\
  $\forall w. \forall p\ [ \ensuremath{\cap}\mathcal{M}(w)
  \ensuremath{\subseteq} p\
  \rightarrow\ p \in \mathcal{M}(w) ]$.\\
  (In words: if the propositions in $\mathcal{M}(w)$ taken together entail $p$,
  then $p$ must itself be in $\mathcal{M}(w)$.) \xe
	%
  Show that this restriction would ensure that the mapping defined in
  \refx{convers} will be one-to-one. \qed
\end{exercise} 

\section*{Further readings}

We have just put in place some of the basics of the analysis of attitudes,
conditionals, and modals. Much of the work in this area will become accessible
to you after the following chapter. For now, we recommend just a few
further readings.

\cite{swanson-2011-hsk-attitudes} is a recent survey on attitudes.

Further connections between mathematical properties of accessibility relations
and logical properties of various notions of necessity and possibility are
studied extensively in modal logic, see \cite{hughes-cresswell-1996-new} and
\cite{garson-2018-sep-logic-modal}, especially section 7 and 8, ``Modal Axioms
and Conditions on Frames'', ``Map of the Relationships between Modal Logics''. 
\note{We encourage you to visit the admirable Open Logic Project
  (\url{https://openlogicproject.org/}), that the \cite{zach-2019-open-modal}
  textbook is part of.}%
An open access textbook on modal logic is \cite{zach-2019-open-modal}.

A thorough discussion of the possible worlds theory of attitudes, and some of
its potential shortcomings, can be found in Bob Stalnaker's work
\citeyearpar{stalnaker-1984-inquiry,stalnaker-1999-contextandcontent}.

Two introductory readings on conditionals are
\cite{fintel-2011-hsk-conditionals,fintel-2012-subjunctives}.

The most important background readings on modals are the two papers
\cite{kratzer-1981-notional,kratzer-1991-modality}. There are updated versions
of Kratzer's classic papers in her volume ``Modals and conditionals''
\parencite{kratzer-2012-book}. A major resource on modality is Paul
Portner's book: \cite{portner-2009-modality-book}. You might also profit from
some survey-ish type papers on modals and modality: \cite{fintel-2005-modality},
\cite{fintel-gillies-2007-ose2}, \cite{swanson-2008-modality},
\cite{hacquard-2009-hsk-modality}.

%%% Local Variables:
%%% mode: latex
%%% TeX-master: "ik-book"
%%% End:
