%!TEX root = IntensionalSemantics.tex
\newcommand{\exts}[2][w,g]{\ensuremath{\sv{\mbox{#2}}^{#1}}}

\chapter{Modality}\label{cha:modality} % (fold)

\chapterprecishere{We turn to modal auxiliaries and related
  constructions. The main difference from attitude constructions is
  that their semantics is more context-dependent. Otherwise, we are
  still quantifying over possible worlds.}

\minitoc

\section{The Quantificational Theory of Modality} \label{sec:quant-theory-modal}

We will now be looking at modal auxiliaries like \expression{may,
  must, can, have to}, etc. Most of what we say here should carry over
straightforwardly to modal adverbs like \expression{maybe, possibly,
  certainly}, etc. We will make certain syntactic assumptions, which
make our work easier but which leave aside many questions that at some
point deserve to be addressed.

\subsection{Syntactic Assumptions} \label{sec:synt-assumpt-1}

We will assume, at least for the time being, that a modal like
\expression{may} is a \term{raising} predicate (rather than a
\term{control} predicate), i.e., its subject is not its own argument,
but has been moved from the subject-position of its infinitival
complement.\sidepar{The issue of raising vs. control will probably be
  taken up later. If you are eager to get started on it and other
  questions of the morphosyntax of modals, read the handout from an
  LSA class Sabine and Kai taught a few years ago:
  \url{http://web.mit.edu/fintel/lsa220-class-2-handout.pdf}.} So, we
are dealing with the following kind of structure:

\exi. \a. Ann may be smart. 
\b. [ Ann [ $\lambda_1$ [ may [ t$_1$ be smart ]]]]

Actually, we will be working here with the even simpler structure
below, in which the subject has been reconstructed to its lowest trace
position. (E.g., these could be generated by deleting all but the
lowest copy in the movement chain.\sidepar{We will talk about
  reconstruction in more detail later.}) We will be able to prove that
movement of a name or pronoun never affects truth-conditions, so at
any rate the interpretation of the structure in \Last[b] would be the
same as of \Next. As a matter of convenience, then, we will take the
reconstructed structures, which allow us to abstract away from the
(here irrelevant) mechanics of variable binding.

\exi. may [ Ann be smart ]

So, for now at least, we are assuming that modals are expressions that
take a full sentence as their semantic argument.\footnote{We will
  assume that even though \expression{Ann be smart} is a non-finite
  sentence, this will not have any effect on its semantic type, which
  is that of a sentence, which in turn means that its semantic value
  is a truth-value. This is hopefully independent of the (interesting)
  fact that \expression{Ann be smart} on its own cannot be used to
  make a truth-evaluable assertion.} Now then, what do modals mean?

\subsection{Quantification over Possible Worlds} \label{sec:quant-over-poss}

The\sidepar{This idea goes back a long time. It was famously held
  by Leibniz, but there are precedents in the medieval literature, see
  \citet{knuuttila:2003:modality-medieval}. See
  \citet{copeland:2002:genesis} for the modern history of the possible
  worlds analysis of modal expressions.} basic idea of the possible
worlds semantics for modal expressions is that they are quantifiers
over possible worlds. Toy lexical entries for \expression{must} and
\expression{may}, for example, would look like this:

\ex. $\sv{\mbox{must}}^{w,g} = \lambda p_{\angles{s,t}}.\ \forall w'\co p(w') = 1$.

\ex. $\sv{\mbox{may}}^{w,g} = \lambda p_{\angles{s,t}}.\ \exists w'\co p(w') = 1$.

This analysis is too crude (in particular, notice that it would make
modal sentences non-contingent \dash there is no occurrence of the
evaluation world on the right hand side!). But it does already have
some desirable consequences that we will seek to preserve through all
subsequent refinements. It correctly predicts a number of intuitive
judgments about the logical relations between \expression{must} and
\expression{may} and among various combinations of these items and
negations. To start with some elementary facts, we feel that
\expression{must} $\phi$ entails \expression{may} $\phi$, but not vice
versa:

\ex. You must stay.\\
Therefore, you may stay. \hfill\textsc{valid}

\ex. You may stay.\\
Therefore, you must stay. \hfill\textsc{invalid}

\ex. \a. You may stay, but it is not the case that you must
stay.\footnote{The somewhat stilted \expression{it is not the
    case}-construction is used in to make certain that negation takes
  scope over \expression{must}. When modal auxiliaries and negation
  are together in the auxiliary complex of the same clause, their
  relative scope seems not to be transparently encoded in the surface
  order; specifically, the scope order is not reliably negation
  $\succ$ modal. (Think about examples with \expression{mustn't},
  \expression{can't, shouldn't, may not} etc. What's going on here?
  This is an interesting topic which we must set aside for now. See
  the references at the end of the chapter for relevant work.) With
  modal \emph{main} verbs (such as \expression{have to}), this
  complication doesn't arise; they are consistently inside the scope
  of clause-mate auxiliary negation. Therefore we can use (b) to
  (unambiguously) express the same scope order as (a), without having
  to resort to a biclausal structure.} \b. You may stay, but you don't
have to stay. \hfill\textsc{consistent}

We judge \expression{must} $\phi$ incompatible with its ``inner
negation'' \expression{must} [\expression{not} $\phi$ ], but find
\expression{may} $\phi$ and \expression{may} [\expression{not} $\phi$
] entirely compatible:

\ex. You must stay, and/but also, you must leave. (leave = not stay).\\
\null\hfill\textsc{contradictory}

\ex. You may stay, but also, you may leave. \\
\null\hfill\textsc{consistent}

We also judge that in each pair below, the (a)-sentence and the
(b)-sentences say the same thing.

\ex. \a. You must stay. \b. It is not the case that you may leave.\\
You aren't allowed to leave.\\
(You may not leave.)\footnote{The parenthesized variants of the (b)-sentences are pertinent here only to the extent that we can be certain that negation scopes over the modal. In these examples, apparently it does, but as we remarked above, this cannot be taken for granted in all structures of this form.}\\
(You can't leave.)

\ex. \a. You may stay. \b. It is not the case that you must leave.\\
You don't have to leave.\\
You don't need to leave.\\
(You needn't leave.)

Given that \expression{stay} and \expression{leave} are each other's
negations (i.e. \exts{leave} = \exts{not stay}, and \exts{stay} =
\exts{not leave}), the LF-structures of these equivalent pairs of
sentences can be seen to instantiate the following
schemata:\footnote{In logicians' jargon, \expression{must} and
  \expression{may} behave as \term{duals} of each other. For
  definitions of ``dual'', see
  \citet[197]{barwise-cooper:1981:generalized} or
  \citet[vol.2,238]{gamut:91}.}

\ex. \a. \expression{must} $\phi$ $\equiv$ \expression{not} [\expression{may} [\expression{not} $\phi$]] \b. \expression{must} [\expression{not} \ensuremath{\psi}] $\equiv$ \expression{not} [\expression{may} \ensuremath{\psi}]

\ex. \a. \expression{may} $\phi$ $\equiv$ \expression{not} [\expression{must} [\expression{not} $\phi$]] \b. \expression{may} [\expression{not} \ensuremath{\psi}] $\equiv$ \expression{not} [\expression{must} \ensuremath{\psi}]

Our present analysis of \expression{must}, \expression{have-to},
\dots{} as universal quantifiers and of \expression{may},
\expression{can}, \dots{} as existential quantifiers straightforwardly
predicts all of the above judgments, as you can easily
prove.\marginnote{More linguistic data regarding the ``parallel
  logic'' of modals and quantifiers can be found in Larry Horn's
  dissertation \citep{horn:1972:dissertation}.}

\ex. \a. $\forall x \phi \equiv \neg \exists \neg \phi$ \b. $\forall x \neg \phi \equiv \neg \exists x \phi$

\ex. \a. $\exists x \phi \equiv \neg \forall x \neg \phi$ \b. $\exists x \neg \phi \equiv \neg \forall x \phi$

\section{Flavors of Modality} \label{sec:flavors}

\subsection{Contingency} \label{sec:contingency}

We already said that the semantics we started with is too
simple-minded. In particular, we have no dependency on the evaluation
world, which would make modal statements non-contingent. This is not
correct.

If one says \expression{It may be snowing in Cambridge}, that may well
be part of useful, practical advice about what to wear on your
upcoming trip to Cambridge. It may be true or it may be false. The
sentence seems true if said in the dead of winter when we have already
heard about a Nor'Easter that is sweeping across New England. The
sentence seems false if said by a clueless Australian acquaintance of
ours in July.

The contingency of modal claims is not captured by our current
semantics. All the \expression{may}-sentence would claim under that
semantics is that there is some possible world where it is snowing in
Cambridge. And surely, once you have read Lewis' quote in Chapter 1,
where he asserts the existence of possible worlds with different
physical constants than we enjoy here, you must admit that there have
to be such worlds even if it is July. The problem is that in our
semantics, repeated here

\ex. $\sv{\mbox{may}}^{w,g} = \lambda p_{\angles{s,t}}.\ \exists w'\co p(w') = 1$.

there is no occurrence of $w$ on the right hand side. This means that
the truth-conditions for \expression{may}-sentences are
world-independent. In other words, they make non-contingent claims
that are either true whatever or false whatever, and because of the
plenitude of possible worlds they are more likely to be true than
false. This needs to be fixed. But how?

Well, what makes \expression{it may be snowing in Cambridge} seem true
when we know about a Nor'Easter over New England? What makes it seem
false when we know that it is summer in New England? The idea is that
we only consider possible worlds \term{compatible with the evidence
  available to us}. And since what evidence is available to us differs
from world to world, so will the truth of a
\expression{may}-statement.

\ex. $\sv{\mbox{may}}^{w,g} = \lambda p.\ \exists w' \mbox{ compatible with the evidence in } w\co p(w') = 1$.\footnote{From now on, we will leave off type-specifications such as that $p$ has to be of type $\angles{s,t}$, whenever it is obvious what they should be and when saving space is aesthetically called for.}

\ex. $\sv{\mbox{must}}^{w,g} = \lambda p.\ \forall w' \mbox{ compatible with the evidence in } w\co p(w') = 1$.

Let us consider a different example:

\ex. You have to be quiet.

Imagine this sentence being said based on the house rules of the
particular dormitory you live in. Again, this is a sentence that could
be true or could be false. Why do we feel that this is a contingent
assertion? Well, the house rules can be different from one world to
the next, and so we might be unsure or mistaken about what they are.
In one possible world, they say that all noise must stop at 11pm, in
another world they say that all noise must stop at 10pm. Suppose we
know that it is 10:30 now, and that the dorm we are in has either one
or the other of these two rules, but we have forgotten which. Then,
for all we know, \expression{you have to be quiet} may be true or it
may be false. This suggests a lexical entry along these lines:

\ex. $\sv{\mbox{have-to}}^{w,g} = \lambda p.\ \forall w' \mbox{ compatible with the rules in } w\co p(w') = 1$.

Again, we are tying the modal statement about other worlds down to
certain worlds that stand in a certain relation to actual world: those
worlds where the rules as they are here are obeyed.

A note of caution: it is very important to realize that the worlds
compatible with the rules as they are in $w$ are those worlds where
nothing happens that violates any of the $w$-rules. This is not at all
the same as saying that the worlds compatible with the rules in $w$
are those worlds where the same rules are in force. Usually, the rules
do not care what the rules are, unless the rules contain some kind of
meta-statement to the effect that the rules have to be the way they
are, i.e. that the rules cannot be changed. So, in fact, a world $w'$
in which nothing happens that violates the rules as they are in $w$
but where the rules are quite different and in fact what happens
violates the rules as they are in $w'$ is nevertheless a world
compatible with the rules in $w$. For example, imagine that the only
relevant rule in $w$ is that students go to bed before midnight. Take
a world $w'$ where a particular student goes to bed at 11:30 pm but
where the rules are different and say that students have to go to bed
before 11 pm. Such a world $w'$ is compatible with the rules in $w$
(but of course not with the rules in $w'$).

Apparently, there are different flavors of modality, varying in what
kind of facts in the evaluation world they are sensitive to. The
semantics we gave for \expression{must} and \expression{may} above
makes them talk about evidence, while the semantics we gave for
\expression{have-to} made it talk about rules. But that was just
because the examples were hand-picked. In fact, in the dorm scenario
we could just as well have said \expression{You must be quiet}. And,
vice versa, there is nothing wrong with using \expression{it has to be
  snowing in Cambridge} based on the evidence we have. In fact, many
modal expressions seem to be multiply ambiguous.

Traditional descriptions of modals often distinguish a number of
``readings'': \term{epistemic, deontic, ability, circumstantial,
  dynamic,} \dots. (Beyond ``epistemic'' and ``deontic,'' there is a
great deal of terminological variety. Sometimes all non-epistemic
readings are grouped together under the term \term{root modality}.)
Here are some initial illustrations.

\ex. \label{epist}\extitle{Epistemic Modality}\\[6pt]
A: Where is John?\\
B: I don't know. He \expression{may} be at home.

\ex. \extitle{Deontic Modality}\\[6pt]
A: Am I allowed to stay over at Janet's house?\\
B: No, but you \expression{may} bring her here for dinner.

\ex. \extitle{Circumstantial/Dynamic Modality}\\[6pt]
A: I will plant the rhododendron here.\\
B: That's not a good idea. It \expression{can} grow very tall.

How are \expression{may} and \expression{can} interpreted in each of
these examples? What do the interpretations have in common, and where
do they differ?

In all three examples, the modal makes an existentially quantified
claim about possible worlds. This is usually called the \term{modal
  force} of the claim. What differs is what worlds are quantified
over. In \term{epistemic} modal sentences, we quantify over worlds
compatible with the available evidence. In \term{deontic} modal
sentences, we quantify over worlds compatible with the rules and/or
regulations. And in the \term{circumstantial} modal sentence, we
quantify over the set of worlds which conform to the laws of nature
(in particular, plant biology). What speaker B in \Last is saying,
then, is that there are some worlds conforming to the laws of nature
in which this rhododendron grows very tall. (Or is this another
instance of an epistemic reading? See below for discussion of the
distinction between circumstantial readings and epistemic ones.)

How can we account for this variety of readings? One way would be to
write a host of lexical entries, basically treating this as a kind of
(more or less principled) ambiguity. Another
way,\marginpar{\centering\includegraphics[height=1in]{angelika.jpg}\\
  {\tiny \href{http://people.umass.edu/kratzer/}{Angelika Kratzer}}}
which is preferred by many people, is to treat this as a case of
context-dependency, as argued in seminal work by
\citet{kratzer:1977:must-can,kratzer:1978:dissertation,kratzer:1981:notional,kratzer:1991:modality}.

According\marginnote{It is well-known that natural language
  quantification is in general subject to contextual restriction. See
  \citet{stanley-szabo:2000:restriction} for a recent discussion.} to
Kratzer, what a modal brings with it intrinsically is just a modal
force, that is, whether it is an existential (possibility) modal or a
universal (necessity) modal. What worlds it quantifies over is
determined by context. In essence, the context has to supply a
restriction to the quantifier. How can we implement this idea?

We encountered context-dependency before when we talked about pronouns
and their referential (and E-Type) readings (H\amp K, chapters 9--11).
We treated referential pronouns as free variables, appealing to a
general principle that free variables in an LF need to be supplied
with values from the utterance context. If we want to describe the
context-dependency of modals in a technically analogous fashion, we
can think of their LF-representations as incorporating or
subcategorizing for a kind of invisible pronoun, a free variable that
stands for a set of possible worlds. So we posit LF-structures like
this:

\exi. \label{newlf} [I$'$ [I must $p_{\angles{n,\angles{s,t}}}$ ] [VP you quiet]]

\sidepar{We are using the notation for variables of types other than
  $e$, introduced by Heim \amp\ Kratzer. See p. 213. An index on a
  variable now is an ordered pair of a natural number and a type.
  \\[\baselineskip] Q: Can you think of overt anaphoric expressions
  that are arguably of the type $\angles{s,t}$, a proposition?}%
$p_{\angles{n,\angles{s,t}}}$ here is a variable over (characteristic
functions of) sets of worlds, which \dash like all free variables
\dash needs to receive a value from the utterance context. Possible
values include: the set of worlds compatible with the speaker's
current knowledge; the set of worlds in which everyone obeys all the
house rules of a certain dormitory; and many others. The denotation of
the modal itself now has to be of type $\angles{st,\angles{st,t}}$
rather than $\angles{st,t}$, thus it will be more like a
quantificational determiner rather than a complete generalized
quantifier. Only after the modal has been combined with its covert
restrictor do we obtain a value of type $\angles{st,t}$.

\ex. \a. \exts{must} = \exts{have-to} = \exts{need-to} = \dots{} =\\
$\lambda p\in D_{\angles{s,t}}.\ \lambda q\in D_{\angles{s,t}}.\ \forall w\in W\ [p(w)=1 \rightarrow q(w)=1]$\marginnote{in set talk: $p\ensuremath{\subseteq}q$}
\b. \exts{may} = \exts{can} = \exts{be-allowed-to} = \dots{} =\\
$\lambda p\in D_{\angles{s,t}}.\ \lambda q\in D_{\angles{s,t}}.\ \exists w\in W\ [p(w)=1\ \&\ q(w)=1]$\marginnote{in set talk: $p\ensuremath{\cap}q\ensuremath{\neq}\ensuremath{\emptyset}$}

On this approach, the epistemic, deontic, etc. ``readings'' of
individual occurrences of modal verbs come about by a combination of
two separate things. The lexical semantics of the modal itself encodes
just a quantificational force, a \emph{relation} between sets of
worlds. This is either the subset-relation (universal quantification;
necessity) or the relation of non-disjointness (existential
quantification; possibility). The covert variable next to the modal
picks up a contextually salient set of worlds, and this functions as
the quantifier's restrictor. The labels ``epistemic'', ``deontic'',
``circumstantial'' etc. group together certain conceptually natural
classes of possible values for this covert restrictor.

Notice that, strictly speaking, there is not just one deontic reading
(for example), but many. A speaker who utters

\ex. You have to be quiet.

might mean: `I want you to be quiet,' (i.e., you are quiet in all
those worlds that conform to my preferences). Or she might mean:
`unless you are quiet, you won't succeed in what you are trying to
do,' (i.e., you are quiet in all those worlds in which you succeed at
your current task). Or she might mean: `the house rules of this
dormitory here demand that you be quiet,' (i.e., you are quiet in all
those worlds in which the house rules aren't violated). And so on. So
the label ``deontic'' appears to cover a whole open-ended set of
imaginable ``readings'', and which one is intended and understood on a
particular utterance occasion may depend on all sorts of things in the
interlocutors' previous conversation and tacit shared assumptions.
(And the same goes for the other traditional labels.)

\subsection{Epistemic vs. Circumstantial Modality}

Is it all context-dependency? Or do flavors of modality correspond to
some sorts of signals in the structure of sentences? Read the
following famous passage from Kratzer and think about how the two
sentences with their very different modal meanings differ in
structure:

\begin{quote}
	
	Consider\marginnote{Quoted from \citet{kratzer:1991:modality}. In
    \citet{kratzer:1981:notional}, the hydrangeas were
    \expression{Zwetschgenbäume} `plum trees'. The German word
    \expression{Zwetschge}, by the way, is etymologically derived from
    the name of the city Damascus (Syria), the center of the ancient
    plum trade.} sentences \Next and \NNext:
	
	\ex. Hydrangeas can grow here.
	
	\ex. There might be hydrangeas growing here.
	
	The two sentences differ in meaning in a way which is illustrated by
  the following scenario.
	
	\medskip ``Hydrangeas''
	
	\medskip Suppose I acquire a piece of land in a far away country and
  discover that soil and climate are very much like at home, where
  hydrangeas prosper everywhere. Since hydrangeas are my favorite
  plants, I wonder whether they would grow in this place and inquire
  about it. The answer is \LLast. In such a situation, the proposition
  expressed by \LLast is true. It is true regardless of whether it is
  or isn't likely that there are already hydrangeas in the country we
  are considering. All that matters is climate, soil, the special
  properties of hydrangeas, and the like. Suppose now that the country
  we are in has never had any contacts whatsoever with Asia or
  America, and the vegetation is altogether different from ours. Given
  this evidence, my utterance of \Last would express a false
  proposition. What counts here is the complete evidence available.
  And this evidence is not compatible with the existence of
  hydrangeas.
	
	\medskip \LLast together with our scenario illustrates the pure
  \term{circumstantial} reading of the modal \expression{can}. [\dots
  ]. \Last together with our scenario illustrates the epistemic
  reading of modals. [\dots] circumstantial and epistemic
  conversational backgrounds involve different kinds of facts. In
  using an epistemic modal, we are interested in what else may or must
  be the case in our world given all the evidence available. Using a
  circumstantial modal, we are interested in the necessities implied
  by or the possibilities opened up by certain sorts of facts.
  Epistemic modality is the modality of curious people like
  historians, detectives, and futurologists. Circumstantial modality
  is the modality of rational agents like gardeners, architects, and
  engineers. A historian asks what might have been the case, given all
  the available facts. An engineer asks what can be done given certain
  relevant facts.
\end{quote}

\noindent Consider also the very different prominent meanings of the
following two sentences, taken from Kratzer as well:

\ex. \a. Cathy can make a pound of cheese out of this can of milk. 
\b. Cathy might make a pound of cheese out of this can of milk.

\begin{exercise}
  Come up with examples of epistemic, deontic, and circumstantial uses
  of the necessity verb \expression{have to}. Describe the set of
  worlds that constitutes the understood restrictor in each of your
  examples. \eex
\end{exercise}

\subsection{Contingency Again}

We messed up. If you inspect the context-dependent meanings we have on
the table now for our modals, you will see that the right hand sides
again do not mention the evaluation world $w$. Therefore, we will
again have the problem of not making contingent claims, indirectly
about the actual world. This needs to be fixed. We need a semantics
that is both context-dependent and contingent.

The problem, it turns out, is with the idea that the utterance context
supplies a \emph{determinate set of worlds} as the restrictor. When I
understand that you meant your use of \expression{must}, in
\expression{you must be quiet}, to quantify over the set of worlds in
which the house rules of our dorm are obeyed, this does not imply that
you and I have to know or agree on which set exactly this is. That
depends on what the house rules in our world actually happen to say,
and this may be an open question at the current stage of our
conversation. What we do agree on, if I have understood your use of
\expression{must} in the way that you intended it, is just that it
quantifies over \emph{whatever set of worlds it may be} that the house
rules pick out.

The technical implementation of this\marginnote{You will of
  course recognize that functions of type $\angles{s,st}$ are simply a
  schönfinkeled version of the \term{accessibility relations} we
  introduced in the previous chapter.} insight requires that we think
of the context's contribution not as a set of worlds, but rather as a
function which for each world it applies to picks out such a set. For
example, it may be the function which, for any world $w$, yields the
set \{$w'$: the house rules that are in force in $w$ are obeyed in
$w'$\}. If we apply this function to a world $w_{1}$, in which the
house rules read ``no noise after 10 pm'', it will yield a set of
worlds in which nobody makes noise after 10 pm. If we apply the same
function to a world $w_{2}$, in which the house rules read ``no noise
after 11 pm'', it will yield a set of worlds in which nobody makes
noise after 11 pm.

Suppose, then, that the covert restrictor of a modal predicate denotes
such a function, i.e., its value is of type $\angles{s,st}$.

\exi. \label{newnewlf} [I' [I must $R_{\angles{n,\angles{s,st}}}$ ] [VP you quiet]]

And the new lexical entries for \expression{must} and \expression{may}
that will fit this new structure are these:

\ex. \a. \exts{must} = \exts{have-to} = \exts{need-to} = \dots{} =\\
$\lambda R\in D_{\angles{s,st}}.\ \lambda q\in D_{\angles{s,t}}.\ \forall w'\in W\ [R(w)(w') =1 \rightarrow q(w')=1]$\marginnote{in set talk: $(R(w)\ensuremath{\subseteq}q$}
\b. \exts{may} = \exts{can} = \exts{be-allowed-to} = \dots{} =\\
$\lambda R\in D_{\angles{s,st}}.\ \lambda q\in D_{\angles{s,t}}.\ \exists w'\in W\ [R(w)(w')=1\ \&\ q(w')=1]$\marginnote{in set talk: $(R(w)\ensuremath{\cap}q\ensuremath{\neq}\ensuremath{\emptyset}$}

Let us see now how this solves the contingency problem.

\ex. Let $w$ be a world, and assume that the context supplies an assignment $g$ such that $g(R_{\angles{17,\angles{s,st}}}) = \lambda w.\ \lambda w'.$ the house rules in force in $w$ are obeyed in $w'$\\[9pt]
\exts{must $R_{\angles{17,\angles{s,st}}}$ you quiet} = \hfill{\tiny (IFA)}\\
\exts{must $R_{\angles{17,\angles{s,st}}}$}$(\lambda w'\ $\exts[w']{you quiet}) = \hfill{\tiny (FA)}\\
\exts{must} (\exts{$R_{\angles{17,\angles{s,st}}}$}) $(\lambda w'\ $\exts[w']{you quiet}) = \hfill{\tiny (lex. entries \expression{you}, \expression{quiet})}\\
\exts{must} (\exts{$R_{\angles{17,\angles{s,st}}}$}) $(\lambda w'.\ $you are quiet in $w'$) = \hfill{\tiny (lex. entry \expression{must})}\\
$\forall w'\in W:$ \exts{$R_{\angles{17,\angles{s,st}}}$}$(w)(w') =1 \rightarrow $ you are quiet in $w'$ = \hfill{\tiny (pronoun rule)}\\
$\forall w'\in W:$ $g$({$R_{\angles{17,\angles{s,st}}}$})$(w)(w') =1 \rightarrow $ you are quiet in $w'$ = \hfill{\tiny (def. of $g$)}\\
$\forall w'\in W$ [the house rules in force in $w$ are obeyed in $w'$ \\
\null\hfill $\rightarrow$ you are quiet in $w'$]

As we see in the last line of \Last, the truth-value of \ref{newnewlf}
depends on the evaluation world $w$.
\begin{exercise}
	
	Describe two worlds $w_{1}$ and $w_{2}$ so that\\
	\exts[w_1,g]{must $R_{\angles{17,\angles{s,st}}}$ you quiet} = 1 and \exts[w_2,g]{must $R_{\angles{17,\angles{s,st}}}$ you quiet} = 0. \eex
\end{exercise}
\begin{exercise}
	
	In analogy to the deontic relation
  $g(R_{\angles{17,\angles{s,st}}})$ defined in \Last, define an
  appropriate relation that yields an epistemic reading for a sentence
  like \expression{You may be quiet}. \eex
\end{exercise}

\subsection{Iteration}

Consider the following example: 

\ex. You might have to leave.

What does this mean? Under one natural interpretation, we learn that
the speaker considers it possible that the addressee is under the
obligation to leave. This seems to involve one modal embedded under a
higher modal. %
\sidepar{There is more to be said about which modals can embed under
  which other modals. See for some discussion the handout mentioned
  earlier:
  \url{http://web.mit.edu/fintel/lsa220-class-2-handout.pdf}.}%
It appears that this sentence should be true in a world $w$ iff some
world $w'$ compatible with what the speaker knows in $w$ is such that
every world $w''$ in which the rules as they are in $w'$ are followed
is such that you leave in $w''$.

Assume the following LF:

\exi. [I$'$ [ might $R_{\angles{1,\angles{s,st}}}$] [VP [ have-to $R_{\angles{2,\angles{s,st}}}$] [IP you leave]]]

\clearpage
Suppose $w$ is the world for which we calculate the truth-value of %
\sidepar{From now on, we will omit the type-designation of variables
  whenever we feel confident that their type is easy to figure out
  from the context.}%
the whole sentence, and the context maps $R_1$ to the function which maps
$w$ to the set of all those worlds compatible with what is known in
$w$. \expression{might} says that some of those worlds are worlds $w'$
that make the tree below \expression{might} true. Now assume further
that the context maps $R_2$ to the function which assigns to any such
world $w'$ the set of all those worlds in which the rules as they are
in $w'$ are followed. \expression{have to} says that all of those
worlds are worlds $w''$ in which you leave.

In other words, while it is not known to be the case that you have to
leave, for all the speaker knows it might be the case.
\begin{exercise}
	
	Describe values for the covert \angles{s,st}-variable that are
  intuitively suitable for the interpretation of the modals in the
  following sentences:
	
	\ex. As far as John's preferences are concerned, you
  \expression{may} stay with us.
	
	\ex. According to the guidelines of the graduate school, every PhD
  candidate \expression{must} take 9 credit hours outside his/her
  department.
	
	\ex. John \expression{can} run a mile in 5 minutes.
	
	\ex. This \expression{has} to be the White House.
	
	\ex. This elevator \expression{can} carry up to 3000 pounds.
	
	For some of the sentences, different interpretations are conceivable
  depending on the circumstances in which they are uttered. You may
  therefore have to sketch the utterance context you have in mind
  before describing the accessibility relation. \eex
\end{exercise}
\begin{exercise}
	
	Collect two naturally occurring examples of modalized sentences
  (e.g., sentences that you overhear in conversation, or read in a
  newspaper or novel -- not ones that are being used as examples in a
  linguistics or philosophy paper!), and give definitions of values
  for the covert \angles{s,st}-variable which account for the way in
  which you actually understood these sentences when you encountered
  them. (If the appropriate interpretation is not salient for the
  sentence out of context, include information about the relevant
  preceding text or non-linguistic background.) \eex
\end{exercise}

\subsection{A technical variant of the analysis}\label{techvariant}

In our account of the contingency of modalized sentences, we adopted
lexical entries for the modals that gave them world-dependent
extensions of type \angles{\angles{s,st}, \angles{st,t}}:

\ex. (repeated from earlier):\\
For any $w \in W$: \exts{must}\\
$\lambda R\in D_{\angles{s,st}}.\ \lambda q\in D_{\angles{s,t}}.\ \forall w'\in W\ [R(w)(w') =1 \rightarrow q(w')=1]$\\
\null\hfill(in set talk: $\lambda R_{\angles{s,st}}.\ \lambda q_{\angles{s,t}}.\ (R(w)\ensuremath{\subseteq}q)$).

Unfortunately, this treatment somewhat obscures the parallel between
the modals and the quantificational determiners, which have
world-independent extensions of type \angles{et, \angles{et,t}}.

Let's explore an alternative solution to the contingency problem,
which will allow us to stick with the world-independent
type-\angles{st,\angles{st,t}}-extensions that we assumed for the
modals at first:

\ex. (repeated from even earlier): \\
\exts{must} = $\lambda p\in D_{\angles{s,t}}.\ \lambda q\in D_{\angles{s,t}}.\ \forall w\in W\ [p(w)=1 \rightarrow q(w)=1]$\\
\null\hfill(in set talk: $\lambda p \in D_{\angles{s,t}}.\ \lambda q \in D_{\angles{s,t}}.\ p \ensuremath{\subseteq} q$).

We posit the following LF-representation:

\exi. \label{brandlf} [I$'$ [I must [ $R_{\angles{4,\angles{s,st}}}$ w*]] [VP you quiet]]

What is new here is that the covert restrictor is complex. The first
part, $R_{\angles{4,\angles{s,st}}}$, is (as before) a free variable
of type \angles{s,st}, which gets assigned an accessibility relation
by the context of utterance. %
\sidepar{Note that as soon as we're introducing an object language
  expression whose extension is a possible world, we will now have
  expressions of type $s$ and should also introduce the domain of
  things of type $s$: $D_s = W$.}%
The second part is a special terminal symbol which is interpreted as
picking out the evaluation world:

\ex. For any $w\in W: \sv{w*}^{w,g} = w$.\footnote{\citet{dowty:1982:time-adverbs} introduced an analogous symbol to pick out the evaluation \emph{time}. We have chosen the star-notation to allude to this precedent.}

When $R_{\angles{4,\angles{s,st}}}$ and \expression{w*} combine (by
Functional Application), we obtain a constituent whose extension is of
type \angles{s,t} (a proposition or set of worlds). This is the same
type as the extension of the free variable $p$ in the previous
proposal, hence suitable to combine with the old entry for
\expression{must} (by FA). However, while the extension of $p$ was
completely fixed by the variable assignment, and did not vary with the
evaluation world, the new complex constituent's extension depends on
both the assignment and the world:

\ex. For any $w\in W$ and any assignment $g$: \\
\exts[w,g]{$R_{\angles{4,\angles{s,st}}}$(w*)} = g(\angles{4,\angles{s,st}})($w$).

As a consequence of this, the extensions of the higher nodes I and
I$'$ will also vary with the evaluation world, and this is how we
capture the fact that \ref{brandlf} is contingent.

Maybe this variant is more appealing. But for the rest of this
chapter, we continue to assume the original analysis as presented
earlier. In the next chapter on conditionals, we will however make
crucial use of this way of formulating the semantics for modals. So,
make sure you understand what we just proposed.

\section{*Kratzer's Conversational Backgrounds} \label{sec:kratz-conv-backgr}

Angelika Kratzer has some interesting ideas on how accessibility
relations are supplied by the context. She argues that what is really
floating around in a discourse is a \term{conversational background}.
Accessibility relations can be computed from conversational
backgrounds (as we shall do here), or one can state the semantics of
modals directly in terms of conversational backgrounds (as Kratzer
does).

A conversational background is the sort of thing that is identified by
phrases like \expression{what the law provides, what we know}, etc.
Take the phrase \expression{what the law provides}. What the law
provides is different from one possible world to another. And what the
law provides in a particular world is a \emph{set of propositions}.
Likewise, what we know differs from world to world. And what we know
in a particular world is a set of propositions. The intension of
\expression{what the law provides} is then that function which assigns
to every possible world the set of propositions $p$ such that the law
provides in that world that $p$. Of course, that doesn't mean that $p$
holds in that world itself: the law can be broken. And the intension
of \expression{what we know} will be that function which assigns to
every possible world the set of propositions we know in that world.
Quite generally, conversational backgrounds are functions of type
\angles{s,\angles{st,t}}, functions from worlds to (characteristic
functions of) sets of propositions.

Now, consider:

\ex. (In view of what we know,) Brown must have murdered Smith.

The \expression{in view of}-phrase may explicitly signal the intended
conversational background. Or, if the phrase is omitted, we can just
infer from other clues in the discourse that such an epistemic
conversational background is intended. We will focus on the case of
pure context-dependency.

How do we get from a conversational background to an accessibility
relation? Take the conversational background at work in \Last. It will
be the following:

\ex. $\lambda w.\ \lambda p.\ p$ is one of the propositions that we know in $w$.

This conversational background will assign to any world $w$ the set of
propositions $p$ that in $w$ are known by us. So we have a set of
propositions. From that we can get the set of worlds in which all of
the propositions in this set are true. These are the worlds that are
compatible with everything we know. So, this is how we get an
accessibility relation:

\ex. \label{convers} For any conversational background f of type \angles{s,\angles{st,t}}, we define the corresponding accessibility relation $R_{f}$ of type \angles{s,st} as follows: \\
$R_{f} := \lambda w.\ \lambda w'.\ \forall p\ [ f(w)(p)=1\ \rightarrow\ p(w')=1 ]$.

\enlargethispage{24pt}In words, $w'$ is $f$-accessible from $w$ iff all propositions $p$ that are assigned by $f$ to $w$ are true in $w'$.

Kratzer uses the term \term{modal base} for the conversational
background that determines the set of accessible worlds. We can be
sloppy and use this term for a number of interrelated concepts:

\begin{enumerate}[(i)] 
	\item the conversational background (type \angles{s,\angles{st,t}}), 
	\item the set of propositions assigned by the conversational
    background to a particular world (type \angles{st,t}),
	\item the accessibility relation (type \angles{s,st}) determined by (i), 
	\item the set of worlds accessible from a particular world (type \angles{s,t}). 
\end{enumerate}

\noindent Kratzer calls a conversational background (modal base)
\term{realistic} iff it assigns to \emph{any} world a set of
propositions that are all true in that world. The modal base
\expression{what we know} is realistic, the modal bases
\expression{what we believe} and \expression{what we want} are not.

\medskip
\noindent What follows are some (increasingly technical exercises) on
conversational backgrounds.

\begin{exercise}
	
	Show that a conversational background $f$ is realistic iff the
  corresponding accessibility relation $R_{f}$ (defined as in \Last)
  is reflexive. \eex
\end{exercise}

\begin{exercise}
	
	Let us call an accessibility relation \term{trivial} if it makes
  every world accessible from every world. $R$ is \term{trivial} iff
  $\forall w\forall w'\!:\ w'\in R(w)$. What would the conversational
  background $f$ have to be like for the accessibility relation
  $R_{f}$ to be trivial in this sense? \eex
\end{exercise}

\begin{exercise}\label{closure}
	
	The definition in \Last specifies, in effect, a function from
  $D_{\angles{s,\angles{st,t}}}$ to $D_{\angles{s,st}}$. It maps each
  function $f$ of type \angles{s,\angles{st,t}} to a unique function
  $R_{f}$ of type \angles{s,st}. This mapping is not one-to-one,
  however. Different elements of $D_{\angles{s,\angles{st,t}}}$ may be
  mapped to the same value in $D_{\angles{s,st}}$.\footnote{In this
    exercise, we systematically substitute sets for their
    characteristic functions. I.e., we pretend that $D_{\angles{s,t}}$
    is the power set of $W$ (i.e., elements of $D_{\angles{s,t}}$ are
    sets of worlds), and $D_{\angles{st,t}}$ is the power set of
    $D_{\angles{s,t}}$ (i.e., elements of $D_{\angles{st,t}}$ are sets
    of sets of worlds). On these assumptions, the definition in \Last
    can take the following form:
	
	\ex. For any conversational background $f$ of type \angles{s,\angles{st,t}}, \\
	we define the corresponding accessibility relation $R_{f}$ of type \angles{s,st} as follows:\\
	$R_{f} := \lambda w.\ \{w':\ \forall p\ [p\in f(w)\ \rightarrow\ w'\in p] \}$.
	
	The last line of this can be further abbreviated to:
	
	\ex. $R_{f} := \lambda w.\ \ensuremath{\cap}f(w)$
	
	This formulation exploits a set-theoretic notation which we have
  also used in condition \TextNext of the second part of the exercise.
  It is defined as follows:
	
	\ex. If $S$ is a set of sets, then $\ensuremath{\cap}S\ :=\ \{x:\ \forall Y\ [Y\in S\ \rightarrow\ x\in Y]\}$.
	
	}
	\begin{itemize}
		
  \item Prove this claim. I.e., give an example of two functions f and
    f' in $D_{\angles{s,\angles{st,t}}}$ for which \Last determines
    $R_{f}$ = $R_{f'}$ .
		
  \item As you have just proved, if every function of type
    \angles{s,\angles{st,t}} qualifies as a `conversational
    background', then two different conversational backgrounds can
    collapse into the same accessibility relation. Conceivably,
    however, if we imposed further restrictions on conversational
    backgrounds (i.e., conditions by which only a proper subset of the
    functions in $D_{\angles{s,\angles{st,t}}}$ would qualify as
    conversational backgrounds), then the mapping between
    conversational backgrounds and accessibility relations might
    become one-to-one after all. In this light, consider the following
    potential restriction:
		
		\ex. Every conversational background $f$ must be ``closed under entailment''; i.e., it must meet this condition:\\
		$\forall w. \forall p\ [ \ensuremath{\cap}f(w) \ensuremath{\subseteq} p\ \rightarrow\ p \in f(w) ]$.
		
		(In words: if the propositions in $f(w)$ taken together entail
    $p$, then $p$ must itself be in $f(w)$.) Show that this
    restriction would ensure that the mapping defined in \LLast will
    be one-to-one. \eex
	\end{itemize}
\end{exercise}

%\section{Embedding Modals}

\section{Supplementary Readings} \label{sec:suppl-read-modals}{\setlength{\parindent}{0pt}\setlength{\parskip}{6pt}

  The most important background readings for this chapter are the
  following two papers by Kratzer:

\begin{bibentrylist}
	\item \bibentry{kratzer:1981:notional}.
	\item \bibentry{kratzer:1991:modality}.
\end{bibentrylist}

Kratzer has been updating her classic papers for a volume of her
collected work on modality and conditionals. These are very much worth
studying: \url{http://semanticsarchive.net/Archive/Tc2NjA1M/}.

A major new resource on modality is Paul Portner's book:

\begin{bibentrylist}
  \item\bibentry{portner:2009:modality-book}.
\end{bibentrylist}

You might also profit from other survey-ish type papers:

\begin{bibentrylist}
  \item \bibentry{fintel:2005:modality}.
  \item \bibentry{fintel-gillies:2007:ose2}.
  \item \bibentry{swanson:2008:modality}.
  \item \bibentry{hacquard:2009:hsk-modality}.
\end{bibentrylist}

On the syntax of modals, there are only a few papers of uneven
quality. Some of the more recent work is listed here. Follow up on
older references from the bibliographies in these papers.

\begin{bibentrylist}
	\item \bibentry{bhatt:1997:haveto}. 
	\item \bibentry{wurmbrand:1999:raising}. 
	\item \bibentry{cormack-smith:2002:modals}. 
	\item \bibentry{butler:2003:modality}. 
\end{bibentrylist}

The following paper explores some issues in the LF-syntax of epistemic
modals:

\begin{bibentrylist}
	\item \bibentry{fintel-iatridou:2003:ec}.
\end{bibentrylist}

Valentine Hacquard's MIT dissertation is a rich source of
cross-linguistic issues in modality, as is Fabrice Nauze's Amsterdam
dissertation:

\begin{bibentrylist}
  \item\bibentry{hacquard:2006:dissertation}.
  \item \bibentry{nauze:2008:thesis}.
\end{bibentrylist}

The semantics of epistemic modals has become a hot topic recently.
Here are some of the main references:

\begin{bibentrylist}
	\item \bibentry{hacking:1967:possibility}.
	\item \bibentry{teller:1972:epistemic}.
	\item \bibentry{derose:1991:epistemic}. 
	\item \bibentry{egan-hawthorne-weatherson:2005:epistemic}. 
	\item \bibentry{egan:2007:epistemic}. 
	\item \bibentry{macfarlane:2006:might}. 	
	\item \bibentry{stephenson:2007:judge-lp}.
	\item \bibentry{hawthorne:2007:danger}.
	\item \bibentry{fintel-gillies:2008:cia-leaks}.
	\item \bibentry{fintel-gillies:2008:mmr}. 
\end{bibentrylist}

A recent SALT paper by Pranav Anand and Valentine Hacquard tackles
what happens to epistemic modals under attitude predicates:

\begin{bibentrylist}
	\item \bibentry{anand-hacquard:2008:epistemics}.
\end{bibentrylist}

Evidentiality is a topic closely related to epistemic modality. Some
references:

\begin{bibentrylist}
  \item \bibentry{willett:1988:evidentials}.
  \item \bibentry{aikhenvald:2004:evidentiality}.
  \item \bibentry{drubig:2001:epistemic}.
  \item \bibentry{blain-dechaine:2007:evidentials}.
  \item \bibentry{mccready-ogata:2007:evidentials}.
  \item \bibentry{speas:2008:evidentials}.
  \item \bibentry{fintel-gillies:2010:mss}. 
\end{bibentrylist}

Modals interact with disjunction and indefinites to generate so-called
\term{free choice}-readings, which are a perennial puzzle. Here is
just a very small set of initial references:

\begin{bibentrylist}
	\item \bibentry{kamp:1973:freechoice}.
	\item \bibentry{zimmermann:2000:fc-disjunction}.
	\item \bibentry{schulz:2005:fcp-synthese}.
	\item \bibentry{aloni:2007:freechoice}.
	\item \bibentry{alonso-ovalle:2006:thesis}.
	\item \bibentry{fox:2007:freechoice}.
	\item \bibentry{rooij:2006:donkeys}.
\end{bibentrylist}

