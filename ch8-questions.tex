% chktex-file 1
\providecommand{\ans}{\textsc{ans}\xspace}
\providecommand{\dox}{\textsc{dox}\xspace}
\chapter{Interrogative clauses}\label{cha:interrogatives}

\minitoc

So far, we have developed a semantic system that compositionally derives
semantic values for ``declarative'' clauses, whether standing on their own as
main or matrix sentences or embedded as constituent clauses (relative clauses,
complement clauses, adjunct clauses). Extensionally speaking, declaratives are
of type $t$ and their denotation is determined relative to an evaluation index
(a world, or a world-time pair) and a variable assignment: $\sv{\phi}^{i,g}$.
The system also supplies (possibly relative to a variable assignment) an
intension of type $\type{s,t}$: %
\note{In set talk: a set of indices. In the simplest system: a set of worlds.}%
a proposition, or a function from indices to truth-values,
$\svcent{\phi}^{g}=\lambda i.\sv{\phi}^{i,g}$.

\note{Another important kind of non-declarative are imperatives, if you're
  interested in which, you should consult
  \cite{portner-2007-imperatives-modals}, \cite{kaufmann-2012-imperatives},
  \cite{fintel-iatridou-2017-modest}.}%
In this chapter, we will expand the system to deal with a central kind of
non-declarative clause: interrogatives. These also can stand on their own:

\pex
\a Did Sakina see Emily?
\a Did Sakina see Emily or did Sakina see Julie?
\a Who did Sakina see?
\xe
%
\kwn And they can appear as embedded clauses:

\pex
\a Melanie wonders whether Sakina saw Emily.
\a Inessa is surprised who Sakina saw.
\a Whether Sakina saw Emily or not, she definitely saw Julie.
\xe

The semanticist's task in the analysis of interrogative clauses is to propose
suitable types of semantic values for them and to show how these semantic values
are built up compositionally. But before we apply ourselves to this task, it is
useful to say a little bit about speech acts, and about the relation between
semantic values and speech acts.

\section{A little pragmatics}
\label{sec:pragmatics}

Leaving semantics out of the picture for the moment, interrogative clauses
differ from declarative clauses in their syntax and in their pragmatics. In
English, interrogative clauses can be told apart syntactically from declarative
clauses by the presence of a clause-initial wh-phrase and/or the inverted order
of subject and auxiliary. The terminology ``interrogative''/``declarative'',
however, alludes to a distinction not in grammatical form but in communicative
function. When uttered as main clauses (i.e., not embedded in a larger
structure), declarative sentences typically serve to make assertions, whereas
interrogative sentences serve to ask questions. These two clause-types serve to
perform different kinds of speech acts.

\cite{stalnaker-1978-assertion} outlined an influential formal model of what
happens in a conversation and what it means to make an assertion. The central
concept is that of a body of publicly shared information, or ``common ground'',
which evolves as the conversation proceeds. A proposition is in the common
ground if each interlocutor is ``disposed to act as if he assumes or believes
the proposition is true, and as if he assumes or believes that his audience
assumes or believes that it is true as well''. The common ground can be
characterized by the set of worlds in which every proposition in the common
ground is true; Stalnaker calls this set of worlds the ``context set''. The act
of making an assertion is a proposal to update \dash in fact, to shrink \dash
the context set. %
\note{Not ``shrink'' in the sense of reducing cardinality. The cardinality of
  the context set may never become less than uncountably infinite.}%
The particular way in which the context set is to be shrunk depends on the
semantic value of the asserted sentence, more specifically, on its intension. To
assert a sentence $\phi$ is to propose that the current context set $c$ be
replaced by a new context set which is the intersection of $c$ with the
intension of $\phi$, i.e., %
\note{We're leaving off the variable assignment parameter, which is legitimate
  if there are no free variables in $\phi$.}%
$c \cap \svcent{\phi}$.

Against this backdrop, how might we think about the speech act of asking a
question? What is the point of this speech act? %
\note{At least this is true if we stick to questions that don't have
  presuppositions (or to questions whose presuppositions are already common
  ground at the point when they are asked). When a question that has a
  presupposition is asked in a context where the presupposition is not yet in
  the common ground, a listener can get new information by accommodating this
  presupposition.}%
Questions, unlike assertions, don't provide information about the world and
therefore do not in themselves lead to updates of the common ground. Their
purpose rather is to constrain the future course of the conversation in a
certain way. In the absence of any particular question under consideration, a
speaker might assert whatever they find interesting or useful. But if someone
asks a question, they thereby express an expectation that the next
conversational move will be one out of a very limited set of possible assertions
\dash intuitively, the next assertion will ``address the question'' rather than
provide some random other information.

Here is a way to formally model this intuition. A question partitions the
context set in a certain way, and it amounts to a request for assertions which,
so to speak, ``carve up'' the context set along the lines determined by this
partition. The requested next assertion should not make any distinctions between
worlds which the partition lumps together.

Here's the formal definition of a partition:

\begin{definition}[Partition]\label{def:partition}
  A set $\mathcal{P}$ is a \emph{partition} of a set $A$ iff

  \begin{enumerate}[(i)]
    \item every element of $\mathcal{P}$ is a non-empty subset of $A$,
    \item any two non-identical elements of $\mathcal{P}$ are disjoint from each
          other, and
    \item the union of all the elements of $\mathcal{P}$ is $A$.
  \end{enumerate}

  The elements of a partition are also called its cells.
\end{definition}

\note{The idea of refining the Stalnakerian model of the context by moving from
  the context set to a partitioned context set goes back to
  \cite{groenendijk-1999-interrogation}. A formally equivalent way is to model
  the context as an equivalence relation (see, for example,
  \cite{jaeger-1996-only-updates} and \cite{hulstijn-1997-issues}). We will
  later see approaches that give a partition semantics to interrogatives. It's
  important to distinguish the two moves. As we'll see in this section, one can
  have a partition pragmatics without a partition semantics. This theme is
  explored in much more sophisticated detail in \cite{fox-2018-partition-exh}.}%
\note{\cite{starr-2020-imperatives} adds to the partitioned context set model a
  way of tracking the speech act effect of imperatives.}%
\note{Our proposed context model is simple compared to some other well-known
  systems, such as \cite{roberts-2012-information-structure} and
  \cite{farkas-bruce-2010-reacting}.}%
With this definition in hand, we refine Stalnaker's model of conversation. Each
stage in a conversation is now characterized not by its context set, but by its
``partitioned context set'', i.e., by a partition of some set of possible
worlds. The union of this partition corresponds to Stalnaker's old context set,
i.e., it contains the worlds compatible with every proposition in the common
ground. The partitioning represents the interlocutors' shared commitment not to
make distinctions between worlds that are ``cell-mates'' \dash at least not for
the time being. This is a renegotiable commitment and, as we will see, it only
stays in force until someone raises a new question. The definition of ``relevant
assertion'' in Definition \ref{def:relevance} below spells out precisely what
the commitment amounts to. In Definition \ref{def:update-by-assertion}, we adapt
Stalnaker's characterization of the context-changing role of assertions to our
new, more elaborate set-up.

\begin{definition}[Update by assertion]\label{def:update-by-assertion}
  To assert a sentence $\phi$ is to propose that the current partitioned context
  set $C$ be replaced by a new partitioned context set which is constructed by
  intersecting each cell of $\mathcal{C}$ with the intension of $\phi$. More
  precisely, $\mathcal{C}$ is to be replaced by

  $$\{p\co p\neq\emptyset\ \&\ \exists p'\in \mathcal{C}.\ p = p' \cap \svcent{\phi}\}$$
\end{definition}

\begin{definition}[Relevance]\label{def:relevance}
  An assertion is \emph{relevant} w.r.t. to a partitioned context set
  $\mathcal{C}$ iff the update it proposes results in a subset of $\mathcal{C}$.
\end{definition}

The notion of ``relevance'' that is formalized here calls for a bit of
clarification. First, here are a couple of more commonly used (and more
transparent) definitions.

\begin{definition*}[Relevance$'$]\label{def:relevance-a}
  A proposition $p$ (i.e. set of worlds) is relevant w.r.t. to a partition
  $\mathcal{Q}$ of a set of worlds iff $p$ is identical to a cell in
  $\mathcal{Q}$ or to a union of cells in $\mathcal{Q}$.
\end{definition*}

\begin{definition*}[Relevance$''$]\label{def:relevance-b}
  A proposition $p$ is relevant w.r.t. to a partition $\mathcal{Q}$ of a set of
  worlds iff there is no pair of worlds $w$ and $w'$ such that both

  \begin{enumerate}[(i)]
    \item there is a cell $c \in \mathcal{Q}$ such $w \in c\ \&\ w' \in c$\\
    (i.e., $w$ and $w'$ are cell-mates),
    \item and $w \in p\ \&\ w' \in p$\\
    (i.e. $p$ has different truth values in $w$ and $w'$).
  \end{enumerate}
\end{definition*}

These two definitions differ slightly in that Definition~\ref{def:relevance-b}
(unlike Definition~\ref{def:relevance-a}) allows $p$ to include arbitrary worlds
that are not in any cell of $\mathcal{Q}$ at all. But both express the essential
requirement that a relevant proposition must not discriminate among cell-mates.
It must not ``cut up'' the space of worlds (in the union of the partition) in
any way that does not coincide with the boundaries between cells.

Definitions \ref{def:relevance-a} and \ref{def:relevance-b} apply to a
proposition, whereas our Definition \ref{def:relevance} applies to an assertion
and presupposes the rule for update-by-assertion. But apart from this
difference, Definition~\ref{def:relevance} captures the same notion of
relevance. (To make the connections fully precise: If $\mathcal{C}$ is a
partitioned context set and $\phi$ is an asserted sentence, then the assertion
of $\phi$ is ``relevant'' w.r.t. $\mathcal{C}$ in the sense of
Definition~\ref{def:relevance} iff $\bigcup \mathcal{C} \cap \svcent{\phi}$ is
``relevant'' w.r.t. $\mathcal{C}$ in the sense of
Definition~\ref{def:relevance-a}, which holds, in turn, iff $\svcent{\phi}$ is
``relevant'' w.r.t. $\mathcal{C}$ in the sense of
Definition~\ref{def:relevance-b}. Prove these equivalences as an exercise.)

To get an intuitive appreciation of this concept, it is important to see that
``relevance'' in this technical sense is a purely negative requirement, so to
speak: it is just the absence of any irrelevant (i.e., not specifically asked
for) information. By this criterion, a ``relevant'' answer may be totally
uninformative; it could be a tautology, or any other proposition that is already
entailed by the common ground. So evidently, relevance is not a sufficient
condition for appropriate answers; there are additional conditions, notably that
an answer \dash or any assertion, for that matter \dash should be informative.
It is also worth pointing out that the formal requirement of relevance is
\emph{prima facie} violated by many answers which in actual fact are judged
felicitous, e.g., when somebody responds to the question \emph{Is it raining?}
by saying: \emph{Well, Becky just came in dripping wet}, or \emph{Maybe}, or
\emph{I don't know}. To make sense of this wider range of intuitively felicitous
answers, a proponent of the present formal model needs to say that in these
cases, the answerer is implicitly changing or amending the question that was
explicitly asked, and instead is addressing a different (though related)
question that they don't formulate explicitly but expect their audience to infer
(``accommodate'').

Now the only piece of our story that is missing is a recipe for ``update by
question''. The idea, as we have said, is that asking a question amounts to
proposing a specific replacement of the current partitioned context set by a new
one. As in the case of assertions, the construction of the new partitioned
context set should be determined by the semantic value of the sentence that was
uttered. We know what we have in mind for particular examples. E.g., when
someone asks \emph{Is it raining?}, the intended outcome is a two-membered
partition, with one cell consisting of worlds in which it is raining and the
other cell of worlds in which it is not raining. If the question is a so-called
``alternative question'' like \emph{Is Julie in Barcelona or did Emily call them
  on the phone?}, the result should be a four-celled partition corresponding to
the four possible configurations of truth-values for the two component
sentences. If the question asked is \emph{Who among these two (Carli and Jodie)
  came to the party?}, the result should be a partition with four cells: one
with worlds in which Carli and Jodie both came, one with worlds in which Carli
but not Jodie came, one with worlds in which Jodie but not Carli came, and one
with worlds in which neither Carli nor Jodie came.

What is the general recipe by which these outcomes are obtained as a function of
the semantic value of each interrogative sentence? To answer this question, we
have to carry out the semanticist's task, i.e., assign semantic values to
interrogative clauses. We will do so in the next section, but for now we will
articulate a basic intuition about the semantic value of interrogatives and
explore how they can be used to construct a partitioned context set.

Perhaps the most widespread idea about the semantic value of interrogatives is
that they denote \emph{sets of propositions}. The particular structural features
of interrogatives are taken to serve to both ``lift'' the type of the denotation
from the type of propositions (the denotation of declaratives) to the type of
sets of propositions, and to generate the individual propositions in the set. We
will look at the mechanics of this in the next section. For now, let's assume
the following semantic value for the sample \emph{wh}-questions \emph{Who among
  these two (Carli and Jodie) came to the party?}:

\ex
$\svt{Who among these two (Carli and Jodie) came to the party?} =$\\
$\{\lambda w.\ \text{Carli came to the party in}\ w, \lambda w.\ \text{Jodie
  came to the party in}\ w\}$
\xe

Now, notice that the propositions in this set are not disjoint (they can both be
true of the same world) and that their union leaves out the worlds where neither
Carli nor Jodie came to the party. So, this set of propositions is not a
partition of the set of all possible worlds and also not a partition of any
context set with minimal information about who came to the party. So, there's
apparently not a seamless link between our partition pragmatics for question
acts and our semantic values for interrogatives.

Should we therefore redo our semantics so that it does, after all, map
interrogative clauses to semantic values that are partitions? This can be done
and has been argued for, and we will return to a discussion of this possibility.
For now, however, we observe that it is not necessary. An equally reasonable
take on the discrepancy between our semantics and our pragmatics is that it
simply reflects the division of labor between semantics and pragmatics. We don't
necessarily need a semantics that directly delivers partitions as semantic
values. All we need is to make precise how the semantic values delivered by the
semantics help determine the partitions that figure in the pragmatics. In other
words, we need to formulate our pragmatic rule for ``update by question'' in
such a way that it spells out how the partition effected by a question-act is
determined by the semantic value of the uttered sentence. We will work up to
this formulation in a few steps. As we will see, there is a straightforward and
general recipe for converting an arbitrary set of propositions into a partition
of a given set of worlds, and our update principle will make use of that recipe.

Informally speaking, each proposition in the question-denotation defines a
``dividing line'' across the space of worlds in the context set, and as we use
multiple propositions to draw multiple lines across this space, we get
increasingly fine-grained partitions as the number of propositions goes up. %
\note{This assumes that the propositions are all ``logically independent'' of
  each other and of the propositions in the original common ground. That is,
  none of them contradicts or is entailed by any other (or any conjunction or
  disjunction of others). If there are logical relations between the
  propositions, then the total number of cells is smaller. Note that the
  definition of ``partition'' requires all cells to be non-empty, so the
  intersection of two incompatible propositions is not a cell.}%
The first proposition cuts the space into two cells (the region where it is true
and the region where it is false). The second proposition subdivides these two
cells further into four; the third proposition leaves us with eight cells; and
so on.

Another way of describing this process is that each new proposition introduces a
new condition that cell-mates must satisfy. At the beginning, before any line
has been drawn, all of the context set is one big cell and every world counts as
a cell-mate of every other one. Then we draw a line for the first proposition
(call it $p_{1}$), and if two worlds differ with respect to the truth-value of
$p_{1}$, they now are no longer cell-mates. $w$ and $w'$ are cell-mates now only
if $p_{1}(w) = p_{1}(w')$. With the second proposition, $p_{2}$, we draw a
second line and ``break up'' yet more of the original cell- mate relations. At
this point, only those pairs of worlds remain cell-mates which are treated alike
by both $p_{1}$ and $p_{2}$. I.e., $w$ and $w'$ are cell-mates now iff
$p_{1}(w) = p_{1}(w')\ \&\ p_{2}(w) = p_{2}(w')$. And so on for the third and
all other propositions in the given set of propositions.

So a set of propositions can be used to define a ``cell-mate relation'', and
thus a partition.

\pex Let $\mathcal{P}$ be a set of propositions and $c$ a set of worlds.%
\label{ex:cellmate-partition}

\a $\sim_{\mathcal{P},c}$, the cellmate relation in $c$ based on $\mathcal{P}$, is defined as
follows:\\
$\forall w,w'\co w \sim_{\mathcal{P},c} w'$ iff
$w\in c\ \&\ w'\in c\ \&\ \forall p\in \mathcal{P}\co p(w)=p(w')$.

\a $\textsc{parts}(\mathcal{P},c)$, the partition of $c$ based on $\mathcal{P}$,
is defined as
follows:\\
$\textsc{parts}(\mathcal{P},c) = \bigl\{p\co \exists w\in c. p = \{w'\co w \sim_{\mathcal{P},c} w'\}\bigr\}$
\xe
%
The cellmate-relation is an equivalence relation (reflexive, symmetric, and
transitive). It holds between any two worlds that give the same truth-value to
each proposition in the ``seed set''. The corresponding partition consists of
the subsets of $c$ that are made up of worlds that are cell-mates.

\note{Note that this update rule implies that each new question ``wipes out''
  the current partitioning of the context set and replaces it by a new one. This
  may not be right and isn't the only way we can go. We may want a system where
  successive questions can serve to \emph{refine} the partition. Even more
  complex possibilities exists, as for example in
  \cite{roberts-2012-information-structure}.}%
%
We are ready to propose a preliminary rule of ``Update by question'':

\begin{restatable}[Update by question (draft)]{definition}{updatebyquestiondraft}%
  \label{def:update-by-question-draft}%
  To ask a question by uttering a sentence $\phi$ that denotes a set of
  propositions $\mathcal{P}$ is to propose that the current partitioned context
  set $\mathcal{C}$ be replaced by the new partitioned context set
  $\textsc{part}(\mathcal{P},\bigcup \mathcal{C})$.
\end{restatable}

\note{The implementation here differs from \cite{karttunen-1977-questions} in
  not using the Montague Grammar framework; instead we presuppose a division of
  labor between syntax and semantics in line with
  \cite{heim-kratzer-1998-semantics}. Aside from implementation, there are some
  small substantive differences as well. Most saliently, the extensions we will
  assign to interrogative clauses will include both true and false propositions.
  (In \cite{hagstrom-2003-GLOT-questions}'s terminology, they will correspond to
  ANSPOSS rather than ANSTRUE.) Moreover, we will depart from Karttunen in the
  treatment of polar questions.

  Karttunen's motivation for including a restriction to true propositions had to
  do with the semantics of question-embedding constructions. So the time to
  discuss it is when we get to embedded questions. Right now we focus on matrix
  questions.}%
In the next section, we turn to the task of compositionally deriving the
semantic value of an interrogative.

\section{Compositional computation of semantic values for interrogative clauses}
\label{sec:compositional-interrogative}

\subsection{An updated version of \cite{karttunen-1977-questions}}%
\label{sec:karttunen-1977}

Interrogative clauses are built largely with the same lexicon and syntactic
rules as declarative clauses; but they also contain certain
interrogative-specific morphemes, syntactic features, or functional heads that
make crucial contributions to their non-declarative meanings. Following
\cite{karttunen-1977-questions}, our syntax for English posits an abstract
(i.e., silent) complementizer that has a non-vacuous semantics, and a feature on
wh-phrases that is semantically vacuous but subject to a certain distributional
constraint.
%
\note{The official denotation in \refx{ex:?} maps propositions to functions from
  propositions to truth-values, or equivalently, it maps propositions to
  characteristic functions of sets of propositions. If we replaced these
  characteristic functions by the corresponding sets, (\ref{ex:?}) would read as
  follows:

  \ex[exno={\ref{ex:?}'}]%
  $\sv{?}^{w} = \lambda p.\{p\}$
  \xe
%
  In other words, it would be a function that maps each proposition to the
  singleton set that has it as its only member.}%
\ex\label{ex:?}%
Karttunen's ``proto-question'' operator, syntactically a C-head:\\
$\sv{?}^{w} = \lambda p_{st}.\lambda q_{st}.\ p=q$.
\xe

\ex lexical entries for interrogative words, e.g.:\\
$\sv{\text{who}^{[\textsc{wh}]}}^{w} = \lambda f_{et}.\ \exists x\ [x\ \text{is
  human in}\ w\ \&\ f(x)=1]$
\xe
%
The meaning of \emph{who} is exactly the same as the meaning of \emph{somebody}.
We will address the role of the feature [\textsc{wh}] in
Section~\ref{sec:wh-syntax}.

In the syntactic derivation of a constituent question, wh-movement applies and
puts the wh-word above the operator in C. For the example \emph{Who did Sakina
  see?} or \emph{who Sakina saw} (we will ignore tense and auxiliary \emph{do},
and thereby ignore the difference between matrix and embedded versions), this
gives rise to a structure like \Next:

\ex who 2 [? Sakina see $t_{2}$] \xe
%
Unfortunately, when we examine the semantic types at each node in \Last, we
discover a type-mismatch that prevents us from interpreting the top-most node.
We fix this problem by positing a slightly more complex C-head: We base-generate
\emph{?} together with another, covert operator as its sister. %
\note{Other vacuous covert operators of this sort are H\&K's relative pronouns
  (and their PRO in ch. 8).}%
This covert operator is semantically vacuous, but it can move and leave a trace
of type $\type{s,t}$. The full syntactic derivation for a constituent question
then proceeds as in \Next.
%
\note{The movement of who is evidently overt (pre-SS) movement (in this
  example). For the movement of the operator, which is not pronounced, we can't
  tell whether it happens before SS or between SS and LF.}%
%
\ex Who did Sakina see?\\
\null [$_{C}$ ? OP] Sakina see who\\
who 2 [ [? OP] Sakina see $t_{2_{e}}$ ] \hfill (wh-movement)\\
OP 1 [ who 2 [ [? $t_{1_{st}}$ ] Sakina see $t_{2_{e}}$ ] ] \hfill (operator
movement)
\xe
%
\kwn The resulting LF looks thus like this in tree format:

\ex
\begin{forest}
baseline,
sn edges,
for tree={s sep=5mm, inner sep=0, l=0}
[{}, my pretty nice empty nodes
[OP$_1$] [[who$_2$] [[[\textbf{?}] [$t_1$]] [[Sakina] [[see] [$t_2$]]]]]
]
\end{forest}
\xe

We will usually drop the type-labels on the traces, but keep in mind that the
operator leaves a trace of type \type{s,t} (i.e., a variable over propositions,
not individuals). Accordingly the topmost application of Predicate Abstraction
will yield a function from propositions to truth-values (type \type{st,t}, the
characteristic function of a set of propositions). %
\note{We leave out the vacuous operator OP and just interpret the predicate
  abstract it creates. $\emptyset$ in the superscript stands for the empty
  variable assignment, see H\&K ch.5}%
Let's compute the meaning of this LF:

\noindent
\ex Computation for LF \Last:\label{ex:wh-comp}\\
\begin{minipage}{1.5\textwidth}
$\sv{1.\ \text{who}\ 2.\ [?\ t_{1}]\ \text{Sakina see}\ t_{2}}^{w,\emptyset}$\\
{\small = (by Predicate Abstraction)}\\
$\lambda p.\ \sv{\text{who}\ 2.\ [?\ t_{1}]\ \text{Sakina
    see}\ t_{2}}^{w,[1\rightarrow p]}$\\
{\small = (by entry for \emph{who} and lambda reduction)}\\
$\lambda p.\ \exists x\ [\ x\ \text{is human in}\ w\ \&\ \sv{2.\ [?\ t_{1}]\ \text{Sakina
    see}\ t_{2}}^{w,[1\rightarrow p]}(x) = 1\ ]$\\
{\small = (by Predicate Abstraction and lambda reduction)}\\
$\lambda p.\ \exists x\ [\ x\ \text{is human in}\ w\ \&\ \sv{[?\ t_{1}]\ \text{Sakina
    see}\ t_{2}}^{w,[1\rightarrow p,2\rightarrow x]} = 1\ ]$\\
{\small = (by IFA)}\\
$\lambda p.\ \exists x\ [\ x\ \text{is human
  in}\ w\ \&\ \sv{?\ t_{1}}^{w,[1\rightarrow p,2\rightarrow x]}(\lambda w'.\ \sv{\text{Sakina
    see}\ t_{2}}^{w',[1\rightarrow p,2\rightarrow x]}) = 1\ ]$
\end{minipage}
\clearpage\begin{minipage}{1.5\textwidth}
{\small = (by FA and dropping irrelevant superscripts)}\\
$\lambda p.\ \exists x\ [\ x\ \text{is human
  in}\ w\ \&\ \svt{?}(\sv{t_{1}}^{[1\rightarrow p,2\rightarrow x]})(\lambda w'.\ \sv{\text{Sakina
    see}\ t_{2}}^{w',[1\rightarrow p,2\rightarrow x]}) = 1\ ]$\\
{\small = (by Traces rule)}\\
$\lambda p.\ \exists x\ [\ x\ \text{is human
  in}\ w\ \&\ \svt{?}(p)(\lambda w'.\ \sv{\text{Sakina
    see}\ t_{2}}^{w',[1\rightarrow p,2\rightarrow x]}) = 1\ ]$\\
{\small = (by entry for \emph{?})}\\
$\lambda p.\ \exists x\ [\ x\ \text{is human
  in}\ w\ \&\ p = \lambda w'.\ \sv{\text{Sakina
    see}\ t_{2}}^{w',[1\rightarrow p,2\rightarrow x]}\ ]$\\
{\small = (by FA, entries for \emph{Sakina}, \emph{see}, Traces Rule)}\\
$\lambda p.\ \exists x\ [\ x\ \text{is human
  in}\ w\ \&\ p = \lambda w'.\ \text{Sakina
  sees}\ x\ \text{in}\ w'\ ]$\\
\end{minipage}
\xe
%
\note{This is the \textsc{ansposs} in \cite{hagstrom-2003-GLOT-questions}'s
  terminology.}%
This characterizes a set of propositions that contains one proposition per
human-in-$w$: the proposition that that human was seen by Sakina.

How about polar and alternative questions? Can we just posit the same operators
in the C-head? Let's try.

\note{Here we show an LF and computation that includes the vacuous operator. As
  an exercise, convince yourself that in the case of polar questions, a simpler
  structure without OP is likewise interpretable, with equivalent results.}%
\ex Did Sakina see Emily?\\
\null[$_{C}$ ? OP] Sakina see Emily\\
OP 1 [ ? $t_{1}$] Sakina see Emily \xe

\ex Computation for LF of the polar question \Last:\\
$\sv{1.\ [?\ t_{1}]\ \text{Sakina see Emily}}^{{w,\emptyset}}$\\
{\small = (by Predicate Abstraction)}\\
$\lambda p_{st}. \sv{[?\ t_{1}]\ \text{Sakina see
    Emily}}^{w,[1\rightarrow p]}$\\
{\small = (by IFA)}\\
$\lambda p_{st}. \sv{?\ t_{1}}^{w,[1\rightarrow p]}(\lambda w'. \svt{Sakina see Emily}^{w',[1\rightarrow p]})$\\
{\small = (by FA)}\\
$\lambda p_{st}. \sv{?}^{w,[1\rightarrow p]}(\sv{t_{1}}^{w,[1\rightarrow p]})(\lambda w'. \svt{Sakina see Emily}^{w',[1\rightarrow p]})$\\
{\small = (by Traces Rule and dropping irrelevant assignment superscripts)}\\
$\lambda p_{st}. \sv{?}(p)(\lambda w'. \svt{Sakina see Emily}^{w'})$\\
{\small = (by entry for ? and lambda reduction)}\\
$\lambda p_{st}.\ [p = \lambda w'. \svt{Sakina see Emily}^{w'}]$\\
{\small = (by FA and entries for \emph{Sakina}, \emph{Emily}, \emph{see})}\\
$\lambda p_{st}.\ [p = \lambda w'.\ \text{Sakina see Emily in}\ w']$
\xe
%
This is the characteristic function of a singleton set containing one
proposition. Is this a good result? If we have in mind that our denotations for
interrogative clauses should directly correspond to an intuitive notion of
``possible answer'', then this is problematic. There is certainly more than one
possible answer to a polar question! But we think of the relation
between the semantics and the pragmatics of interrogative clauses in a less
simple-minded way. A singleton set of a proposition is used to induce a two-way
partition of the context set: one cell containing worlds where the proposition
is true and another where it isn't. Any relevant response needs to eliminate one
of the cells.

\clearpage
To complete the current section, we consider an alternative question:
%
\note{The slashes are intended as a crude representation of the distinctive
  intonational contour that characterizes the alternative-question reading.}%
\ex Did Sakina see Emily /or did he see Julie\textbackslash ?\\
DS: [[$_{C}$ ? OP] Sakina see Emily] or [[$_{C}$ ? OP] Sakina see Julie]\\
LF: OP 1 [ [ [? $t_{1}$] Sakina see Emily] or [ [? $t_{1}$] Sakina see Julie] ]
\xe
%
In the alternative question in \Last, operator movement must be ``across the
board'' (ATB), with the result that a single binder binds two coindexed traces.
This is the only way to obtain an interpretable structure, given the semantic
type of \emph{or}, which is \type{t,\type{t,t}}.

\ex Computation for LF of the alternative question in \Last:\\
$\sv{1.\ [?\ t_{1}]\ \text{Sakina see Emily or}\ [?\ t_{1}]\ \text{Sakina see
    Julie}}^{w,\emptyset}$\\
{\small = (by Predicate Abstraction)}\\
$\lambda p_{st}.\ \sv{[?\ t_{1}]\ \text{Sakina see Emily
    or}\ [?\ t_{1}]\ \text{Sakina see
    Julie}}^{w,[1\rightarrow p]}$\\
{\small = (by FA twice and entry for or)}\\
$\lambda p_{st}.\ \sv{[?\ t_{1}]\ \text{Sakina see Emily}}^{w,[1\rightarrow p]}\
\vee\ \sv{[?\ t_{1}]\ \text{Sakina see Julie}}^{w,[1\rightarrow p]}$\\
{\small = (by IFA, FA, entry for ?, etc \dash see previous computations)}\\
$\lambda p_{st}.\ [ p = \lambda w'.\svt{Sakina see Emily}^{w',[1\rightarrow p]}\
\vee\ p = \lambda w'.\svt{Sakina see Julie}^{w',[1\rightarrow p]}]$\\
{\small = (by FA and entries for \emph{see} etc.)}\\
$\lambda p_{st}.\ [ p = \lambda w'.\ \text{Sakina see Emily
  in}\ w'\ \vee\ p = \lambda w'.\ \text{Sakina
  see Julie in}\ w']$\\
\xe
%
This is the characteristic function of a set containing two propositions \dash
the same set, in fact, that is denoted by the constituent question \emph{who did
  Sakina see} if the set of humans happens to be just \{Emily, Julie\}.

\subsection{Back to pragmatics: Mapping interrogative denotations to partitions}
\label{sec:back-to-pragmatics}

We can now make good on the promise to provide a bridge between the
compositional semantics of interrogatives and their use in updating a context.
We gave a draft version earlier in
Definition~\ref{def:update-by-question-draft}, repeated here:

\updatebyquestiondraft*%

To finalize the definition, we need to use our semantics to supply a set of
propositions to serve as the seed for the partition. This is straightforward in
the case of polar and alternative questions, each of which has as its semantic
value a set of propositions that is evaluation-world independent. But in the
case of wh-questions, an issue arises: their semantic value depends
non-trivially on the evaluation world.

\ex $\svt{Who did Sakina see?}^{w}$\\
$= \lambda p.\ \exists x\co x\ \text{is human
  in}\ w.\ p = \lambda w'.\ \text{Sakina saw}\ x\ \text{in}\ w'$
\xe
%
What set of propositions the interrogative denotes in a world $w$ depends on who
the humans in that world are. In case this seems esoteric, consider a more
substantively restricted wh-phrase:

\ex $\svt{Which students called?}^{w}$\\
$= \lambda p.\ \exists x\co x\ \text{is a student
  in}\ w.\ p = \lambda w'.\ \text{Sakina saw}\ x\ \text{in}\ w'$
\xe
%
\phantomsection\label{par:context-uncertainty}%
The semantic value of \Last depends on who the students in the evaluation world
are. It is easy to imagine circumstances where it is not common ground who the
students are. So, how should we update the context with a question like \Last?

\note{Option 1 would say something like: To ask a question by uttering a
  sentence $\phi$ in world $w$ is to propose that the current partitioned
  context set $\mathcal{C}$ be replaced by the new partitioned context set
  $\textsc{part}(\sv{\phi}^{w},\bigcup \mathcal{C})$.}%
\emph{Option 1: The world of utterance}. We could say that the context gets
partitioned based on the set of propositions denoted by the interrogative
relative to the world in which the question is asked. However, this would miss
the fact that the participants in the conversation do not know which world they
are in. The context set is precisely meant to model that they have some common
ground on what world they're in but no more: there are always multiple
candidates for what the actual world is. Furthermore, it is presumably
inescapable that participants make some false presuppositions, in which case the
actual world in which the utterance occurs isn't even part of the context set
(even though everyone is acting as if the common ground contains the actual
world).

So, it seems clear that we need to interpret the interrogative with respect to
the worlds in the context set and use the resulting set of propositions to
partition the context set. But now we need to face the possibility that the
participants in the conversation where \Last is uttered are uncertain about who
the (relevant) students are. To set up a minimal test scenario, assume that
everyone knows that $a$ is a student but there's uncertainty about whether $b$
is. Now, assume that \Last is uttered against that background. There are worlds
in the context set where $a$ and $b$ are the students, and other worlds where
only $a$ is a student. Our semantics will deliver two different sets of
propositions as the semantic value of the interrogative for the two different
kinds of worlds in the context set: for the $a,b$-worlds, we get the set
containing the proposition that $a$ called and the proposition that $b$ called;
while for the $a$-worlds, we get the set that contains only the proposition that
$a$ called. How should the interrogative partition the context set?

\clearpage
\emph{Option 2: Collecting all the propositions}. To obtain the set of
propositions that is used to partition the context set, we could collect all the
propositions that for \emph{some} world in the context set are in the semantic
value of the interrogative:

\note{Equivalently, form the big union of all the sets of propositions that are
  the semantic value of the interrogative in worlds of the context sets:
  \[\bigcup_{w\in c} \sv{\phi}^{w}\]}
\ex The set of propositions that is used to partition a context set $c$ upon
the utterance of an interrogative $\phi$ is
\[\{p\co \exists w\in c.\ p\in\sv{\phi}^w\}\]
\xe

For our minimal test scenario, that means that the context set will be
partitioned by the the set containing the proposition that $a$ called and the
proposition that $b$ called. By our definition of relevance, this means that the
answer ``$b$ called'' will be relevant. But notice that when the context set is
updated by the assertion that $b$ called, there will still be worlds left in the
context set where $b$ is not a student (while it is now accepted that $b$ called).
This is incorrect: answering that $b$ called when what was asked is \emph{Which
  students called?} surely commits one to $b$ being a student.

\emph{Option 3: Presupposition of consensus}. \cite{stalnaker-1978-assertion} in
his analysis of the speech act of assertion formulated three principles that he
views as ``essential conditions of rational communication'':

\begin{enumerate}[1.]
  \item A proposition asserted is always true in some but not all of the
        possible worlds in the context set.
  \item Any assertive utterance should express a proposition, relative to each
        possible world in the context set, and that proposition should have a
        truth-value in each possible world in the context set.
  \item The same proposition is expressed relative to each possible world in the
        context set.
\end{enumerate}
%
What's relevant for us here is the third principle, which ensures that there is
common ground on which proposition the speaker is proposing to add to the common
ground. We suggest that essentially the same principle applies to the speech act
of asking a question: the same set of propositions needs to be expressed by the
interrogative relative to each possible world in the context set. This then
amounts to the assumption that for each wh-phrase that is used, the
interlocutors agree on a specific set of individuals as its intended range. This
means that in our test scenario, the question \emph{Which students called?} is a
pragmatic error: asking the question presupposes that it is common ground who
the students are.

Hence, the final version of our definition is as follows:

\begin{definition}[Update by question (final)]%
  \label{def:update-by-question-final}%
  Asking a question by uttering a sentence $\phi$ is only felicitous in a
  partitioned context set $\mathcal{C}$ if
  $\forall w,w'\in \bigcup C\co \sv{\phi}^{w}=\sv{\phi}^{w'}$.

  If felicitous, asking a question by uttering a sentence $\phi$ is to propose
  that the current partitioned context set $\mathcal{C}$ be replaced by the new
  partitioned context set $\textsc{part}(\sv{\phi}^{w},\bigcup \mathcal{C})$,
  where $w$ is an arbitrary world in $\bigcup \mathcal{C}$.
\end{definition}



\subsection{Syntax: the role of the [WH] feature}
\label{sec:wh-syntax}

In the theory as it stands so far, the semantic identification of wh-words with
indefinites (e.g. $\svt{who}^{w}=\svt{somebody}^{w}$) leads to overgeneration of readings.
Consider the minimal pair in \Next.

\pex\label{ex:wh-pol}
\a Who did Sakina see?
\a Did Sakina see somebody?
\xe

We have shown in Section~\ref{sec:karttunen-1977} how the Karttunen theory
derives the intended denotation for \Last[a] (= \refx{ex:wh-comp}). But from the
Deep Structure we posited in \refx{ex:wh-comp}, we could have derived more
  than one interpretable structure, namely not just \Next[a] (which we
  considered in \refx{ex:wh-comp} above) but also \Next[b].

\pex\label{ex:wh-high-low}
\a LF-high: OP 1 [ who 2 [ [? $t_{1}$] Sakina see $t_{2}$ ] ]
\a LF-low: OP 1 [[? $t_{1}$] who 2 [ Sakina see $t_{7}$ ] ]
\xe
%
Similarly we can derive two interpretable LFs for \LLast[b].

\pex\label{ex:somebody-high-low}%
DS: [$_{C}$ ? OP] Sakina see somebody
\a LF-high: OP 1 [ somebody 2 [ [? $t_{1}$] Sakina see $t_{2}$ ] ]
\a LF-low: OP 1 [[? $t_{1}$] somebody 2 [ Sakina see $t_{2}$ ] ]
\xe

In all of these potential LFs, the IP-sister of C, as well as the mother-node of
C, are semantically of type $t$. Therefore, both should be suitable adjunction
sites for the generalized quantifiers \emph{who} and \emph{somebody} as far as
type-compatibility goes. (The mnemonic labels `high' and `low' indicate the
difference in the quantifier's scope.)

But what do these LFs mean? We have already computed \refx[a]{ex:wh-high-low}
and seen that it expresses an appropriate meaning for \refx[a]{ex:wh-pol}.
Specifically, if we combine our semantics with our pragmatics, and if the domain
of people that who ranges over consists of just Emily and Julie,
\refx[a]{ex:wh-high-low} determines a 4-cell partition. This matches the
intuition that the possible fully exhaustive answers to \refx[a]{ex:wh-pol} are
that Sakina saw only Emily, that he saw only Julie, that he saw both Emily and Julie,
and that he saw neither. But such a 4-cell partition is clearly not what an
utterance of \refx[b]{ex:wh-pol} sets up. \refx[b]{ex:wh-pol} is a polar
question and elicits answers such as Yes or No, expressing the propositions that
Sakina saw somebody and that he saw nobody. So here we want our semantics and
pragmatics to determine a 2-cell partition. The LF-low in
\refx[b]{ex:somebody-high-low} does deliver precisely this. (Compute this as an
exercise.)

So LF-high in \refx[a]{ex:wh-high-low} captures the attested meaning of the
wh-question in \refx[a]{ex:wh-pol}, and LF-low in \refx[b]{ex:somebody-high-low}
captures the attested meaning of the polar question in \refx[b]{ex:wh-pol}. In
each case, this is the only attested reading for the English sentence: the
wh-question cannot be read as a polar question, and the polar question cannot be
read as a wh-question. Our theory so far fails to derive this un-ambiguity. The
LF-low in \refx[b]{ex:wh-high-low} is semantically equivalent to the LF-low in
\refx[b]{ex:somebody-high-low} and thus represents an unattested polar reading
for the wh-question. Similarly, the LF-high in \refx[a]{ex:somebody-high-low} is
equivalent to the LF-high in \refx[a]{ex:wh-high-low} and thus represents an
unattested wh-question reading for the polar question with \emph{somebody}. We
need to amend our theory so that only one of the LFs is generated for each of
the sentences.

A standard solution \dash effectively Karttunen's \dash is to invoke a syntactic
constraint that regulates scopal relations between existential DPs (\emph{who},
\emph{somebody}) and the interrogative complementizer. As our example teaches
us, \emph{who} apparently can only scope above \emph{?} (more specifically,
between \emph{?} and its associated \emph{OP}), whereas \emph{somebody} is
barred from scoping there and must instead scope below \emph{?}. The following
stipulation enforces this generalization.

\ex Wh-Licensing Principle:\\
At LF, a phrase $\alpha$ occupies a specifier position of \emph{?} if and only
if $\alpha$ has the feature [WH].
\xe
%
\Last relies on appropriate assumptions about which phrases have the feature
[WH]. For the time being, assume that certain words such as \emph{who},
\emph{what}, \emph{how} are marked as [WH] in the lexicon. These then will be
the only phrases that can be located right above \emph{?} in well-formed LFs,
and moreover, they cannot be located anywhere else. LF-high for the wh-question
in \refx[a]{ex:wh-high-low} complies with \Last, as does LF-low for the polar
question in \refx[b]{ex:somebody-high-low}. LF-low in \refx[b]{ex:wh-high-low}
is ruled out because it has a phrase marked [WH] in a location other than
spec-of-\emph{?}, and LF-high in \refx[a]{ex:somebody-high-low} is prohibited
because it has a phrase in spec-of-\emph{?} which lacks [WH].

\begin{exercise}
  Consider the multiple wh-question \Next.

  \ex Who likes who? \xe
%
  Propose an LF, say how it is derived in the syntax, compute its semantic
  interpretation, and then compute the partition it imposes. For simplicity,
  assume in this last part that the domain of people that who ranges over is
  just \{s, e\}, and that the common ground before the question is totally
  uninformative, i.e, it is a partitioned context set whose union is W (the set
  of all worlds whatsoever). How many cells does the partition have and what are
  its cells? Regarding syntax, attend in particular to the satisfaction of the
  Wh-Licensing Principle.
\end{exercise}

\begin{exercise}
  In written language, a question containing \emph{or}, such as \Next, can be
  ambiguous.

  \ex Did you talk to David or Norvin?\xe
%
  The questioner may want to know which of the two professors you talked to
  (``alternative-question reading'') or just whether you talked to at least one
  of them (``polar-question reading''). Perhaps the alternative reading is more
  salient out of the blue, but the polar reading can be facilitated by an
  appropriate context. (E.g. imagine that your squib is on
  multiple-wh-constructions, and you have previously been told that you ought to
  consult a faculty member who has published on this topic, namely David or
  Norvin.)

  Your task in the exercise is to analyze the two readings of \Last, by
  proposing an appropriate LF for each reading and discussing its syntactic
  derivation as well as its semantic and pragmatic interpretation. You should
  assume that English has only one unambiguous word \emph{or}, and its semantic
  type is $\type{t,\type{t,t}}$. This means that, for both readings, you must
  posit some amount of elided material in the right disjunct, since \emph{or}
  can only coordinate constituents of type $t$.
\end{exercise}

\section{Embedded questions and question-embedding verbs}
\label{sec:embedded-questions}

A first superficial survey of English verbs that take complement clauses turns
up three groups. %
\note{\cite{white-2021-believing-whether} argues that things are not that
  simple.}%
One group, exemplified by the verb \emph{believe}, consists of verbs that take
\emph{that}-clauses but are ungrammatical with an embedded interrogative clause.

\pex
\a Sakina believes that Becky called.
\a *Sakina believes who called.
\xe

Another rather large second group consists of verbs that can take either. This
includes \emph{know}, \emph{remember}, and \emph{tell}.

\pex
\a Sakina knows/remembers/told Emily that Becky called.
\a Sakina knows/remembers/told Emily who called.\label{ex:know-Q}
\xe

Third, there are verbs which take interrogative complements but are
ungrammatical with \emph{that}-clauses.

\pex
\a *Sakina asked/is wondering that Becky called.
\a Sakina asked/is wondering who called.
\xe

We will focus at first on the middle group, which
\cite{lahiri-2002-questions-book} dubbed the class of ``responsive'' verbs.

\subsection{Responsive verbs, a type-mismatch, and Dayal's strategy}
\label{sec:responsive-verbs}

Let's look at the responsive verb \emph{know}. We have at least a preliminary
analysis for verbs like \emph{know} in sentences like \emph{John knows that Ann
  called}. Such verbs take a proposition and a person as their arguments, and
their meaning encodes universal quantifcation over a certain set of possible
worlds. Since \emph{know} is a factive verb, it also triggers the presupposition
that its complement is true. For simplicity, we assume here that apart from the
factive presupposition, the meaning of \emph{know} is the same as the meaning of
\emph{believe}, and so we can write the lexical entry in \Next.
%
\ex%
\note{Here, \textsc{dox} maps an individual $x$ and a world $w$ into the set of
  worlds compatible with what $x$ believes in $w$.}%
$\svt{know}^{w} = \lambda p_{st}\co p(w)=1.\ \lambda x_{e}.\ \forall w' [w'\in \dox(x,w) \rightarrow p(w')=1]$
\xe
%
This entry was designed to work for \emph{know} with a \emph{that}-clause. What
would happen if we tried to interpret an LF with an interrogative clause as the
sister of \emph{know}? Evidently, we would run into a type-mismatch. On our
current analysis of interrogative clauses, their extensions are of type
\type{st,t} and their intensions of type \type{s,\type{st,t}}. Neither type can
compose with the type of know in \Last by means of any of our semantic rules.

What shall we do? Older literature on question-embedding made verbs like
\emph{know} lexically ambiguous, with distinct (though not unrelated) lexical
entries for declarative-taking \emph{know} and interrogative-taking \emph{know}.
More recent work has pursued the strategy of positing a single unambiguous verb
and readjusting the semantic type of the interrogative complement.
\cite{groenendijk-stokhof-1982-wh-complements} did this first, and another
influential version of this approach originates with
\cite{dayal-1996-locality-wh}. Dayal proposed that the combination of the verb
with the Karttunen-denotation of its complement is mediated by an ``answer
operator'', which maps sets of propositions to propositions. We will follow
Dayal's general strategy in these notes. We will entertain a couple of possible
meanings for the answer operator and talk about the empirical considerations
that bear on the question which meaning is correct.

The rough intuition to be implemented is that ``to know who called'' means
something like ``to know the answer to the question `who called?'''.
(Paraphrases of this form work for all the verbs in this group, hence Lahiri's
term ``responsive verbs''.) The answer to a question is a proposition, hence an
object of a suitable semantic type to feed to the meaning of \emph{know} in
\Last. If the LFs of sentences like \refx{ex:know-Q} contain an operator that
maps a question-denotation to the proposition that's the answer to that
question, we have a solution to the type-mismatch problem.

Since our syntax for interrogative clauses already happens to posit a silent
operator at the top edge of the clause (albeit one that we have so far treated
as semantically vacuous) we need not actually make the structure more complex.
Instead we can assume (following a suggestion by Danny Fox) that our new answer
operator appears instead of the previous vacuous one. This means that we
base-generate it inside C as the sister of \emph{?} and move it up for
interpretability, leaving a type-\type{s,t} trace as before. Our LF-structure
for a sentence with know and an interrogative complement then looks as in
\Next[b].

\pex\label{ex:knows-who}
\a Sakina knows who called.
\a Sakina [$_{VP}$ knows [$_{CP}$ \ans 1 [ who 2 [[$_{C}$ ? $t_{1}$]
$t_{2}$ called]]]]
\xe

We will now focus on the task of proposing a meaning for \ans which not only
fixes the type-mismatch, but also yields reasonable truth conditions for the
\emph{know}-sentence.

\subsection{An \ans operator inspired by Karttunen}
\label{sec:ans-karttunen}

Our initial proposal for the semantics of \ans effectively follows
\cite{karttunen-1977-questions} in terms of the truth conditions it predicts for
the \emph{know}-sentence as a whole (although the compositional implementation
is different). The \ans operator defined in \Next maps the set of propositions
denoted by the interrogative to the single proposition which is the conjunction
of all its true elements.

\ex
$\sv{\ans}^{w} = \lambda \mathcal{Q}_{\type{st,t}}.\ \forall p [\mathcal{Q}(p)=1\ \&\ p(w)=1 \rightarrow p(w')=1]$
\xe

\ex~[exno=\lastx']\note{equivalent formulation (modulo sets vs. characteristic
functions)}%
$\sv{\ans}^{w}= \lambda \mathcal{Q}_{\type{st,t}}.\ \bigcap\{p\in \mathcal{Q}\co w\in p\}$\\
\hfill (the intersection of all members of $\mathcal{Q}$ that are true in $w$)
\xe

Let's put this entry to work in a computation for the example sentence.

\pex\label{ex:knows-who-comp}%
\a computation of presupposition:\\
Let $w$ be a world. Then \\
$\svt{Sakina knows who called}^{w}$ is defined\\
\note{Observe that the mother node of \emph{know} is interpreted by plain
  Functional Application (not by Intensional Functional Application, which would
  have applied if \emph{know} were taking a declarative complement). This is
  because the \emph{extension} of the constituent headed by \ans is of type
  \type{s,t}.}%
{\small iff (by FA twice)}\\
$\svt{know}^{w}(\svt{\ans 1 who 2 ?-$t_1$ $t_2$ called}^{w})(s)$ is defined\\
{\small iff (by entry for \emph{know})}\\
$\svt{\ans 1 who 2 ?-$t_1$ $t_2$ called}^{w}(w)=1$\\
{\small iff (by FA)}\\
$\svt{\ans}^{w}\big(\svt{1 who 2 ?-$t_1$ $t_2$ called}^{w}\bigr)(w)=1$\\
{\small iff by entry for \ans}\\
$\forall p [ \svt{1 who 2 ?-$t_1$ $t_2$ called}^{w}(p)=1\ \& p(w)=1 \rightarrow p(w)=1 ]$\\
This is a tautology, so we know that $\svt{Sakina knows who called}^{w}$ is
defined for all $w$.
\a computation of truth condition:\\
Let $w$ be a world. Then\\
$\svt{Sakina knows who called}^{w} = 1$\\
{\small iff (by FA twice)}\\
$\svt{know}^{w}(\svt{\ans 1 who 2 ?-$t_1$ $t_2$ called}^{w})(s) = 1$\\
{\small iff (by entry for \emph{know} and truth of presupposition)}\\
$\forall w' [w'\in \dox(s,w) \rightarrow \svt{\ans 1 who 2 ?-$t_1$ $t_2$ called}^{w}(w')=1]$
\a We interrupt for an embedded computation:\\
$\svt{\ans 1 who 2 ?-$t_1$ $t_2$ called}^{w}(w')=1$\\
{small iff (by entry for \ans)}\\
$\forall p [ \svt{1 who 2 ?-$t_1$ $t_2$ called}^{w}(p)=1\ \&\ p(w)=1 \rightarrow p(w')=1 ]$\\
{\small iff (by earlier computations)}\\
$\forall p [\exists x [x\ \text{is a human
  in}\ w\ \&\ p=\lambda w''.\ x\ \text{called
  in}\ w'']\ \&\ p(w)=1 \rightarrow p(w')=1 ]$\\
{\small iff (by logic of quantifiers)}\\
$\forall p\forall x [x\ \text{is a human
  in}\ w\ \&\ p=\lambda w''.\ x\ \text{called
  in}\ w''\ \&\ p(w)=1 \rightarrow p(w')=1 ]$\\
{\small iff (by logic of identity)}\\
$\forall x [ x\ \text{is a human in}\ w\ \&\ [\lambda w''.\ x\ \text{called
  in}\ w''](w)=1 \rightarrow [\lambda w''.\ x\ \text{called in}\ w''](w')=1]$\\
{\small iff (by $\lambda$-reduction)}\\
$\forall x [ x\ \text{is a human in}\ w\ \&\ x\ \text{called
  in}\ w \rightarrow x\ \text{called in}\ w']$ \a resuming main computation:
\dots \\
{\small iff (by plugging in result of embedded computation)}\\
$\forall w' [w'\in \dox(s,w) \rightarrow \forall x [ x\ \text{is a human
  in}\ w\ \&\ x\ \text{called
  in}\ w \rightarrow x\ \text{called in}\ w']]$\\
{\small iff (by logic of quantifiers)}\\
$\forall x [x\ \text{is a human in}\ w\ \&\ x\ \text{called
  in}\ w \rightarrow \forall w' [w'\in \dox{s,w} \rightarrow x\ \text{called
  in}\ w']]$ \xe
%
In other words, the sentence \emph{Sakina knows who called} is true in $w$ if
and only if, for every person who in fact called in $w$, Sakina believes (in
$w$) that this person called.

\subsection{Problems, and an \ans operator inspired by Groenendijk \& Stokhof}
\label{sec:ANS-GS}

The essence of \ans in our present analysis is that for any evaluation world
$w$, when applied to the semantic value $\mathcal{Q}$ of an interrogative (which
is a set of proposition), it gives the proposition that is the conjunction of
all the propositions in $\mathcal{Q}$ that are true in $w$. This is well known
to make some troublesome predictions. Let's start with the most glaring case,
then move to subtler ones, before we diagnose the general problem and proceed to
solve it.

The example we analyzed above involved an embedded constituent question. What if
we embedded a polar question instead?

\pex
\a Sakina knows whether Emily called.
\a LF: Sakina knows [$_{CP}$ \ans 1[ [$_{C}$? $t_{1}$] Emily called ] ]
\xe
%
Recall that our denotation for a polar question is a singleton set. The sister
of \ans in \Last[b] denotes the set whose only member is the proposition that
Emily called. If we complete the calculation, we get the following truth
condition.

\ex presupposition of \Last[b]: tautological\\
$\svt{\Last[b]}^{w} = 1$ iff\\
Emily called in
$w \rightarrow \forall w' [w'\in\dox(s,w) \rightarrow \text{Emily called
  in}\ w']$
\xe
%
This says that the sentence \LLast[a] is true in $w$ if either one of the
following two conditions is met: either (i) Emily did not call in $w$, or else
(ii) she did call in w and Sakina believes in w that she did. This is not
satisfactory. What it gets right is that, if Emily called but Sakina is unaware
of this, then the sentence is false. But it also predicts that, if Emily didn't
call, then the sentence is true no matter what Sakina believes \dash even if she
wrongly believes that she did call.

A gut reaction to this problem is that the culprit is our semantics for polar
questions, not our \ans operator. This is what Karttunen would have said.
Indeed, he gave a different semantics for polar questions and did not have this
problem. %
\note{How might we do that? Perhaps by giving a meaning to \emph{whether},
  letting it denote a function that maps a singleton set $\{p\}$ to the set
  $\{p, \neg p\}$.}%
In our variant of his theory, if we minimally changed the semantics of polar
questions so that the sister of \ans were to denote the 2-membered set \{that
Emily called, that Emily didn't call\}, the truth conditions would come out
correct without any revision to our entry for \ans. (Exercise: Convince yourself
of this.) This looks like a good way out \dash at least at first. But when we
look at further problem cases, we will come to see it is a move that is neither
sufficient nor necessary.

\note{The problem we're discussing here was noticed by
  \cite{karttunen-1977-questions} in a footnote, and he fixed it by complicating
  his lexical entry for interrogative-taking \emph{know}. He ended up stating
  the truth condition in the form of a disjunction, with a special clause for
  the case where the question-denotation only contains false propositions.
  \cite{heim-1994-interrogatives-IATL} showed how to generalize Karttunen's
  special clause to a general solution for all the problem cases we consider in
  this section. The solution we will present in these lecture notes is not quite
  the same as Heim's. See papers by
  \cite{rullmann-beck-1998-presupposition-which},
  \cite{beck-rullmann-1999-flexible}, \cite{sharvit-2002-embedded-questions},
  and \cite{sharvit-guerzoni-2003-reconstruction} for discussion and
  comparison.}%
Let's return to the case of the embedded constituent question in
\refx{ex:knows-who} and scrutinize the truth conditions we derived in
\refx{ex:knows-who-comp} a bit more carefully. Suppose that w is a world in
which nobody called. Then the universal quantification we computed in
\refx[b]{ex:knows-who-comp} is trivially true: Whatever Sakina's beliefs in $w$
may be, the material conditional `[$x$ called in $w$ $\rightarrow$ Sakina
believes in $w$ that $x$ called]' is true for every $x$ (since the antecedent is
always false). So the sentence \refx{ex:knows-who} is predicted to be true, for
example, in a world where nobody called but Sakina falsely believes that Emily
and Julie called. This does not conform to our intuitions.

Finally, as \cite{groenendijk-stokhof-1982-wh-complements} forcefully pointed
out, even if we only consider worlds in which some people did in fact call, the
truth conditions imposed by our current (and Karttunen's) semantics are too lax.
Suppose that only Emily called, but Sakina thinks that Emily, Julie, and
Delphine all called. Would we say that Sakina knows who called? We'd be
reluctant to. But our semantics deems the sentence \emph{Sakina knows who
  called} to be true in this scenario. After all, Sakina does believe of every
person who in fact called (namely, of Emily), that that person called.

This is all that our predicted truth conditions require. If our analysis were
right, it simply shouldn't matter how many false beliefs Sakina has about people
calling who did not in fact call.

Groenendijk \& Stokhof argued that the correct semantics for \emph{Sakina knows
  who called} is what they dubbed ``strongly exhaustive'' \dash i.e., the
\emph{know}-sentence is true only when Sakina is fully informed about who called
and who didn't. She believes that they called of all the people who did in fact
call, and she believes that they didn't call of all the ones who didn't. Can we
revise our entry for the answer operator so that it delivers this more stringent
truth condition? Yes, here is how.

\ex strongly exhaustive answer operator:\\
$\sv{\ans}^{w} = \lambda \mathcal{Q}_{\type{st,t}}.\lambda w'.\ \forall p [p\in\mathcal{Q} \rightarrow p(w)=p(w')]$
\xe

\ex~[exno=\lastx']%
\note{equivalent formulation, using the definition of
  ``cell-mate'' from \refx[a]{ex:cellmate-partition}. Here, capital $W$ in the
  subscript to the $\sim$-relation is the set of all possible worlds.}%
$\sv{\ans}^{w} = \lambda \mathcal{Q}_{\type{st,t}}.\ \{w'\co w'\sim_{\mathcal{Q},W}w\}$
\xe

What does this semantics for \ans do? It takes the set of propositions
$\mathcal{Q}$ denoted by the underlying interrogative and maps the evaluation
world to the worlds that agree with it on the truth-value of all propositions in
$\mathcal{Q}$. In other words, it maps the evaluation world to its cell-mates
relative to $\mathcal{Q}$.

This new semantics solves all the problems that we saw in this section. For
embedded polar questions, it delivers the prediction that if Emily didn't call,
then \emph{Sakina knows whether Emily called} is only true if Sakina knows that
Emily didn't call. For embedded constituent questions, it predicts that if $x$
didn't call, then \emph{Sakina knows who called} is not true unless Sakina knows
that $x$ didn't call. As a special case of this latter prediction, we derive
that if nobody called, then \emph{Sakina knows who called} is only true if
Sakina knows about each person that they didn't call.

\begin{exercise}
  Verify these claims, by doing the computations.
\end{exercise}

\subsection{\ans in matrix questions?}
\label{sec:matrix-ans}

Our latest, strongly exhaustive answer operator bears an obvious logical
relation to the pragmatic rule ``update by question''
(Definition~\ref{def:update-by-question-final}) that we posited earlier to link
the semantic values of matrix interrogative clauses to the speech acts that they
serve to perform. In both cases, we use the set of propositions $\mathcal{Q}$ to
lump together any evaluation world with those worlds that agree with it on the
truth-values of all propositions in $\mathcal{Q}$. In the case of interrogatives
embedded under responsive predicates like \emph{know}, this set of worlds that
are cell-mates with the evaluation world is fed to the attitude predicate as its
prejacent proposition. In the case of matrix questions, we use the set of
disjoint but exhaustive cells to partition the context set.

Given this similarity, we may want to consider positing \ans in matrix questions
as well, instead of the vacuous covert operator $OP$. This would mean rewriting
the ``update by question'' rule. The idea would be that the pragmatic rule can
use the intension of the uttered sentence to construct the cells of the new
partition. The recipe, informally, is to apply this intension to each world in
the current context set and intersect each result with the current context set.

\begin{definition}[Update by question,
  draft revision]\label{def:updatebyquestiondraftrevision}%
  \note{Note that the relevant sentences $\phi$ are of the form ``\ans $Q$'' and
    relative to any world $w$ denote the proposition that is true of any world
    that agrees with $w$ on all the propositions in the set denoted by $Q$
    relative to $w$.}%
  To ask a question by uttering a sentence $\phi$ is to propose that the current
  partitioned context set $\mathcal{C}$ be replaced by the new partitioned
  context set

\[\{p\co \exists w\in\bigcup\mathcal{C}.\ p = \sv{\phi}^{w}\cap \bigcup\mathcal{C}\}.\]

\end{definition}
%
We take each world $w$ in the prior context set in turn. We evaluate the matrix
question $\phi$ ($= \ans\ Q$) in $w$ and intersect the resulting proposition
with the context set, thus finding those worlds in the context set that agree
with $w$ on all the propositions in $\sv{Q}^{w}$. We collect the resulting set
of sets of worlds to serve as the new partitioned context set.

There is a tricky issue here, which is a reprise of what we discussed on
page~\pageref{par:context-uncertainty}ff. Take again a test scenario where
everyone knows that $a$ is a student but there's uncertainty about whether $b$
is. And there's maximal uncertainty about who called. Now, assume that \Next is
uttered against that background:

\ex Which students called?\xe
%
There are worlds in the context set where $a$ and $b$ are the students, and
other worlds where only $a$ is a student. We saw earlier that in such a context,
it is best to rule \Last out as pragmatically infelicitous because the set of
propositions denoted by the interrogative varies among the worlds in the context
set. This was easy enough to do in the system we were working with at that
point. Now, however, we are considering the possibility that what is being
uttered is really ``\ans which students called''. That structure has different
semantic values across the worlds in the context set simply because there are
different true answers in those worlds (otherwise why ask the question?). %
\note{In an earlier version of these lecture notes, it was erroneously claimed
  that we could evade the problem this way.}%
So, we can't enforce a presupposition that the interrogative have the same value
(the same strong true answer) across the context set.

The set of sets of worlds we would get from
Definition~\ref{def:updatebyquestiondraftrevision} in our test scenario would
not in fact be a partition of the context set:

\begin{enumerate}[(i)]
  \item a world $w$ where $a$ called and $b$ is not a student will be lumped
        with any world where $a$ called and $b$ is a student, no matter whether
        $b$ called (since the proposition that $b$ called is not in the
        denotation in $w$ of the underlying interrogative);
  \item a world $w'$ where $a$ called and $b$ is a student who called will be
        lumped only with other worlds where $a$ called and $b$ is a student who
        called;
  \item the two sets generated by $w$ and $w'$ are not disjoint, since they both
        contain worlds where $a$ called and $b$ is a student who called.
\end{enumerate}

The only way we can see to prevent this situation is to state directly in the
``update by question'' rule that the update is only felicitous if it results in
a partition:

\begin{definition}[Update by question,
  final revision]\label{def:updatebyquestionfinalrevision}%
  To ask a question by uttering a sentence $\phi$ is to propose that the current
  partitioned context set $\mathcal{C}$ be replaced by the new context set

  \[\{p\co \exists w\in\bigcup\mathcal{C}.\ p = \sv{\phi}^{w}\cap \bigcup\mathcal{C}\}\]

  If the resulting set of sets of worlds is not a partition of
  $\bigcup\mathcal{C}$, the utterance is infelicitous.
\end{definition}

\note{In particular, note that we have derived a condition on any wide-scope
  restriction on wh-phrases that requires the set characterized by that
  restriction to be ``settled'' in the prior context set. It would be
  interesting to compare this result to ideas about ``d-linking'' that are found
  in the literature on questions.}%
We conclude this subsection by noting that we have now encountered a kind of
presupposition that is not grounded in the denotational semantics of the
sentences uttered but emerges from the rules and principles of pragmatics.

\subsection{Embedding under rogative verbs}
\label{sec:embedding-rogative}

Access to the intension of the LF headed by \ans also figures plausibly in the
semantics of sentences with non-responsive (``rogative'') question-embedders,
such as the verbs \emph{ask} and \emph{wonder}. These verbs have lexical
meanings that suggest rough paraphrases in which \emph{know} or \emph{tell} is
in the scope of another intensional operator, e.g., \emph{ask} = `request to be
told', \emph{wonder} = `want to know'. Following Groenendijk \& Stokhof, let's
hypothesize that these verbs differ in semantic type from responsive verbs.
Their (internal) argument is of type $\type{s,\type{s,t}}$ rather than
$\type{s,t}$. A concrete entry for \emph{wonder} along these lines is \Next.

\note{$\textsc{des}$ maps an individual and a world to those worlds that satisfy
  the individual's desires as they are in the given world.}
\ex
$\svt{wonder}^{w} = \lambda q_{s,st}.\lambda x_{e}.$\\
\hfill$\forall w' [w'\in\textsc{des}(x,w) \rightarrow \forall w'' [w''\in\dox(x,w') \rightarrow q(w')(w'')=1]]$
\xe

The higher semantic type straightforwardly gives us the prediction that
\emph{wonder} cannot combine with a \emph{that}-clause. If we attempted to
interpret such an LF, we would encounter a type-mismatch. A \emph{that}-clause
has an extension of type $t$ and an intension of type $\type{s,t}$. Whether we
used plain FA or IFA, we wouldn't obtain the type-\type{s,st}-function that
wonder is looking for. But if we embed an interrogative clause (headed by \ans)
under \emph{wonder}, we will succeed. The extension of the mother-node of \ans
is type \type{s,t}, and its intension is type \type{s,st}. So IFA allows us to
interpret the structure.

\begin{exercise}
  Convince yourself that, given entry \Last, the predicted truth condition for
  \emph{Sakina wonders who called} matches the informal paraphrase that we gave
  in the text.
\end{exercise}

\section{Wh-movement with pied-piping and reconstruction}
\label{sec:pied-piping}

As an example of the phenomenon of so-called pied-piping, we will analyze the
sentence \emph{How many cats did you adopt?} Before we get to the point that
makes the example interesting for our purposes, we must fill in a rudimentary
account of plurals and gradability.

\subsection{Background on plurals and \emph{many}}
\label{sec:backgr-plur-many}

The basic idea in most current semantic treatments of plural DPs is that plural
definites and pronouns denote entities in $D_{e}$, just like singular definites,
pronouns, and proper names. The only difference is that the entities denoted by
plurals are more complex (and typically spatially discontinuous). A distinction
is made within the domain $D_{e}$, between so-called ``atoms'' or ``atomic
individuals'' (the referents of singular DPs) and ``pluralities'' or
``non-atomic individuals'' (the referents of plural DPs). Non-atomic individuals
contain atomic individuals as (proper) parts; e.g., if Lucy is one of the
players, then $\svt{Lucy}$ (i.e., Lucy) is an atomic part of
$\svt{the players}$. An atomic individual, on the other hand, has no atomic
parts other than itself. (An atom counts as an atomic part of itself.)

Given that $D_{e}$ contains pluralities along with atoms, predicate extensions
of type $\type{e,t}$ are functions that apply to both atoms and pluralities. In
the case of common nouns, English has a morphological number distinction which
seems to have semantic import:

\pex
\a $\svt{cat}^{w} = \lambda x.$ is a cat in $w$
\a $\svt{cats}^{w} = \lambda x.$ every atomic part of $x$ is a cat in $w$
\xe
%
\note{Notice that an entity that $\svt{cats}^{w}$ maps to 1 is not
  \emph{necessarily} a plurality. As interpreted in \Last[b], the plural noun
  \emph{cats} is also true of a single cat, because of the fact that an atom is
  an atomic part of itself. It would be possible to revise \Last[b] so that it
  requires $x$ to be non-atomic. But as we will see below, the current
  formulation actually works better. The reason is, in a nutshell, that ``One.''
  is a perfectly good answer to a \emph{how-many} question.}%
Being a cat entails being an atomic individual (this is how we agree to
understand our metalanguage). Therefore, the denotation of the singular noun as
defined in \Last[a] maps every plurality to 0. The pluralized noun in \Last[b],
on the other hand, maps to 1 those pluralities whose atomic parts are all cats.

Verbs can show morphological number too, but we assume that this is always due
to morphological agreement with a number-marked subject, and that the number
morphology on the verb is not interpreted itself. As far as semantics is
concerned, verbs are ``number neutral'' and typically can be true
indiscriminately of both atoms and pluralities. This is reflected, for example,
in a lexical entry like \Next.

\ex $\svt{meow}^{w} = \lambda x.$ every atomic part of $x$ meows in $w$ \xe
%
The condition in \Last can be met by both pluralities and atoms. A plurality is
mapped to 1 iff all its atomic parts meow, and an atom is mapped to 1 if it
itself meows. (Recall that every atom counts as an atomic part of itself.)

We can count the atomic parts of a plural individual. For example, the plural
individual composed of Lucy, Kathellen, and Shanice has 3 atomic parts. Let's
have a concise notation for this.

\ex Let $x$ be an element of $D_{e}$. Then\\
    $\#(x) :=$ the cardinality of the set $\{y\co y\ \text{is an atomic part
      of}\ x\}$.
\xe

\note{Actually, Hackl assumes (with most of the literature on adjectives and
  gradability) that there is an additional basic type $d$ (for ``degrees'')
  separate from type $e$. The number argument of \emph{many} is a special case
  of a degree argument, and the type for \emph{many} is then
  $\type{d,\type{et,\type{et,t}}}$.}%
\note{Hackl's analysis implies that the superficially simplest uses of
  \emph{many}, as in \emph{Many cats meowed}, are actually more complex at LF:
  the argument position of \emph{many} is bound by a covert \textsc{pos}
  (``positive operator''), which is a quantifier over numbers (degrees) and
  means something like `a number (degree) above the contextually specified
  threshold'.}%
With this little bit of plural semantics in place, we can now introduce
\cite{hackl-2001-thesis}'s semantics for \emph{many} and an appropriate
semantics for interrogative \emph{how} that will go with it. Hackl proposes that
\emph{many} is not by itself a quantificational determiner of type
$\type{et,\type{et,t}}$. Rather it is looking for an argument which denotes a
number, and only after it has been saturated with such an argument, the
resulting phrase is a quantificational determiner. So the type of \emph{many} is
type $\type{e,\type{et,\type{et,t}}}$ \dash assuming that numbers are abstract
individuals of some kind, hence members of $D_{e}$ \dash and its entry is as in
\Next.

\ex $\svt{many} = \lambda n\co n$ is a number.
$\lambda f_{et}.\lambda g_{et}.\ \exists x [\#(x)=n\ \&\ f(x)=1\ \&\ g(x)=1]$
\xe
%
When building a sentence with \emph{many}, in the simplest case we would fill
the first argument slot of \emph{many} with a word that refers to a number. This
might be an anaphoric demonstrative pronoun \emph{that}, which in appropriate
discourse contexts can refer to a previously mentioned number, say the number 3.
Or it could be a numeral word, like \emph{three}, which we take here to have a
meaning of type $e$ and be a proper name of the number 3. These options give us
interpretable syntactic representations like \Next[a,b].

\pex\label{ex:that-three}
\a\null [ [ that$_{7}$ many ] cats ] meowed
\a\null [ [ three many ] dogs ] barked
\xe

\Last[a] is straightforwardly pronounced as it stands, whereas for \Last[b],
Hackl assumes that \emph{many} is unpronounced after numeral words, so this
structure surfaces as \emph{Three dogs barked}. Let us compute truth-conditions,
using \LLast. Suppose we have a contextually given assignment for \Last[a] which
maps the variable 7 to the number 3, and we evaluate the sentence in the actual
world $@$. Then, by using FA three times to apply $\svt{many}$ to its three
arguments, we obtain the truth-condition in \Next.

\ex $\exists x [\#(x)=3\ \&\ \svt{cats}^{@}(x)=1\ \&\ \svt{meow}^{@}(x)=1]$ \xe
%
Now we use our entries for \emph{cats} and \emph{meow}, and this becomes \Next.

\ex $\exists x [ \#(x) = 3$\\
$\&$ every atomic part of $x$ is a cat in @\\
$\&$ every atomic part of $x$ meows in @ $]$
\xe
%
In other words, there is a plurality composed of three meowing cats. Which is
loud but correct.

\clearpage
\subsection{\emph{How many}-questions}
\label{sec:how-many-qs}

\note{The main source for the argument in this section is Arnim von Stechow's
  paper ``Against LF pied-piping'' \parencite{stechow-1996-LFPiedPiping}.}%
In a \emph{how-many} question, the argument slot that was saturated by that or
three in \refx{ex:that-three} is instead occupied by the wh-word \emph{how}. In
Karttunen’s theory, this will be an existential quantifier, equivalent to
\emph{some number}.

\ex $\svt{how} = \lambda f_{et}.\ \exists n [n\ \text{is a
  number}\ \&\ f(n)=1]$\\
\note{For Hackl, it would be type $\type{dt,t}$, a generalized quantifier over
  degrees.}%
(type $\type{et,t}$, a generalized quantifier)
\xe
%
This semantic type is not interpretable in situ as the sister of \emph{many},
and must undergo (covert) movement for interpretability. With this in mind,
let’s attempt a syntactic derivation for the question \emph{How many cats did
  you adopt?}

\pex\label{ex:LF-cats}
\a\null [$_{C}$ ? OP] [you adopted how many cats]
\a OP 5 [ [? $t_{5}$] [you adopted how many cats] ]
\a OP 5[how many cats 1 [[? $t_{5}$] you adopted $t_{1}$]
\a OP 5[how 2[$t_{2}$ many cats 1[[? $t_{5}$] you adopted $t_{1}$] ] ]
\xe
%
We have three movements that derive the LF from the base generated structure in
\Last[a]: the movement of the empty OP in \Last[b], the wh-movement of the
wh-phrase \emph{how many cats} in \Last[c], and the QR of the quantifier
\emph{how} in \Last[d].

We can check the semantic types to confirm that we have derived an interpretable
structure (do this as an exercise). Let’s compute what \Last[d] means. (The details
are left as an exercise. Here we conflate sets of propositions with their
characteristic functions.)

\ex $\svt{\Last[d]}^{@} =$\\
\dots\\
$= \{p\co \exists n\ [n\ \text{is a
  number}\ \&\ \exists x [\#(x)=n\ \&\ \svt{cats}^{@}(x)=1]\ \&$\\
\hfill$p = \lambda w.\svt{adopt}^{w}(x)(\text{you})]\}$
\xe
%
\note{We are exploiting the equivalence of various scopal arrangements in a
  formula with existential quantifiers and conjunctions. The following four are
  all equivalent.

  $\exists x[Fx\ \&\ \exists y[Rxy\ \&\ Gy]]$\\
  $\exists x\exists y[Fx\ \&\ Rxy\ \&\ Gy]$\\
  $\exists y\exists x[Fx\ \&\ Rxy\ \&\ Gy]$\\
  $\exists y[\exists x[Fx\ \&\ Rxy]\ \&\ Gy]$}%
With a little bit of Predicate Logic reasoning, we can rewrite this equivalently
as follows:

\ex $\{p\co \exists x [\boxed{\exists n [n\ \text{is a
  number}\ \&\ \#(x)=n]}\ \&\ \svt{cats}^{@}(x)=1\ \&$\\
\hfill$p = \lambda w.\svt{adopt}^{w}(x)(\text{you})]\}$
\xe
%
We can now contemplate the underlined part and convince ourselves that this part
is a tautology. It just says that $x$ has some number or other of atomic parts,
which cannot fail to be true. So we might as well drop this conjunct and rewrite
\Last as \Next.

\ex $\{p\co \exists x [\svt{cats}^{@}(x)=1\ \&\
p = \lambda w.\svt{adopt}^{w}(x)(\text{you})]\}$
\xe
%
Interestingly now, this is precisely the meaning we would have derived for the
question \emph{Which cats did you adopt?} In other words, if
\refx[d]{ex:LF-cats} really were the LF (or one of the LFs) for the \emph{how-
  many}-question \emph{How many cats did you adopt?}, then we would be making
the bad prediction that this question is synonymous with (or at least shares a
reading with) the \emph{which}-question \emph{Which cats did you adopt?} This
would be unfortunate for our theory.

Upon closer inspection of \refx[d]{ex:LF-cats}, however, it turns out that our
theory does not actually generate this LF. We have neglected to check whether
\refx[d]{ex:LF-cats} conforms to the Wh-Licensing Principle, repeated here.

\ex
At LF, a phrase $\alpha$ occupies a specifier position of \emph{?} if and only
if $\alpha$ has the feature [WH].
\xe
%
In \refx[d]{ex:LF-cats}, we have two phrases that are scoped above ?, namely
\emph{how} and \emph{$t_{2}$ many cats}. Let’s say that the positions they
occupy both count as specifier positions of ? (similar to what one has to say
for multiple questions like \emph{who ate what?} \dash see your homework). Then
\Last would require that both of these phrases have the feature [WH]. But only
\emph{how} actually does, at least on our current assumption that [WH] is a
lexical property possessed by only a small set of words. The other phrase that
is scoped above ? in \refx[d]{ex:LF-cats} is \emph{$t_{2}$ many cats}, which
does not carry the feature [WH]. So the structure \refx[d]{ex:LF-cats} is
filtered out by the Wh-Licensing principle as syntactically ill-formed. And this
is a good thing, because it means we don't generate the unwelcome reading in
\LLast.

We still have to worry, however, about how we generate the reading that our
example actually does have. Is there a second, well-formed, LF for our example?
The answer is yes if our syntax allows for ``reconstruction'', i.e. some
mechanism by which overtly moved phrases can be restored to (one of) their
pre-moved positions at LF. In order to satisfy the Wh-Licensing Principle,
reconstruction must apply to the phrase \emph{$t_{2}$ many cats} (though not, of
course, to \emph{how}). Where can this phrase be reconstructed to? Well, if we
restored it all the way down to its original base position as the object of
\emph{adopt}, it wouldn't be interpretable there, because quantifiers are not
interpretable in object positions. But if we can assume that wh-movement
proceeds (or at least is allowed to proceed) successive cyclically, and that an
object wh-phrase can stop over e.g. at the edge of VP before it moves on to
Spec-CP, we can target this intermediate landing site for reconstruction.
(Assuming the VP-internal subject hypothesis, VPs are semantically of type $t$
and hence quantifiers are interpretable at their edges.) By means of
reconstruction, we can thus obtain another LF that is both interpretable and in
compliance with the WH-Licensing principle.

\ex OP 5[how 2[[? $t_{5}$] $t_{2}$ many cats 1[you adopted $t_{1}$] ] ] \xe
%
The denotation of \Last (which you should compute as an exercise!) is \Next.

\ex $\{p\co \exists n [n\ \text{is a
  number}\ \&\ p = \lambda w.\ \exists x [\#(x)=n\ \&\ \svt{cats}^{w}(x)=1\ \&\
\svt{adopt}^{w}(x)(\text{you})]\}$
\xe
%
This set contains one proposition per number. It contains the proposition that
you adopted 1 cat, the proposition that you adopted 2 cats, the proposition that
you adopted 3 cats, etc. This prediction accords well with what we feel are
expected answers to the question \emph{how many cats did you adopt?}

\subsection{More on reconstruction}
\label{sec:more-reconstruction}

Earlier in these notes in Section~\ref{sec:syntactic-reconstruction}, we briefly
introduced a version of the ``copy theory of movement'', in which reconstruction
can be implemented as deletion of a higher copy. We reproduce the relevant
passage here:

\begin{quote}
  \begin{small}
    This assumes that movement generally
    proceeds in two separate steps, rather than as a single complex operation as we
    have assumed so far. Recall that in H\&K, it was stipulated that every
    movement effects the following four changes:

    \begin{enumerate}
      [(i)]
      \item a phrase $\alpha$ is deleted,
      \item an index \emph{i} is attached to the resulting empty node (making it a
      so-called trace, which the semantic rule for ``Pronouns and Traces''
      recognizes as a variable),
      \item a new copy of $\alpha$ is created somewhere else in the tree (at the
      ``landing site''), and
      \item the sister-constituent of this new copy gets another instance of the index
      \emph{i} adjoined to it (which the semantic rule of Predicate Abstraction
      recognizes as a binder index).
    \end{enumerate}

    If we adopt the Copy Theory, we assume instead that there are three distinct
    operations:

    \begin{description}

      \item[``Copy'':] Create a new copy of $\alpha$ somewhere in the tree, attach an
            index \emph{i} to the original $\alpha$, and adjoin another instance of
            \emph{i} to the sister of the new copy of $\alpha$. (= steps (ii), (iii), and
            (iv) above)

      \item[``Delete Lower Copy'':] Delete the original $\alpha$. (= step (i) above)

      \item[``Delete Upper Copy'':] Delete the new copy of $\alpha$ and both instances
            of \emph{i}.
    \end{description}
    %
    The Copy operation is part of every movement operation, and can happen anywhere
    in the syntactic derivation. The Delete operations happen at the end of the LF
    derivation and at the end of the PF deletion. We have a choice of applying
    either Delete Lower Copy or Delete Upper Copy to each pair of copies, and we can
    make this choice independently at LF and at PF. (E.g., we can do Copy in the
    common part of the derivation and than Delete Lower Copy at LF and Delete Upper
    Copy at PF.)
  \end{small}
\end{quote}

When we try to apply this machinery to reconstruction of pied-piped material in
wh-movement, there is a detail that needs attention: After having created two
copies (or more) of \emph{how many cats}, what happens when we ``move''
\emph{how}? Presumably we create another, higher, copy of \emph{how} and coindex
it with the lower copy of \emph{how}. With which lower copy of \emph{how}? With
the one in the uppermost copy of \emph{how many cats}? If so, then after
deletion of that uppermost copy of \emph{how many cats}, the binder index next
to the top copy of \emph{how} will no longer bind any trace and the structure
will not be interpretable. What we want is to end up with is an LF in which the
top copy of \emph{how} binds a variable in a lower (retained) copy of \emph{how
  many cats}. There may be various ways to achieve this. Perhaps we can simply
choose a derivation in which \emph{how} ``moves'' out of a lower copy of
\emph{how many cats}, not out the highest copy.
(\cite{richards-1998-PrincipleMinimalCompliance}'s Principle of Minimal
Compliance might explain why this long movement is legitimate and yet relies for
its legitimacy on the previous movement of the larger phrase.) Alternatively, we
may posit a general principle ``Copies must remain copies'', which says that, as
long as a structure contains more than one copy of a given phrase, every
alteration to one of these copies must identically affect them all. In our case,
this would mean that when we create the top copy of \emph{how}, we must coindex
it with all its existing copies, in the highest as well as all lower instances
of \emph{how many cats}. This principle could hopefully be made to fall out from
a suitable formalization of copy theory, perhaps in a multi-dominance framework.

%%% Local Variables:
%%% mode: latex
%%% TeX-master: "IntensionalSemantics"
%%% End:



