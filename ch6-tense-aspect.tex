% chktex-file 1
\newcommand{\pres}{\textsc{pres}\xspace}
\newcommand{\past}{\textsc{past}\xspace}
\newcommand{\pfv}{\textsc{pfv}\xspace}
\newcommand{\prog}{\textsc{prog}\xspace}

\chapter{Beginnings of tense and aspect}\label{cha:tense-aspect}

\minitoc

\section{A first proposal for tense}
\label{sec:first-proposal}

\note{This chapter is even more the outcome of collaborative efforts than other
  chapters. We are very much indebted to Roger Schwarzschild, who has used our
  notes several times in his teaching and is the source of many edits and
  additions.}%
Tense logic, or temporal logic, is a branch of logic first developed by the
aptly named Arthur Prior in a series of works, in which he proposed treating
tense in a way that is formally quite parallel to the treatment of modality
discussed in Chapter 3. Since tense logic (and modal logic) typically is
formulated at a high level of abstraction regarding the structure of sentences,
it doesn’t concern itself with the internal make-up of ``atomic'' sentences and
thus treats tenses as sentential operators (again, in parallel to the way modal
operators are typically treated in modal logic). We will begin by integrating a
version of Prior’s tense logic into our framework.

\note{We remain vague for now about what we mean by ``times'' (points in time?
  time intervals?). This will soon need clarification, and we will decide that
  we should mean ``intervals''.}%
The first step is to switch to a version of our intensional semantic system
where instead of a world parameter, the evaluation function is sensitive to a
parameter that is a pair of a world and a time. Such a pair will also be called
an ``index''. We use metalanguage variables $i$, $i'$, \dots\ for indices, and
write $w_{i}$ and $t_{i}$ to pick out the world in $i$ and the time in $i$
respectively (i.e., $i = \pair{w_{i},t_{i}}$). Predicates will now have lexical
entries that incorporate their sensitivity to both worlds and times:

\ex $\svt{tired}^{i} = \lambda x\in D.\ x\ \text{is tired
  in}\ w_{i}\ \text{at}\ t_{i}$ \xe

The composition principles from Heim \& Kratzer and the preceding chapters stay
the same, except that type $s$ is now the type of indices, and intensions are
functions from indices to extensions. For example, the intension of sentence is
now a function from world-time pairs to truth-values. %
\note{This necessitates a slight rewriting of our previous entries for modals
  and attitude verbs. We will attend to this when we get to relevant examples
  later on.}%
We might call this a ``temporal proposition'', to distinguish it from a function
from just worlds to truth-values, but we will often just call it a
``proposition''.

In this framework, we can formulate a very simple-minded first analysis of the
present and past tenses and the future auxiliary \emph{will.} As for (LF) syntax
let’s assume that complete sentences are TPs, headed by T (for ``tense''). There
are two morphemes of the functional category T, namely \past (past tense) and
\pres (present tense). The complement of T is an MP or a VP. MP is headed by M
(for ``modal''). Morphemes of the category M include the modal auxiliaries
\emph{must}, \emph{can}, etc., which we talked about in previous chapters, the
semantically vacuous \emph{do} (in so-called ``\emph{do}-support'' structures),
and the future auxiliary \emph{will}. Evidently, this is a semantically
heterogeneous category, grouped together solely because of their common syntax
(they are all in complementary distribution with each other). The complement of
M is a VP. When the sentence contains none of the items in the category M, we
assume that MP isn’t projected at all; the complement of T is just a VP in this
case. %
\note{Many subordinate clauses \dash those we call ``finite'' \dash also always
  have a TP. As for embedded clauses more generally (including infinitives
  etc.), we don't need to take a stand here.}%
(TP is always projected in a root clause, whether there is an MP or not.) We
thus have LF-structures like the following. (The corresponding surface sentences
are given below, and we won’t be explicit about the derivational relation
between these and the LFs. Assume your favorite theories of syntax and
morphology here.)

\ex \lb[TP] Svenja \lb[T$'$] \pres \lb[VP] t \lb[V$'$] be tired \rb \rb \rb \rb\\
= Svenja is tired.\label{ex:is-tired} \xe

\ex~ \lb[TP] Svenja \lb[T$'$] \past \lb[VP] t \lb[V$'$] be tired \rb \rb \rb \rb\\
= Svenja was tired.\label{ex:was-tired} \xe

\ex~ \lb[TP] Svenja \lb[T$'$] \pres \lb[MP] t \lb[M$'$] woll \lb[VP] t \lb[V$'$]
be tired \rb \rb \rb \rb \rb \rb\\
= Svenja will be tired. \xe
%
\note{We use ``\emph{woll}'' as the name of the root underlying \emph{will} and
  \emph{would}, following \cite{abusch-1988-wccfl} and
  \cite[p.32]{ogihara-1989-thesis}; \textcite[fn.14, p.22]{abusch-1997-sequence}
  attributes the coinage of \emph{woll} to Mats Rooth in class lectures at UT
  Austin.}%
\emph{woll} in \Last stands for the underlying uninflected form of the auxiliary
which surfaces as \emph{will} in the present tense (and as \emph{would} in the
past tense). When we have proper name subjects, we will assume for simplicity
that they are reconstructed into their VP-internal base position.

What are the meanings of \pres, \past, and \emph{woll}? For \pres, the simplest
assumption that seems to work is that it is semantically vacuous. This means
that the interpretation of the LF in \refx{ex:is-tired} is identical to the
interpretation of the bare VP \emph{Svenja be tired}:

\ex For any index $i$: $\svt{\pres (Svenja be tired)}^i$ =
$\svt{Svenja be tired}^{i}$ = 1\\ iff Svenja is tired in $w_{i}$ at $t_{i}$.\xe
%
Does this adequately capture the intuitive truth-conditions of the sentence
\emph{Svenja is tired}? It does if we make the following general
assumption:

\ex Utterance Rule:\\
An utterance of a sentence (= LF) $\phi$ that is made in a world $w$ at a time
$t$ counts as true iff $\sv{\phi}^{\pair{w,t}} = 1$ (and as false if
$\sv{\phi}^{\pair{w,t}} = 0$).
\xe
%
This assumption ensures that (unembedded) sentences are, in effect, interpreted
as claims about the time at which they are uttered (``utterance time'' or
``speech time''). If we make this assumption and we stick to the lexical entries
we have adopted, then we are driven to conclude that the present tense has no
semantic job to do. A tenseless VP \emph{Svenja be tired} would in principle be
just as good as \refx{ex:is-tired} to express the assertion that Svenja is tired
at the utterance time. Apparently, it is just not well-formed as an unembedded
structure, but this fact then must be attributed to principles of syntax rather
than semantics.

What about \past? When a sentence like \refx{ex:was-tired} \emph{Svenja was
  tired} is uttered at a time $t$, then what are the conditions under which this
utterance is judged to be true? A quick answer is: an utterance of
\refx{ex:was-tired} at $t$ is true iff there is some time before $t$ at which
Svenja is tired. This suggests the following entry:

\ex For any index $i$:
$\svt{\past}^{i} = \lambda p \in D_{\type{s,t}}.\
\exists t\ \text{before}\ t_{i}\co p(\pair{w_{i,},t}) = 1$ \xe
%
So, the past tense seems to be an existential quantifier over times, restricted
to times before the utterance time.

\note{Are there also tenses with universal force ? Two possible candidates that
  call for closer examination: gnomic tenses (e.g. in Ancient Greek), and the
  (universal reading of the) English perfect (as in \emph{I have been tired
    since yesterday morning)}. Both have been written about in the formal
  semantics literature (the latter extensively \dash you could start with
  \cite{iatridou-anagnostopolou-izvorski-2001-perfect} and
  \cite{fintel-iatridou-2019-since}.}%
For \emph{will}, we can say something completely analogous:

\ex For any index $i$:
$\svt{woll}^{i} = \lambda p \in D_{\type{s,t}}.\
\exists t\ \text{after}\ t_{i}\co p(\pair{w_{i,},t}) = 1$ \xe
%
Apparently, \past and \emph{woll} are semantically alike, even mirror images of
each other, though they are of different syntactic categories. The fact that
\past is the topmost head in its sentence, while \emph{woll} appears below
\pres, is due to the fact that our syntax happens to require a T-node in every
complete sentence. Semantically, this has no effect, since \pres is vacuous.

\note{Strict linear orders are transitive, irreflexive, asymmetric, and
  connected. See Section~\ref{subsec:math-orderings} on the basics of order
  theory. Some rethinking is needed once we move to intervals.}%
Both \LLast and \Last presuppose that the set or times comes with an intrinsic
order. For concreteness, assume that the relation ‘precedes’ (in symbols: <) is
a strict linear order on the set of all times. The relation ‘follows’, of
course, can be defined in terms of ‘precedes’ ($t$ follows $t'$ iff $t'$
precedes $t$).

\section{Time frame adverbials}
\label{sec:frames}

In this section, we take a brief look at temporal adverbials, specifically
so-called frame adverbials, such as:

\ex Svenja was tired on February 1, 2001.\label{ex:on} \xe
%
There are two ideas that come to mind. One is that phrases like \emph{on
  February 1, 2001} are restrictors of temporal operators (kind of like
\emph{if}-clauses are restrictors of modals). The other idea is that they are
modifiers of the proposition in the temporal operator's scope. If we want to go
with the first idea, we have to make some changes. Our current \past and
\emph{woll} are unrestricted (1-place) operators, so there is no place for a
restrictor. The second idea is easier to implement, and we try that first.

A propositional modifier is a function from propositions to truth-values, where
``proposition'' for us now means ``temporal proposition''. Here is an entry for
on February 1, 2001. %
\note{Technically, the modifier returns a truth-value, not a proposition. We get
  back a proposition only when we compute the intension of the phrase that
  includes the modifier.}%
Intuitively, this modifier takes a proposition and returns a proposition that
puts an added condition on the time-coordinate of its index-argument.

\ex
$\svt{on February 1, 2001}^{i} = \lambda p \in D_{\type{s,t}}.$\\
\hfill$[p(i) = 1\ \&\ t_{i}\ \text{is part of February 1, 2001}]$\label{ex:pp}
\xe

\ex~ LF: \past \lb[VP] \lb[VP] Svenja be tired \rb \lb[PP] on February 1, 2001 \rb
\rb\label{ex:on-lf} \xe

\begin{exercise}
  Imagine that sentence \refx{ex:on} is not given the LF in \Last, but this one,
  with the PP attached higher:

  \ex LF: \lb[T$'$] \past \lb[VP] Svenja be tired \rb \rb\, \lb[PP] on
  February 1, 2001\rb \xe
%
  What would the truth-conditions of this LF be? Does this result correspond at
  all to a possible reading of this sentence (or any other analogous sentence)?
  If not, how could we prevent such an LF from being produced? \qed

\end{exercise}

The truth conditions that we derive given \refx{ex:pp} and \refx{ex:on-lf} look
good: the sentence is predicted true as uttered if there is a time which is both
before the utterance time and within Feb 1, 2001 and at which Svenja is tired, and
it is predicted false if there is no such time. But arguably this is not
\emph{exactly} right. Suppose that somebody uttered this sentence at an
utterance time that \emph{preceded} the date in the adverbial, say at some time
in the year 2000. Our analysis predicts that this utterance is false. But in
fact it feels more like a presupposition failure; the speaker is heard to be
taking for granted that Feb 1, 2001 is in the past of his speaking. Standard
presupposition tests confirm this. For example, the negated sentence (\emph{Svenja
  wasn't tired on Feb 1, 2001}) and the polar question (\emph{Was Svenja tired on
  Feb 1, 2001?}) also convey that the speaker assumes he is speaking after Feb
1, 2001.

\clearpage
\note{It also has the virtue of avoiding the potential overgeneration issue that
  you looked at in the exercise above. Q: How so?}%
If we want to account for this more fine-grained intuition, the restrictor
approach has an advantage after all. Let's revise the entries for \past and
\emph{woll} so that they denote 2-place operators, and moreover they encode a
non-emptiness presupposition.

\ex For any index $i$:\\
$\svt{\past}^{i} = \lambda p\co \exists t [t < t_{i}\ \&\ p(w_{i},t) = 1].$\\
\hfill $\lambda q. \exists t [ t < t_{i}\ \&\ p(w_{i},t)=1\ \&\ q(w_{i},t) = 1]$
\xe

\ex~%
\note{How about present tense? Should we make this presuppositonal as well \dash
  which would imply it is not, after all, completely vacuous? Frame adverbials
  in present tense sentences do occur. Typically they are adverbials like
  \emph{today, on this beautiful Monday}, which in virtue of their own meaning
  already are required to contain the speech time. The following entry would
  make room for them and duplicate this requirement as a presupposition:

  \ex For any index $i$:
  $\svt{\pres}^{i} = \lambda p\in D_{\type{s,t}}\co p(w_{i},t_{i}).\
  \lambda q\in D_{\type{s,t}}.\ q(w_{i},t_{i}) = 1$ \xe
%
  For the purposes of these lecture notes, we leave the matter open and stick
  with the vacuous meaning for \pres in the discussions of the upcoming sections.
}%
For any index $i$:\\
$\svt{woll}^{i} = \lambda p\co \exists t [t > t_{i}\ \&\ p(w_{i},t) = 1].$\\
\hfill $\lambda q. \exists t [ t > t_{i}\ \&\ p(w_{i},t)=1\ \&\ q(w_{i},t) = 1]$
\xe
%
Furthermore, let's change (i.e., simplify) the meaning of the adverbial so that
it has a suitable type to serve as the temporal operator's first argument. The
LF-structure must be accordingly different as well. Instead of \refx{ex:on-lf}
above, we now posit \NNext, where the adverb forms a constituent with the tense.
This requires the surface order to be derived by some reordering, perhaps
extraposition of the adverbial.

\ex\label{ex:on-part}%
$\svt{on February 1, 2001}^{i} = 1$ iff $t_{i}$ is part of February 1, 2001
\xe

\ex~ LF: \lb[T$'$] \past \lb[PP] on February 1, 2001 \rb \rb \lb[VP] Svenja be
tired \rb \label{ex:pp-lf}\xe
%
The meanings we now derive contain the desired presuppositions: The past tense
sentence \refx{ex:on} presupposes that Feb 1, 2001 is at least in part before
the utterance time, the future sentence \emph{Svenja will be tired on Feb 1, 2001}
presupposes that this date is at least in part after the utterance time, and the
present tense sentence \emph{Svenja is tired on Feb 1, 2001} presupposes that the
utterance time is on this date. Apart from the presuppositions, the meanings are
the same as before. On the down-side, the new analysis posits both more complex
meanings for the tenses and a less direct correspondence between LF constituency
and surface structure. Furthermore, how is it supposed to apply to simple
sentences without adverbials? Not every tensed sentence contains an obligatory
frame adverb, after all. We are forced to say there is a covert restrictor
whenever there isn't an overt one. But this, upon reflection, turns out to be a
virtue, as we will see in the next section.

\note{Considering the mismatch between the LF in \refx{ex:pp-lf} and the surface
  order, note that we saw a similar issue of apparent mismatch when we decided
  to treat \emph{if}-clauses as restrictors of modals. Both issues might be
  addressed by simply letting modal and temporal operators take their arguments
  in the opposite order (something suggested by
  \cite{chierchia-1995-dynamics-book}). Rewriting the lexical entries in this
  way is a routine exercise. We leave this matter open. The syntax of frame
  adverbials is a non-trivial object of study.}%
\begin{exercise}
When a quantifier appears in a tensed sentence, we expect two scope construals.
Consider a sentence like this:

\ex Every professor (in the department) was a teenager in the Sixties. \xe
%
We can imagine two LFs:

\ex\ [\past in the sixties]  [every professor be a teenager] \xe

\ex~\ [every professor] 7 [ [\past in the sixties] [$t_{7}$ be a teenager] ] \xe
%
Describe the different truth-conditions which our system assigns to the two LFs.
Is the sentence ambiguous in this way? If not this sentence, are there analogous
sentences that do have the ambiguity? \qed
\end{exercise}

\begin{exercise}
  Our official entry for \emph{every} makes it a time-insensitive (and
  world-insensitive) item:

  \ex For any $i$,
  $\svt{every}^{i} = \lambda f_{\type{e,t}}.\lambda g_{\type{e,t}}.\ \forall x\co
  f(x)=1 \rightarrow g(x)=1$ \xe
%
  Consider now two possible variants (we have boxed the portion where they
  differ):

  \ex For any $i$,
  $\svt{every}^{i} = \lambda f_{\type{e,t}}.\lambda g_{\type{e,t}}.\ \forall x
  \ \boxed{\text{at}\ t_{i}}\co
  f(x)=1 \rightarrow g(x)=1$ \xe

  \ex~ For any $i$,
  $\svt{every}^{i} = \lambda f_{\type{e,t}}.\lambda g_{\type{e,t}}.\ \forall x\co
  f(x)=1\ \boxed{\text{at}\ t_{i}} \rightarrow g(x)=1$ \xe
  %
  Does either of these alternative entries make sense? If so, what does it say?
  Is it equivalent to our official entry? Could it lead to different predictions
  about the truth-conditions of English sentences? \qed
\end{exercise}

\section{Are tenses referential?}
\label{sec:referential}

Our first semantics for the past tense, in Section~\ref{sec:first-proposal},
treated it as an unrestricted existential quantifier over times. This seems
quite adequate for examples like \Next, which seem to display the expected
unrestricted existential meaning:

\ex Georgia went to a private school.\label{ex:school}\xe
%
All we learn from \Last is that at some point in the past, whenever it was that
Georgia went to school, she went to a private school.

Partee in her famous paper ``Some structural analogies between tenses and
pronouns in English'' \parencite{partee-1973-analogies} presented an example
where tense appears to act more ``referentially'':

\ex I didn’t turn off the stove.\label{ex:stove} \xe
%
``When uttered, for instance, halfway down the turnpike, such a sentence clearly
does not mean either that there exists some time in the past at which I did not
turn off the stove or that there exists no time in the past at which I turned
off the stove. The sentence clearly refers to a particular time \dash not a
particular instant, most likely, but a definite interval whose identity is
generally clear from the extralinguistic context, just as the identity of the
\emph{he} in [\emph{He shouldn’t be in here}] is clear from the context.''

Partee argues, in effect, that neither of the two plausible LFs that our system
from Section~\ref{sec:first-proposal} derives can correctly capture the meaning
of \Last. Given that the sentence contains a past tense and a negation, there
are two possible scopings of the two operators:

\pex
\a \past \textsc{neg} I turn off the stove.
\a \textsc{neg} \past I turn off the stove.
\xe

\begin{exercise}
  Using our old semantics from \ref{sec:first-proposal}, show that neither LF in
  \Last captures the meaning of \LLast correctly.
\end{exercise}

In a commentary on Partee’s paper (at the same conference it was presented at),
Stalnaker pointed out that a minor amendment of the Priorean theory can deal
with \LLast. One just needs to allow the existential quantifier to be
contextually restricted to times in a salient interval. Since natural language
quantifiers are typically subject to contextual restrictions, this is not a
problematic assumption. Note that Partee formulated her observation in quite a
circumspect way: ``The sentence refers to a particular time''; Stalnaker’s
suggestion was that the reference to a particular time is part of the
restriction to the quantifier over times expressed by tense, rather than tense
itself being a referring item.

\cite{ogihara-1995-tense-embedded, ogihara-1996-tense-book} argued that the
restricted existential quantification view is in fact superior to Partee’s
analysis, since Partee’s analysis needs an existential quantifier anyway. It is
clear that the time being referred to in the stove-sentence \LLast is a
protracted interval (the time during which Partee was preparing to leave her
house). But the sentence is not interpreted as merely saying that this interval
is not a time at which she turned off her stove. That would only exclude a
fairly absurd kind of slow-motion turning-off-of-the-stove (turning off the
stove only takes a moment). Instead, the sentence says that in the salient
interval there is no time at which she turned off the stove. %
\note{The alternative is to say that the existential quantifier is not expressed
  by tense but comes from somewhere else \dash perhaps from aspect, or from the
  lexical entry of the verb itself. We will come back to these options.}%
Clearly, we need an existential quantifier in there somewhere and the Priorean
theory provides one.

Ogihara made the point with the following example:

\pex[labelformat=A:,samplelabel=Patricia]
\a[label=Patricia] Did you see Solène?
\a[label=Lea] Yes, I saw her, but I don’t remember exactly when.
\xe
%
The question and answer in this dialogue concern the issue of whether Lea saw
Solène at some time in a contextually salient interval.

Stalnaker's and Ogihara's conclusions converge with what we already ended up
with in Section~\ref{sec:frames}, after considering the interaction of tenses
with time frame adverbials. In order to capture presuppositions of tensed
sentences with frame adverbials, we already modified Prior's original proposal
and made room for a restrictor in the semantics of the past tense. Given this
revised analysis of the past tense as a 2-place existential quantifier, it is
unsurprising, in fact expected, that an implicit, contextually salient
restrictor should be present when there isn't an overt one. What then about
example \refx{ex:school}, \emph{Georgia went to a private school}, for which the
unrestricted analysis seemed to do well? Let us say that the covert restrictor
in this case picks out a very long interval, perhaps Georgia's entire life-time,
or even the entire past from the big bang to the utterance time, or all
eternity. (What exactly the right restrictor is in this case, and what makes it
contextually available, may be a bit unclear, but we leave it at that.)

\begin{exercise}
  Assuming the restricted existential quantifier analysis of past tense that we
  adopted in Section~\ref{sec:frames}, which of the scope constellations in
  \LLast captures the meaning of \refx{ex:stove} correctly?
\end{exercise}

\section{Referential tense and perfective/imperfective aspect}
\label{sec:aspect}

\subsection{Referential tense after all}
\label{subsec:after-all}

Let us return to Partee's stove and the prospects of a ``referential'' theory of
tense. Our discussion of Partee's example (following Stalnaker and Ogihara) just
now came to the conclusion that we did need past tense to be an existential
quantifier over times, albeit a contextually restricted one. The stove-example
is interpreted as a claim about a particular contextually relevant interval. But
the speaker's claim is not merely that she didn't turn off the stove at that
interval. That in itself would be compatible with her turning off the stove at
some smaller interval inside the contextually relevant interval. The speaker's
claim is stronger: she did not turn off the stove at any time that is contained
in this interval. This is a negative existential claim. So there needs to be an
existential quantifier somewhere in the LF and below the scope of \emph{not},
and we concluded that past tense must be supplying it. But this conclusion is
not inescapable. Granted that there has to be an existential quantifier
somewhere \dash but couldn't it be somewhere else than in the meaning of tense?
One alternative that comes to mind is to locate it in the lexical meaning of the
verb (here \emph{turn off}). This means we abandon the lexical entry in \Next
and instead adopt the one in \NNext.

\ex $\svt{turn-off}^{i} = \lambda y. \lambda x. x\ \text{turns
  off}\ y\ \text{in}\ w_{i}\ \text{at}\ t_{i}$ \label{ex:turnoff-at}\xe

\ex~ $\svt{turn-off}^{i} = \lambda y. \lambda x. x\ \text{turns
  off}\ y\ \text{in}\ w_{i}\ \boxed{\text{in}}\ t_{i}$\label{ex:turnoff-in} \xe
%
The difference between `at' and `in' looks small at first, but if we reflect on
the meaning of `in', we see the hidden existential quantifier. When something
happens in an interval, it happens at some part of the interval. We can make
this more transparent in the metalanguage and rewrite \Last as \Next.

\ex
$\svt{turn-off}^{i} = \lambda y. \lambda x. \exists t \subseteq t_{i}\co x\
\text{turns off}\ y\ \text{in}\ w_{i}\ \text{at}\ t$ \label{ex:turnoff-ex}\xe
%
The subset sign here stands for the containment relation between time intervals.
A time interval can be defined as a certain kind of set of moments, as in \Next,
so the subset relation is well defined.

\ex A set of moments $S$ is an interval iff for any two moments that are in $S$,
every moment between them is also in $S$.\xe
%
Another way to clarify the distinction between `at' and `in' is to use the kind
of metalanguage that is familiar from the literature on Davidsonian event
semantics.

\pex abbreviations in ``event talk'':
\a      turn-off($e, x, y$)  =  $e$ is an event of turning off $y$ by agent $x$
\a      $\tau(e)$ = the (exact) time-interval occupied by event $e$\\
        also called the ``run-time'' or ``temporal trace'' of e
\xe

\ex~ event-talk formulation of \refx{ex:turnoff-at}, the old entry with 'at':\\
$\svt{turn-off}^{i} = \lambda y.\lambda x.\exists e [\text{turn-off}(e,x,y)\ \&\
e\ \text{is in } w_{i}\ \&\ \tau(e) = t_{i}]$
\xe

\ex~ event-talk formulation of \refx{ex:turnoff-in}, the new entry with 'in':\\
$\svt{turn-off}^{i} = \lambda y.\lambda x.\exists e [\text{turn-off}(e,x,y)\ \&\
e\ \text{is in } w_{i}\ \&\ \tau(e) \subseteq t_{i}]\label{ex:turnoff-ev}$
\xe
%
Let us spell out now how Partee's proposal for the meaning of past tense can be
upheld after all, once we assume the lexical semantics specified in
\refx{ex:turnoff-in}/\refx{ex:turnoff-ex}/\refx{ex:turnoff-ev}. The first task
here is to write new lexical entries for the tense morphemes, which encode
Partee's idea that tenses refer to specific time intervals and are semantically
and pragmatically akin to personal pronouns. We will defer the full execution of
this task until later and make do for the time being with a couple of
syncategorematic ad hoc rules for the interpretation of TPs.

\ex %
\note{This contextually salient time is also called the ``topic time''
  \parencite{klein-1994-time} or the ``reference time'' (a term which goes back
  to \cite{reichenbach-1947-elements}, but which has various other uses in the
  literature).}%
$\sv{\past\ \phi}^{i} = 1$ iff $\sv{\phi}^{\pair{w_{i},t'}} = 1$, where $t'$ is
the contextually salient\\ \hfill time before $t_{i}$ (no truth value defined if
there is no such time) \label{ex:past-syn}\xe

\ex~ $\sv{\text{woll}\ \phi}^{i} = 1$ iff $\sv{\phi}^{\pair{w_{i},t'}} = 1$,
where $t'$ is the contextually salient\\ \hfill time after $t_{i}$ (no truth
value defined if there is no such time) \xe
%
(\pres remains vacuous as before, i.e., $\sv{\pres\ \phi}^{i} = \sv{\phi}^{i}$.)

The stove example has the two potential LFs in \Next.

\pex
\a \past \textsc{neg} I turn off the stove.
\a \textsc{neg} \past I turn off the stove.
\xe
%
We can compute the truth conditions for both of these under the new semantics
for \past and \emph{turn off}, and it turns out that they are the same.

\ex $\svt{\Last[a]}^{i} = 1$ iff $\svt{\Last[b]}^{i} = 1$ iff\\
\hfill$\neg \exists e [\text{turn-off}(e,x,y)\ \&\ e\ \text{is
  in}\ w_{i}\ \&\ \tau(e) \subseteq t']$,\\ \hfill where $t'$ is the
contextually salient time before $t_{i}$\\ \hfill (no truth-value defined if
there is no such time) \xe
%
The fact that both scopal orders yield the same truth conditions is arguably a
point in favor of this approach. The English sentence is not in fact perceived
as ambiguous. Our earlier approach, on which past tense was a contextually
restricted existential quantifier, did not make this prediction \dash at least
not without the help of additional assumptions (such as a syntactic constraint
on the position of negation with respect to other heads on the clausal spine).
Now that the existential quantifier comes bundled with the lexical verb, its
scope is automatically ``frozen'' below everything that scopes over the verb.

\subsection{Event semantics and perfective aspect}
\label{sec:event-perfective}

\note{We share the goal of integrating Davidsonian event semantics and
  traditional intensional semantics with \cite{stechow-beck-2015-event-times},
  from whom we borrow a number of ideas.}%
Up to now, we have presupposed a pre-Davidsonian view of lexical meanings, on
which verbs take only individuals or propositions as their arguments. Even when
we recently inserted some event-talk into the metalanguage of our lexical
entries, we still defined the denotation of a verb like \emph{turn off} as a
function from two individuals to a truth value. In this section, we switch to a
Davidsonian treatment of verbs as predicates of events and integrate this with
our conception of sentence-intensions as temporal propositions.

\note{Many practitioners of event semantics assume that the event argument is
  the only real argument of the verb, whereas the subject, object, etc. are
  arguments of abstract theta-role heads that combine with the verb in the
  manner of modifiers. Here we remain agnostic on this matter. For concreteness,
  we assume that verbs take all the traditional arguments in addition to their
  event-argument, but the other view is equally compatible with everything we
  will say. We just abstract away from the internal compositional semantics of
  the VP.}%
In an extensional Davidsonian semantics, lexical entries look like \Next.

\ex $\svt{laugh} = \lambda x.\lambda e.\ e$ is an event of $x$ laughing\\
abbreviated: $\svt{laugh} = \lambda x.\lambda e.\ \text{laugh}(e,x)$
\label{ex:laugh}\xe
%
Assuming that events are not in $D_{e}$, but have their own basic type $\nu$,
VPs thus are of type $\type{\nu,t}$. (All the verb's non-event arguments are
merged inside the VP.)

In a semantics that is both Davidsonian and intensional, do we have to rewrite
these entries? For example, should we perhaps rewrite \Last as \Next?

\ex $\svt{laugh}^{i} = \lambda x.\lambda e.\ e$ is an event of $x$ laughing
$\boxed{\text{in}\ w_{i}\ \text{at}\ t_{i}}$\label{ex:laugh-i} \xe
%
That depends. Here we follow Kratzer and assume that each event occurs in just
one world and at just one time. %
\note{This assumption is made here mostly to keep things simple. It is not
  innocuous and not uncontroversial. See e.g. \cite{hacquard-2009-aspect-modal}
  for an analysis of root modals that makes crucial use of the idea that an
  actual event exists in non-actual worlds and has different properties there.}%
It is not possible for a given $e$ to be an event of $x$ laughing in one world
and to be some other kind of event in another world. Nor is it possible for one
and the same $e$ to be an event of $x$ laughing at one time and something else
at another time. %
\note{Strictly speaking, we should now write \Next, but since $i$ in \LLast does
  not occur on the right side of =, \LLast can be shorthand for \Next.

  \ex For any index $i$,
  $\svt{laugh}^{i} = \lambda x.\lambda e.\ \text{laugh}(e,x)$\xe}%
Reformulations such as \refx{ex:laugh-i} are uncalled for then, and we can
essentially stick with \refx{ex:laugh}.

But how then does world and time dependence enter the semantic computation? And
how can tenses and modal operators combine with VPs? VPs are now type $\type{\nu,t}$,
which leads to a type-mismatch if we try to combine them directly with a modal
operator or with a tense (regardless of whether the tense is a Priorian temporal
operator or a Partee-style referential tense). The way out of this problem is to
posit a more complex clause structure, with a further functional head that
intervenes between T (or M) and V. This is called an ``aspect'' head (category
label ``Asp''), and its semantic job is to existentially bind the event argument
of the VP and return a world- and time-sensitive denotation of type $t$.

\kwn\note{\Next combines the standard formal analysis of perfective aspect
  (among many others: \cite{klein-1994-time, kratzer-1998-analogies}) with the
  semantics of \cite{stechow-beck-2015-event-times}'s Modl head. It locates the
  event both in a time interval and in a possible world.}%
One instance of Asp is the so-called ``perfective'', for which we posit the
following entry.

\ex $\sv{\pfv}^{i} = \lambda P_{\type{\nu,t}}.\exists e [P(e) = 1\ \&\
\tau(e) \subseteq t_{i}\ \&\ e \le w_{i}]$\\
\hfill $\le\ :=$ is part of (= occurs in)\\
\hfill $\tau\ :=$ the run time of (temporal trace of)
\xe
%
\pfv is morphologically zero in English, so we can posit it in the LFs of
sentences with simple tensed verbs. Here is an example.

\pex
\a Barbara turned off the stove.
\a LF: \lb[TP] \past \lb[AspP] \pfv \lb[VP] Barbara turn-off the stove \rb \rb
\rb \xe
%
Using the syncategorematic rule \refx{ex:past-syn} for referential \past, our
entry \LLast for \pfv, and a Davidsonian entry for the verb, we compute the
following interpretation. (Do this as an exercise.)

\ex $\svt{\Last[b]}^{i} = 1$ iff $\exists e [\text{turn-off}(e,\ \text{B,
  the stove})\ \& \ \tau(e) \subseteq t'\ \&\ e \le w_{i}]$,\\
\hfill where $t'$ is the contextually salient time before $t_{i}$\\
\hfill (no truth-value defined if there is no such time)
\xe
%
This is the same meaning that we obtained in the previous section, when we had
built the existential quantification into the lexical meaning of the verb. What
used to be the meaning of VP is now the meaning of AspP. We have located the
event-quantifier in its own functional head, but otherwise it is the same
analysis.

\begin{exercise}
  What about the negated sentence that was Partee's original example? Where can
  we now generate negation in an interpretable LF? Does the current analysis
  still predict that the sentence is not in fact ambiguous? \qed
\end{exercise}

\subsection{The English progressive}
\label{sec:progressive}

\note{See e.g. \cite{arregui-rivero-salanova-2014-imperfectivity} for a recent
  approach to cross-linguistic semantic variation in imperfective aspects.}%
\note{We are not serious about morphology here. The meaning may well be carried
  by an abstract head and the \emph{be} a vacuous element.}%
Besides perfective aspect, there is imperfective aspect \dash or more
accurately, there is probably a family of imperfective aspects in different
languages that have some shared and some non-shared properties. English has an
imperfective aspect known as the ``progressive'', with an overt morphology that
consists of a copula which governs a present-participial form of the VP
(\emph{V-ing}). We posit a functional head be-\prog as the aspect head in the
English progressive construction.

The basic intuition behind much work on the perfective-imperfective distinction
is that, whereas perfective aspect locates an event within the evaluation time,
imperfective aspect does the reverse, i.e., it places the evaluation time within
the event time. If we formalize this intuition directly, without introducing any
further differences from the perfective, we come up with \Next.

\clearpage
\ex\note{This is similar to the first formal analysis of the progressive, due to
  \cite{bennett-partee-1978-tense-aspect}. They did not work in an event
  semantics, however. Also, their semantics required $t_{i}$ to be a
  \emph{non-final} subinterval of $\tau(e)$, rather than merely
  $t_{i} \subseteq \tau(e)$. This requirement seemed too strong in light of
  examples such as \cite{dowty-1977-progressive}'s \emph{John was watching TV
    when he fell asleep} (which does not say that TV-watching continued beyond
  the point of falling asleep). However, as Dowty showed, it turned out to be
  the right requirement in the context of the modalized analysis that he
  proposed, see below.}%
First attempt:\label{ex:prog-simple}\\
$\svt{be-\prog}^{i} = \lambda P_{\type{\nu,t}}. \exists e [P(e) = 1\ \& \
\boxed{t_{i} \subseteq \tau(e)}\ \&\ e \le w_{i}]$
\xe
%
It is well-known, however, that there is also a difference in how the event is
related to the evaluation world. While perfective places the event within the
actual world, the progressive permits it to be partly in another world, so to
speak. This point, which was at the center of \cite{dowty-1977-progressive}'s
seminal work on the progressive, is brought home by examples like \Next.

\ex \note{Dowty dubbed this the ``imperfective paradox'', although it's not
  really a paradox, just a counterexample to a certain analysis that looked
  plausible at first.}%
John was going to the store when he ran into Svenja.\label{ex:interrupt} \xe
%
We can't infer from this sentence that John actually made it to the store, or
will ever make it there. The sentence leaves this open. Perhaps John does
complete his trip to the store after the encounter, and perhaps he doesn't. The
truth-conditions of the sentence \Last are compatible with either scenario. The
entry in \LLast, on the other hand, would require that there be a John-going-
to-the-store event which occupies a super-interval of the time of the encounter
with Svenja and which \emph{occurs in the actual world}. So \LLast can't be quite
right.

Dowty's analysis of the progressive says instead that a John-going-to-the-store
event occurs in certain \emph{possible} worlds. These possible worlds are
related to the actual world in a particular way: they are worlds which share a
history with the actual world up to a certain point and then develop (possibly
counterfactually) in such a way that no events that were already in progress get
interrupted (``inertia worlds''). The idea is, very roughly, that the sentence
tells us: either John actually went to the store, or if he didn't, then at least
he \emph{would have gone} there if he hadn't been interrupted. %
\note{See among many others, \cite{landman-1992-progressive,
    portner-1998-progressive}.}%
Since the publication of Dowty's paper, there has been a succession of
sophisticated counterexamples and refinements to his original proposal, but this
is beyond the scope of this introduction. Here is a version based on Dowty.

\ex\note{Apart from introducing quantification over other worlds, \Next also
  differs from \LLast in that it strengthens the requirement on the temporal
  relation between $t_{i}$ and $\tau(e)$: not only must $\tau(e)$ contain all of
  $t_{i}$, but it must moreover extend into the time after $t_{i}$. This is
  intended to capture the intuition that e.g. \Last is not appropriate if John
  already reaches the store during his encounter with Svenja; see
  \cite{dowty-1977-progressive} for discussion.}%
second attempt (and final version for us):\label{ex:prog-dowty}
\begin{multline*}
  \svt{be-\prog}^{i} = \lambda P_{\type{\nu,t}}. \forall w \bigl[w \in
\text{Inert}(i) \rightarrow \exists e [P(e) = 1\ \& \\
t_{i} \subset^{<} \tau(e)\ \&\ e \le w]\bigr]
\end{multline*}
where $\subset^{<}$ abbreviates: ``is a non-final subinterval of''\\
(that is: $\tau(e)$ includes every moment in $t_{i}$ as well as some moment
after the end of $t_{i}$) \xe
\ex~ Definition: $w \in \text{Inert}(i)$  iff\\
$w$ is exactly like $w_{i}$ up to the end of $t_{i}$ and then develops in such a
way that no events are interrupted. \xe
%
We will see in a minute that there is a class of VPs for which the
truth-conditions predicted by \LLast come very close to those predicted by the
simpler \refx{ex:prog-simple}. But examples like \refx{ex:interrupt} show that
this must not always hold.

Let's do a simple example.

\pex\label{ex:laughing}
\a Sari is laughing.
\a LF: \lb[TP] \pres \lb[AspP] be-\prog \lb[VP] Sari laugh \rb \rb \rb
\a $\svt{\refx[b]{ex:laughing}}^{i} = 1$ iff\\
$\forall w \bigl[ w \in \text{Inert}(i) \rightarrow \exists e [t_{i} \subset^{<}
\tau(e)\ \&\ e \le w\ \&\ \text{laugh}(e,\text{Sari})]\bigr]$
\xe
%
Just as it stands, \Last[c] does not logically entail that any laughing happens
in the world $w_{i}$ (i.e., in the utterance world $w_{u}$ if this is an
unembedded assertion). It only talks about the inertia worlds. However, there is
a property of the lexical meaning of \emph{laugh} that permits us to draw
further inferences. Laughing events are made up of lots of sub-events which
themselves are laughing events, down to very little ones that don't last much
more than an instant. Given this, consider a world in Inert$(w_{u},t_{u})$, say
$w$. If \Last[b] is true in $w_{u}$ at $t_{u}$ , it follows that $w$ contains an
event of Sari laughing whose run-time includes $t_{u}$. Among the subevents of
this event, which themselves are events of Sari laughing, there will most likely
be one that is early enough and small enough to have transpired by the end of
$t_{u}$. %
\note{The only condition under which this would not hold is if the laughing
  starts right at the beginning of $t_{u}$ and $t_{u}$ itself is too short to
  fit even a minimal laughing event. This would have to be a very short
  utterance time, shorter than it realistically takes to say \emph{Sari laughs},
  so we disregard this possibility. But we will later see a problem with this.}%
And since up to the end of $t_{u}$, the histories of $w$ and $w_{u}$ are
identical, this small Sari-laughing event in $w$ must have a perfectly matching
counterpart in $w_{u}$. That's why we infer from \Last[a] that there is actual
laughing at the utterance time.

This is the kind of example for which \refx{ex:prog-dowty} and the simpler entry
\refx{ex:prog-simple} predict almost identical truth conditions.
\refx{ex:prog-dowty} demands something slightly stronger, namely that moreover
the laughing continues at least a little bit beyond the utterance time unless it
is interrupted (which means it \emph{would} have continued). So they are not
quite equivalent, but the difference is very subtle.

Importantly, however, this almost-equivalence depends on the particular property
of the meaning of the VP that we just exploited in our reasoning. Had the VP
been \emph{Sari go to the store}, it would have been a very different matter.
Events of Sari going to the store are \emph{not} made up of lots of smaller
events which each are events of Sari going to store. They are made up of smaller
events which are events of Sari going \emph{towards} the store, but since most
of these don't end with Sari at the store, they are not events of Sari going
\emph{to} the store. So if we are told that every $w \in$ Inert$(i)$ contains an
event of Sari going to the store which occupies a super-interval of $t_{i}$, we
cannot infer that Sari goes to the store in $w_{i}$. We can merely infer that
$w_{i}$ contains an event that is indistinguishable from those parts of the
inertia-worldly trips-to-the-store which fall \emph{before the end of} $t_{i}$.
In other words, we infer that $w_{i}$ contains the \emph{beginning} of a
Sari-go-to-the-store event, but not necessarily anything more.

The attentive reader may have wondered why we used a past tense example to
illustrate the perfective in the previous section, but a present tense example
for the progressive in the current section. Indeed, it is incumbent upon us to
examine what the theory predicts for every possible combination of a tense and
an aspect.

\subsection{Stativity effects}
\label{sec:stativity}

\note{The examples in \Next do not have prominent habitual (generic) readings.
  Ignore such readings if you can get them anyway. The \# judgments apply to an
  intended episodic reading (describing a single event).

  Simple present tense on a non-stative verb is systematically grammatical when
  the sentence has a generic or habitual interpretation, or when it describes
  the content of a plan or schedule. We don't worry about these cases here,
  since they very plausibly involve a covert modal operator of some kind that
  applies to the VP before any tense or (higher) aspect. (See for example,
  \cite{copley-2008-plan} or \cite{thomas-2014-tense-mbya}.) That modal operator
  may itself be an aspect head, or it may create a bigger VP which is a
  predicate of states. In the latter case, whatever explains the acceptability
  of stative VPs under present tense will also explain the acceptability of
  present tense generics/habituals. See below. Some kind of modal analysis might
  also work for some of the cases in the list, like stage directions and plot
  summaries, but less plausibly to e.g. the sportscaster style or the historical
  present.}%
It is well known that non-stative predicates in the simple present tense have a
limited range of felicitous uses. Sentences such as those in \Next are
spontaneously judged as odd by speakers of English.

\pex\label{ex:statives}
\a \#Sari laughs.
\a \#Sari wakes up.
\a \#Sari goes to the store.
\xe
%
Let us see what our theory predicts. We see no progressive morphology, but there
has to be an aspect head for the sentence to express a proposition, so the
aspect must be \pfv. With present tense semantically vacuous, we then have LFs
and predicted meanings like \Next for \Last[c].

\ex LF: \lb[TP] \pres \lb[AspP] \pfv \lb[VP] Sari go to the store \rb \rb  \rb\\
true at $i$ iff $\exists e [\tau(e)\subseteq t_{i}\ \&\ e\le w_{i}\ \&\
\text{go-to}(e,\text{Sari, the store})]$
\xe
%
This says that if \emph{Sari goes to the store} is asserted in $w_{u}$ at
$t_{u}$, the assertion is true iff there is a Sari-go-to-the store event in
$w_{u}$ whose run-time is contained within $t_{u}$. This is a somewhat
implausible scenario, given that trips to the store typically take longer than
the production of such a short sentence. One may be tempted to attribute the
strangeness of \LLast[c] to this fact. But upon reflection, that doesn't look
like the right explanation. We can set up a scenario that eliminates the
implausibility. Imagine Sari was already very close to the store, and/or she is
on a very fast vehicle \dots. The judgment about \LLast[c] is not really
affected by such manipulations, but we would expect it to be if pragmatic
plausibility were all that mattered. And the pragmatic explanation looks even
less convincing when we consider the other examples in \LLast. Waking-up events
are very short, if not instantaneous, so such events should have no problem
fitting inside the utterance time and \LLast[b] should be just fine. As regards
\LLast[a], we have already said that longer laughing events are made up of
shorter laughing events. So if Sari laughs for any duration that overlaps with
the utterance time, there is probably a laughing event within the utterance
time, and \LLast[a] should be fine as well.

\note{There are few formal semantic analyses of the historical present. An
  exception is \cite{zucchi-2005-present}.}%
Friends of pragmatic approaches like to remind us that the examples in \LLast
are not ungrammatical. Sentences of this sort are acceptable in a variety of
special contexts or registers, such as play-by-play sportscasting, the
historical or narrative present, newspaper headlines, stage directions, plot
summaries, explicit performatives, \dots, to name some. It is appealing to say
that the essence of (at least some of) these special uses is a pretense that the
utterance time is something other than what it is, a pretense that one is
speaking at a time closer to the events being reported, at a
pretend-utterance-time that is earlier and/or longer than one's actual
utterance. This may or may not be right. At any rate, it does not directly
address the question why \LLast[a-c] are unacceptable \emph{outside} of these
special registers or contexts. %
\note{Cf. \cite{bennett-partee-1978-tense-aspect}: ``We regard a speech act as
  occurring at a moment of time and understand the assertion as being true at
  that moment. Accordingly, we are inclined to only use the reportive simple
  present when the act being described seems to be almost instantaneous and to
  be occurring at the moment of utterance.'' See also \cite{dowty-1979-verbs}.}%
One seems to need a concomitant assumption that the ``ordinary'' register
involves a different pretense, namely that the utterance time is \emph{shorter}
than it actually is, in fact, that it is a mere instant in the technical sense
(a singleton of one moment), and hence too short to contain even a getting-up
event or a minimal laughing-event.

\note{As Milo Philipps-Brown (pc) pointed out, one worry about this assumption
  is that it undermines our earlier reasoning about the progressive \emph{Sari
    is laughing}. There we attributed the intuition that this sentence entails
  the existence of laughter in $w_{u}$ to the fact that $t_{u}$ was long enough
  to contain a minimal laugh. It is not clear how to resolve this tension.
  Perhaps we can get out of it by convincing ourselves that we judge the
  utterance true, after all, if all that actually happens before the
  interruption is an instant sized beginning of a minimal laugh.}%
For the sake of the argument, let's see how it may help to stipulate that
$t_{u}$ is always an instant. To get the desired mileage out of this assumption,
we must also sharpen some specifics regarding the lexical meanings of verbs. %
\note{These ideas are common in the literature and go back at least to
  \cite{taylor-1977-continuity}. See \cite{filip-2012-lexical-aspect} for a
  recent and comprehensive survey.}%
These assumptions are not uncontroversial, but widely accepted in the
literature: None of the VPs in \LLast describe events that can possibly have
run-times that are instants. Any VP that entails a change of state \dash whether
it is a change that takes time (like getting from some place else to the store)
or a virtually ``instantaneous'' change (like from asleep to awake) \dash
because of that applies only to events whose run-time contains at least two
moments (one at which the previous state holds and one at which the result state
holds). Likewise, any VP that describes an activity or movement or other
happening of some sort (like laughing, oscillating, raining, even sleeping)
describes events that may have very short run times but never just a single
moment. These assumptions about lexical semantics make the predicted
truth-conditions for clauses with \pfv (as computed in \Last) impossible to
satisfy unless $t_{i}$ is a proper interval, i.e., not a singleton.

From this perspective there is a straightforward account of what makes stative
predicates different. Once we change the VPs in \LLast to stative ones, the
simple present tense becomes perfectly fine (in every register).

\pex
\a Sari is tired.
\a Sari is at home.
\a Sari owns a factory.
\xe

Suppose the distinguishing semantic feature is precisely that predicates like
\emph{tired}, \emph{at home}, and \emph{own a factory} describe eventualities
(``states'') whose run-times can be instants. A state of Sari being tired may be
long or short, but it is necessarily made up of shorter and shorter sub-states
which are also states of Sari being tired. And not only that \dash it is even
made up of such sub-states that occupy a single instant. The latter makes
\emph{tired} different from \emph{laugh} or even \emph{move}, which apply to
eventualities whose run-times may be infinitesimally short but are still always
proper intervals. What does this buy us? It lets us say that the sentences in
\Last have the exact same parses as those in \refx{ex:statives}, with a
perfective aspect head, and yet they have truth-conditions which can be
satisfied by an instant.

\ex LF: \pres [ \pfv [ Sari \emph{be} tired ]]\\
meaning: true at $i$ iff $\exists e [\tau(e) \subseteq t\ \& e \le w_{i}\ \&\
\text{tired}(e, \text{Sari})]$\\
lexical entry: $\svt{tired} = \lambda x. \lambda e.\ e$ is a state of $x$ being
tired
\xe
%
(Type $\nu$ must be understood in such a way that $D_{\nu}$ includes states in
addition to ``events'' in a narrow sense. \cite{bach-1986-algebra} coined the
term ``eventuality'' for this broader sense of ``event''.)

\note{It is often said in this context that progressives pattern with statives
  in the present tense because progressive VPs \emph{are} stative. This is not
  literally true on our analysis, because \emph{be}-\prog is an aspect head and
  AspPs are not predicates of states (or of eventualities of any kind). One
  might, however, entertain a different analysis on which (at least some of) the
  operators we are used to calling ``aspects'' have meanings of type
  $\type{\nu t, \nu t}$. (Another head higher in the structure would then have
  to be responsible for binding the state argument and introducing the world and
  time.)}%
So, together with the stipulation that the utterance time is treated as an
instant, this approach to the stative/non-stative distinction provides an
explanation for why stativity is required in the simple present tense. We can
also reassure ourselves that present progressives are still expected to be
uniformly good even if $t_{u}$ must be an instant. This is because
\emph{be}-\prog places the event in a super-interval of $t_{i}$.

Whether or not the assumption that $t_{u}$ is an instant can ultimately be defended,
it is important to be aware that the stativity effect we witness in present
tense matrix clauses is replicated perfectly in certain environments which are
neither matrix nor (morphologically) present. These environments include the
complements of epistemic necessity modals and the infinitival complements of
verbs like \emph{believe} and \emph{claim}.

\pex\label{ex:must}\a\note{The judgment in (a) presupposes an intended episodic
  reading for the VP. To the extent that the VP can be read habitually, the
  epistemic reading becomes available. The point here is that the judgments for
  the constructions in \refx{ex:must}--\refx{ex:believe} are parallel to the
  judgments for the same VPs in the simple present tense.}%
Sari must sleep/go to the store.\\
deontic reading only
\a Sari must be at home/ be sleeping.\\
epistemic reading okay \xe

\ex~ Sari claimed to *work/*go to work/be at work/be going to work.\xe

\ex~ Sari believed Svenja to *sleep/*go to the store/be at home/be sleeping.
\label{ex:believe}\xe
%
We will return to this observation in a later section, in connection with the
discussion of so-called ``Sequence of Tense''.

\section{Formalizing the referential analysis}
\label{sec:formalizing-referential}

Above we stated Partee's proposal as follows:

\ex[exno=\ref{ex:past-syn}]%
$\sv{\past\ \phi}^{i} = 1$ iff $\sv{\phi}^{\pair{w_{i},t'}} = 1$, where $t'$
is the contextually salient\\ \hfill time before $t_{i}$ (no truth value defined
if there is no such time)
\xe
%
Apart from not being fully compositional, this is a bit vague for us to work
with when (in the next chapter) we consider complex sentences with several
occurrences of past tense. Let us therefore make it a little more precise.

Partee suggested that past tense was analogous to a pronoun like \emph{he}. We
are used to representing pronouns as variables (see e.g. Heim \& Kratzer), so
Partee-style tenses too should then have denotations that are sensitive to a
variable assignment. So let's make two assumptions: First, each occurrence of
\past and \emph{woll} at LF must carry a numerical subscript, like a pronoun.
Second, we add a new type $i$ (for ``intervals'') and we assume that variable
assignments can assign elements of $D_{i}$ to object-language variables
(numerical indices). Notice that this new type $i$ is not the same as our
existing type $s$ (world-interval-pairs), though the second member of an element
of $D_{s}$ is always a member of $D_{i}$.

We now write the following entries. (\pres remains vacuous for now and therefore
needs no entry.)

\ex $\sv{\past_{n}}^{i,g} = \lambda p\in D_{st}\co g(n) \in D_{i}\
\&\ g(n) < t_{i}.\ p(w_{i},g(n)) = 1$\label{ex:formal-past}\xe

\ex~ $\sv{\text{woll}_{n}}^{i,g} = \lambda p\in D_{st}\co g(n) \in D_{i}\
\&\ g(n) > t_{i}.\ p(w_{i},g(n)) = 1$\label{ex:formal-woll}\xe
%
Let's illustrate how this works in a simple example.

\ex Barbara turned off the stove.\\
new LF:  $\past_{7}$ [ \pfv [ Barbara turn off the stove ]]\xe
\ex~ $\svt{\past$_{7}$ [ \pfv [ Barbara turn off the stove ]]}^{i,g}$
is defined\\
\hfill iff $7 \in \text{dom}(g)\ \& \ g(7)\in D_{i}\ \& \ g(7) < t_{i}$\\
when defined,
$\svt{\past$_{7}$ [ \pfv [ Barbara turn off the stove ]]}^{i,g} = 1$\\
\hfill iff
$\exists e [\tau(e) \subseteq g(7)\ \&\ e \le w_{i}\ \& \
\text{turn-off}(e,\text{Barbara, the stove})]$
\xe
%
If this is a matrix sentence, we evaluate it with respect to the utterance world
and time, and we also rely on the utterance context to furnish a suitable
assignment (call it $g_{u}$), which maps the free variable $7$ to a time
interval that precedes $t_{u}$. Intuitively, this is the salient past interval
that the speaker has in mind when making this past-tense claim (also called the
``topic time''). So we have:

\ex An utterance $u$ of the LF ``\past$_{7}$[\pfv [Barbara turn off the
stove]]'' is felicitous only if\\
\hfill $g_{u}$ is such that
$7\in \text{dom}(g_{u})\ \&\ g_{u}(7)\in D_{i}\ \&\ g_{u}(7) < t_{u}$,\\
and it is moreover true iff\\
\hfill
$\exists e [\tau(e) \subseteq g(7)\ \&\ e \le w_{i}\ \& \ \text{turn-off}(e,\text{Barbara,
  the stove})]$ \xe

It will also be useful to clarify how frame adverbials might be treated in a
Partee-style approach to tense. A natural idea here is that a frame adverb
contributes a further presupposition about the intended topic time. Recall our
treatment of frame adverbs as having extensions of type $t$.

\ex[exno=\ref{ex:on-part}]
$\svt{on February 1, 2001}^{i} = 1$ iff $t_{i}$ is part of February 1, 2001
\xe
%
To make room for this in the LF of a sentence with a Partee-style tense, we want
to revise the entries from \refx{ex:formal-past} and \refx{ex:formal-woll} and
give \past and \emph{woll} a second argument.

\ex $\sv{\past_{n}}^{i,g} = \lambda p\in D_{st}\co g(n) \in D_{i}\
\&\ g(n) < t_{i}\ \& \ p(w_{i},g(n)) = 1.$\\
\hfill $\lambda q_{st}. q(w_{i},g(n)) = 1$\xe
%
\kwn For example:

\ex Barbara turned off the stove on February 1, 2001.\\
LF: [ \past$_{7}$ on February 1, 2001 [ \pfv [ Barbara turn off the stove]]]\\
uttered felicitously only if \\
\hfill
$7\in \text{dom}(g_{u})\ \&\ g_{u}(7)\in D_{i}\ \&\ g_{u}(7) < t_{u}\ \&\ g_{u}(7) \subseteq$
Feb 1, 2001,\\
and uttered truly iff $\exists e [\tau(e) \subseteq g(7)\ \&\ e \le w_{u}\ \& \
\text{turn-off}(e,\text{Barbara, the stove})]$
\xe
%
Notice that there is still some room for context-dependency, in that the speaker
may be referring to either the whole of Feb 1st or to a proper part of it (e.g.
the morning of that day). But the role of context is greatly reduced by the
contribution of the adverb.

The revised, 2-place, entry for the future is analogous. When we have frame
adverbs with present tense, we also need a non-vacuous semantics for \pres, but
this is no different in the Partee-approach than it was in the Priorian
approach.

To conclude, let us highlight how the Partee-style, ``referential'', analysis of
tenses differs from the Prior-as-modified-by-Stalnaker-style analysis, and also
what they have in common. The essential difference is that Partee-style
$\past_{n}$ and \emph{woll}$_{n}$ do not express existential quantification over
times, but instead rely on a contextually furnished variable assignment to
supply a particular time. In the Partee-approach, the denotations of past and
future clauses are always context-dependent; in the modified-Prior approach,
they only are if there happens be a silent restrictor together with the
existential quantifier. Both approaches assume that the extensions of
$\past_{(n)}$ and \emph{woll}$_{(n)}$ are sensitive to the evaluation time, and
both assume that these items \emph{shift} the evaluation time for their
complements. The two approaches also agree on the treatment of \pres, which,
according to both, does not shift evaluation time and makes no semantic
contribution other than a presupposition when there is a frame adverb.

%%% Local Variables:
%%% mode: latex
%%% TeX-master: "ik-book"
%%% End:
