%!TEX root = IntensionalSemantics.tex
\chapter{Basics of Tense and Aspect}\label{cha:tense} % (fold)

\chapterprecishere{We explore an analysis of tense that treats tenses as intensional operators manipulating a time parameter of evaluation. The treatment is formally quite parallel to the treatment of modals in Chapter~\ref{cha:modality}. We touch on many basic questions about tense and aspect, without exploring them fully.}

\minitoc

\section{A First Proposal for Tense}

Tense logic\marginnote{See the Stanford Encyclopedia of Philosophy entry on temporal logic: \url{http://plato.stanford.edu/entries/logic-temporal/} and the website for Prior studies: \url{http://www.prior.aau.dk/index2.htm}.}, or temporal logic, is a branch of logic first developed by the aptly named Arthur Prior in a series of works, in which he proposed treating tense in a way that is formally quite parallel to the treatment of modality discussed in Chapter \ref{cha:modality}. Since tense logic (and modal logic) typically is formulated at a high level of abstraction regarding the structure of sentences, it doesn't concern itself with the internal make-up of ``atomic'' sentences and thus treats tenses as sentential operators (again, in parallel to the way modal operators are typically treated in modal logic). We will implement a version of Prior's tense logic in our framework.

The first step is to switch to a version of our intensional semantic system where instead of a world parameter, the evaluation function is sensitive to a time parameter (and a variable assignment). Eventually, we will want to deal with the full complexity and relativize the evaluation function to both worlds and times, but for now, we will just relativize to times. The composition principles developed in Chapter~\ref{cha:beginnings_of_intensional_semantics} will be adopted \emph{mutatis mutandis}. Predicates will now have lexical entries that incorporate their sensitivity to time:

\ex. $\sv{\mbox{tired}}^{t,g} = \lambda x \in D.\ x \mbox{ is tired at } t$.\footnote{We remain vague for now about what we mean by ``times'' (points in time? time intervals?). This will need clarification, We will also see the need to clarify what we mean by ``at'' in the metalanguage in this entry and others.}

It is customary in the literature to introduce a new basic type for times; for now, we will recycle the designation $s$ as the type for times. Then, for example, the intension of sentence will again be of type \type{s,t}, but now that would be a temporal proposition, a function from times to truth-values.

In this framework, we can now formulate a very simple-minded first analysis of the present and past tenses and the future auxiliary \emph{will}. As for (LF) syntax let's assume that (complete matrix) sentences are TPs, headed by T (for ``tense''). There are two morphemes of the functional category T, namely \emph{PAST} (past tense) and \emph{PRES} (present tense). The complement of T is an MP or a VP. MP is headed by M (for ``modal''). Morphemes of the category M include the modal auxiliaries \emph{must}, \emph{can}, etc., which we talked about in previous chapters, the semantically vacuous \emph{do} (in so-called ``do-support'' structures), and the future auxiliary \emph{will}. Evidently, this is a semantically heterogeneous category, grouped together solely because of their common syntax (they are all in complementary distribution with each other). The complement of M is a VP. When the sentence contains none of the items in the category M, we assume that MP isn't projected at all; the complement of T is just a VP in this case. We thus have LF-structures like the following. (The corresponding surface sentences are given below, and we won't be explicit about the derivational relation between these and the LFs. Assume your favorite theories of syntax and morphology here.)

\exi.  \label{pres}[TP Mary [T' PRES [VP t [V' be tired ]]]] \\= Mary is tired.

\exi.  \label{past}[TP Mary [T' PAST [VP t [V' be tired ]]]] \\= Mary was tired.

\exi.  [TP Mary [T' PRES [MP t [M' woll [VP t [V' be tired ]]]]]] \\= Mary will be tired.

When we have proper name subjects, we will pretend for simplicity that they are reconstructed somehow into their VP-internal base position. (We will talk more about reconstruction later on.)

What are the meanings of \emph{PRES}, \emph{PAST}, and \emph{will}? For \emph{PRES}, the simplest assumption is actually that it is semantically vacuous. This means that the interpretation of the LF in \ref{pres} is identical to the interpretation of the bare VP \emph{Mary be tired}:

\ex. For any time $t$:\\
$\sv{\mbox{PRES (Mary be tired)}}^{t} = \sv{\mbox{Mary be tired}}^{t} = 1 $
iff Mary is tired at $t$.

Does this adequately capture the intuitive truth-conditions of the sentence \emph{Mary is tired}? It does if we make the following general assumption:

\ex. An utterance of a sentence (= LF) \ensuremath{\phi} that is made 
at a time $t$ counts as true iff $\sv{\ensuremath{\phi}}^{t} = 1$ (and as false 
if $\sv{\ensuremath{\phi}}^{t} =0$).

This assumption ensures that (unembedded) sentences are, in effect, interpreted as claims about the time at which they are uttered (``utterance time'' or ``speech time''). If we make this assumption and we stick to the lexical entries we have adopted, then we are driven to conclude that the present tense has no semantic job to do. A tenseless VP \emph{Mary be tired} would in principle be just as good as \ref{pres} to express the assertion that Mary is tired at the utterance time. Apparently it is just not well-formed as an unembedded structure, but this fact must be attributed to principles of syntax rather than semantics.

What about \emph{PAST}? When a sentence like \ref{past} \emph{Mary was tired} is uttered at a time $t$, then what are the conditions under which this utterance is judged to be true? A quick (and perhaps ultimately wrong) answer is: an utterance of \ref{past} at $t$ is true iff there is some time before $t$ at which Mary is tired. This suggests the following entry:

\ex. For any time $t$:\\
$\sv{\mbox{PAST}}^{t} = \ensuremath{\lambda}p\ensuremath{\in}D_{\type{s,t}}. \ensuremath{\exists}t'\mbox{ before }t: p(t') = 1$\label{ex:sempast}

So, the past tense seems to be an existential quantifier over times, restricted to times before the utterance time.

For \emph{will}, we can say something completely analogous:

\ex. For any time $t$:\\
$\sv{\mbox{will}}^{t} = \ensuremath{\lambda}p\ensuremath{\in}D_{\type{s,t}}. \ensuremath{\exists}t'\mbox{ after }t: p(t') = 1$\label{ex:semfut}

Apparently, \emph{PAST} and \emph{will} are semantically alike, even mirror images of each other, though they are of different syntactic categories. The fact that \emph{PAST} is the topmost head in its sentence, while \emph{will} appears below \emph{PRES}, is due to the fact that syntax happens to require a T-node in every complete sentence. Semantically, this has no effect, since \emph{PRES} is vacuous.

Both \ref{ex:sempast} and \ref{ex:semfut} presuppose that the set $T$ comes with an intrinsic order. For concreteness, assume that the relation `precedes' (in symbols: $<$) is a strict linear order on $T$.\footnote{Definition: A relation $R$ is a strict linear order on a set $S$ iff it has the following four properties:

\vspace{-6pt}\begin{enumerate}[(i)]
	\item $\ensuremath{\forall}x\ensuremath{\forall}y\ensuremath{\forall}z ((Rxy \& Ryz) \ensuremath{\rightarrow} Rxz)$ ``Transitivity"
	\item $\ensuremath{\forall}x (\ensuremath{\neg}Rxx)$ ``Irreflexivity"
	\item $\ensuremath{\forall}x\ensuremath{\forall}y (Rxy \ensuremath{\rightarrow} \ensuremath{\neg}Ryx)$ ``Asymmetry", and
	\item $\ensuremath{\forall}x\ensuremath{\forall}y (x \ensuremath{\neq} y \ensuremath{\rightarrow} (Rxy \ensuremath{\vee} Ryx))$ ``Connectedness"
\end{enumerate}
} The relation `follows', of course, can be defined in terms of `precedes' ($t$ follows $t'$ iff $t'$ precedes $t$).

There are many things wrong with this simple analysis. We will not have time here to diagnose most of the problems, much less correct them. But let's see a couple of things that work out OK and let's keep problems and remedies for later.

\subsection{\emph{former}}

There is a brief discussion on p. 72 of H\amp\ K about the inadequacy of an extensional semantics for the adjective \emph{former} as in 

\ex. John is a former teacher.

We can now write a semantics for \emph{former}. While there are a bunch of people who are currently teachers, there are others that aren't now teachers but were at some previous time. The latter are the ones that the predicate \emph{former teacher} should be true of, In other words, \emph{former teacher} is a predicate that is true of individuals just in case the predicate \emph{teacher} was true of them at some previous time (and is not true of them now). So, \emph{former} needs to be an intensional operator that ``displaces'' the evaluation of time of its complement from ``now'' to some previous time. To be able to do that, it needs to take the intension of its complement as its argument. This suggests the following lexical entry:

\ex. $\sv{\mbox{former}} = \lambda f \in D_{\type{s,\type{e,t}}}. \lambda x. [f(t)(x)=0\ \&\ \exists t' \mbox{ before } t: f(t')(x) = 1].$

\begin{exercise}

H\amp\ K on p.72 mention the adjective \emph{alleged} in one breath with \emph{former}. Formulate a lexical entry for \emph{alleged} as used in \emph{John is an alleged murderer.} [This will use our original intensional system with a world parameter] \eex
\end{exercise}

\subsection{Some Time Adverbials}

At least to a certain extent, we can also provide a treatment of temporal adverbials such as:

\ex. Mary was tired on February 1, 2001.

The basic idea would be that phrases like \emph{on February 1, 2001} are propositional modifiers. Propositions are the intensions of sentences. At this point, propositions are functions from times to truth-values. Propositional modifiers take a proposition and return a proposition with the addition of a further condition on the time argument.

\ex. $\sv{\emph{on February 1, 2001}}^{t} \\ = \ensuremath{\lambda}p\ensuremath{\in}D_{\type{s,t}}$. 
[ p(t) = 1 \& t is part of Feb 1, 2001 ]

\exi.  \label{lfpp1}[T' PAST [VP [VP Mary [V' be tired]] [PP on February 1, 2001]]]

An alternative would be to treat \emph{on February 1, 2001} as a ``sentence'' by itself, whose intension then would be a proposition.

\ex. $\sv{\emph{on February 1, 2001}}^{t} = 1$ iff $t$ is part of February 1, 2001

\ex. $\sv{\emph{on}}^{t} = \ensuremath{\lambda}x.\ t$ is part of $x$

To make this work, we would then have to devise a way of combining two tenseless sentences (\emph{Mary be tired} and \emph{on February 1, 2001}) into one. We could do this by positing a silent \emph{and} or by introducing a new composition rule (``Propositional Modification''?).

Let's not spend time on such a project.

\begin{exercise}

Imagine that \emph{Mary was tired on February 1, 2001} is not given the LF in \ref{lfpp1} but this one:

\exi. [T' [T' PAST [VP Mary [V' be tired]]] [PP on February 1, 2001]]

What would the truth-conditions of this LF be? Does this result correspond at all to a possible reading of this sentence (or any other analogous sentence)? If not, how could we prevent such an LF from being produced?\eex
\end{exercise}

\begin{exercise}

When a quantifier appears in a tensed sentence, we might expect 
two scope construals. Consider a sentence like this:

\ex. Every professor (in the department) was a teenager in the Sixties.

We can imagine two LFs:

\ex.  PAST [ [every professor be a teenager] [in the sixties] ]

\ex.  [every professor] $\lambda 2$ [ PAST [ [t$_2$ be a teenager] [in the sixties] ]

Describe the different truth-conditions which our system assigns to the two LFs.

Is the sentence ambiguous in this way?

If not this sentence, are there analogous sentences that do have the ambiguity?\eex
\end{exercise}

\begin{exercise}

The following entry for \emph{every} makes it a time-\underline{in}sensitive item:

\ex. $\sv{\emph{every}}^{t} = \ensuremath{\lambda}f\ensuremath{\in}D_{\type{e,t}}. \ensuremath{\lambda}g\ensuremath{\in}D_{\type{e,t}}.\ensuremath{\forall}x 
[ f(x) = 1 \ensuremath{\rightarrow} g(x) = 1 ]$

Consider now two possible variants (we have underlined the portion where they differ):

\ex. $\sv{\emph{every}}^{t} = \ensuremath{\lambda}f\ensuremath{\in}D_{\type{e,t}}. \ensuremath{\lambda}g\ensuremath{\in}D_{\type{e,t}}.\ensuremath{\forall}x\ \underline{\mbox{at }t}\ [ f(x) = 1 \ensuremath{\rightarrow} g(x) = 1 ]$

\ex. $\sv{\emph{every}}^{t} = \ensuremath{\lambda}f\ensuremath{\in}D_{\type{e,t}}. \ensuremath{\lambda}g\ensuremath{\in}D_{\type{e,t}}.\ensuremath{\forall}x 
[ f(x) = 1\ \underline{\mbox{at }t} \ensuremath{\rightarrow} g(x) = 1\ \underline{\mbox{at }t} ]$

Does either of these alternative entries make sense? If so, what does it say? Is it equivalent to our official entry? Could it lead to different predictions about the truth-conditions of English sentences?\eex
\end{exercise}


\subsection{A Word of Caution}
Compare the semantics given for \emph{former} and the one for \emph{PAST}:

\ex. $\sv{\mbox{former}} = \lambda f \in D_{\type{s,\type{e,t}}}. \lambda x.[f(t)(x)=0\ \&\ \exists t' \mbox{ before } t: f(t')(x) = 1].$

\ex. $\sv{\mbox{PAST}}^{t} = \ensuremath{\lambda}p\ensuremath{\in}D_{\type{s,t}}.\ \ensuremath{\exists}t'\mbox{ before }t: p(t') = 1$

Notice that these entries have an interesting consequence:

\ex. \a. John is a former teacher.
\b. John was a teacher.

The two sentences in \Last differ in their truth-conditions. The sentence in \Last[a] can only be true if John is not a teacher anymore while this is not part of the truth-conditions of the sentence in \Last[b]. To see that this analysis is in fact correct, consider this:

\ex. Last night, John was reading a book about tense.
\a. !! The authors are former Italians.
\b.  The authors were Italian.

Consider the past tense in \Last[b]. It is not (necessarily) interpreted as claiming that the authors are not Italian anymore. But this is in fact required by \Last[a].

There are some cases where it seems that the past tense does trigger inferences that one would not expect from the lexical entry that we gave. Surely, if I tell you \emph{My cousin John was a teacher} you will infer that he isn't a teacher anymore. In fact, you may even infer that he is not alive anymore. One promising approach that tries to reconcile a semantics like ours with the possibility of stronger inferences in some contexts is based on pragmatic considerations, see \cite{musan:1997:lifetime}.

Examples like the one in \Last are problematic for widely held naive conceptions of what the past tense means. One often hears that \emph{PAST} expresses the fact that ``the time of the reported situation precedes the speech time''. If this were to mean that the time of the book's authors being Italian precedes the speech time, this would presumably wrongly predict that they would have to be not Italian anymore for the sentence to be true (or usable).

\section{Are Tenses Referential?}

Our semantics for the past tense treats it essentially as an existential quantifier over times (albeit in the meta-language), the same way we treated possibility modals as existential quantifiers over (accessible) worlds. This seems quite adequate for examples like \Next, which seem to display the expected quantified meaning:

\ex. John went to a private school.

All we learn from \Last is that at some point in the past, whenever it was that John went to school, he went to a private school.

Partee in her famous paper ``Some structural analogies between tenses and pronouns in English'' \citep{partee:1973:analogies} presented an example where tense appears to act more ``referentially'':

\ex. I didn't turn off the stove.

``When uttered, for instance, halfway down the turnpike, such a sentence clearly does not mean either that there exists some time in the past at which I did not turn off the stove or that there exists no time in the past at which I turned off the stove. The sentence clearly refers to a particular time \dash not a particular instant, most likely, but a definite interval whose identity is generally clear from the extralinguistic context, just as the identity of the \emph{he} in [\emph{He shouldn't be in here}] is clear from the context.''

Partee here is arguing that neither of the two plausible LFs derivable in our current system correctly captures the meaning of \Last. Given that the sentence contains a past tense (which we have treated as an existential quantifier over past times) and a negation, we need to consider two possible scopings of the two operators:

\ex. \a. PAST NEG I turn off the stove.
\b. NEG PAST I turn off the stove.

\begin{exercise}
	
Show that neither LF in \Last captures the meaning of \LLast correctly.\eex

\end{exercise}

At this point, we will not develop Partee's analysis in formally explicit detail. If tenses refer to times, it would be easiest to give up on the treatment of times as evaluation parameters and move to a system where times are object language arguments of time-sensitive expressions. We will see a system of that nature later on.

In a commentary on Partee's paper at the same conference it was presented at, Stalnaker pointed out that the Priorean theory can in fact deal with \LLast, if one allows the existential quantifier over times to be contextually restricted to times in the salient interval of Partee leaving her house \dash since natural language quantifiers are typically subject to contextual restrictions, this is not a problematic assumption. (Note that Partee formulated her observation in quite a circumspect way: ``The sentence refers to a particular time''; Stalnaker's suggestion is that the reference to a particular time is part of the restriction to the quantifier over times expressed by tense, rather than tense itself being a referring item (of type $s$).)

\begin{exercise}
	Assuming a restricted existential quantification à la Stalnaker, which of the LFs in \Last captures the meaning of \LLast correctly?\eex
\end{exercise}

\citet{ogihara:1995:tense-embedded} argues that the restricted existential quantification view is in fact superior to Partee's analysis, since Partee's analysis needs an existential quantifier anyway.\marginnote{\citealt{partee:1984:anaphora} adopts an existential quantifier analysis.} Note that it is clear that the time being referred to is a protracted interval (the time during which Partee was leaving her house). But the sentence is not interpreted as saying that this interval is not a time at which she turned off her stove, which would have to be a fairly absurd turning-off-of-the-stove (turning off the stove only takes a moment and doesn't take up a significant interval). Instead, the sentence says that \emph{in} that salient interval there is no time at which she turned off the stove. Clearly, we do need an existential quantifier in there somewhere and the Priorean theory provides one.\footnote{Clearly, the alternative is to say that the existential quantifier is not expressed by tense but comes from somewhere else, perhaps aspect, perhaps in the lexical meaning of \emph{turn off}. We will not pursue those options here.} Ogihara makes the point with the following example:

\ex. \a.[John:] Did you see Mary? 
\b.[Bill:] Yes, \emph{I saw her}, but I don't remember exactly when.

The question and answer in this dialogue concern the issue of whether Bill saw Mary at \emph{some} time in a contextually salient interval.

\section{The Need for Intervals}

We have just seen a reason to recognize that natural language can talk not just about moments of time but also about intervals (connected sets of moments), which is a fairly trivial fact; after all, what does \emph{the year 2010} refer to if not an interval of time? We have to go even farther, though. It can be shown that we need the time parameter of the evaluation function to be able to be an interval. Consider the tenseless clause \emph{John build a house} and consider a situation where John starts building a house (the only house he has ever built) on April 1, 2009 and finishes building it on April 1, 2010. Now, which times do we want to be times \emph{at which} ``John build a house'' is true? If we allow the clause to be true at moments during the building, we would make it true at other times during the building (the ones after the first times) that \emph{John built a house}, but that is wrong. So, the time(s) at which ``John build a house'' cannot be before April 1, 2010. And clearly, times after April 1, 2010 cannot be times at which ``John build a house'' is true. So, perhaps, the only time at which ``John build a house'' is true is the moment on April 1, 2010 when he finishes building the house? But then we would incorrectly predict that on the day before, when he has already been building the house for almost a year, we can truthfully say that \emph{John will build a house}. So, no moment of time can be the time at which ``John build a house'' is true. The solution is that the time at which ``John build a house'' is true is exactly the interval that starts with the first moment of the building project and ends with the last nail hammered into the wall. Then, we can say before April 1, 2009 that John will build a house and after April 1, 2010, that John built a house.

What can we say during the building of the house, though? The English present tense is not correctly used in this circumstance:

\ex. !!John builds a house.

Our analysis may be read as predicting this fact. Assume that for an unembedded clause, the time parameter is set to be the speech time. But what is the speech time? Perhaps, it is the exact interval it takes to utter the particular clause being evaluated. If so, an example like \Last can only possibly be true if the speech interval exactly coincides with the reported event, here the building of the house. That is, the speaker of \Last would have to ensure that she starts speaking at the very first moment of John's building the house, continues speaking rather slowly, and then finishes speaking with the very last nail. It is intriguing to note that sentences like \Last become acceptable in situations where a sentence is conceived of as exactly coinciding the event being reported, namely play-by-play sports commentary (``He passes the ball to Messi'').\footnote{We cannot go into this fascinating topic further here, but there is much more to explore about the peculiar nature of \Last. \citet{bennett-partee:1978:tense-aspect} assume that the speech time is a moment and use that assumption to derive the nature of \Last. \citet{ejerhed:1974:tense} calls the typical use of \Last, the ``voyeur present''; see also \citealt{cooper:1986:tense}.}

What English needs to do instead is to use the progressive:

\ex. John is building a house.

\Last expresses that the speech time is included in an interval of John building a house. Elements that connect the evaluation time to the time at which a predicate holds are usually called \emph{aspectual} operators or simply \emph{aspects}. The English progressive then is an aspectual expression. We will look closer at its meaning in a little while.

\section{Aktionsarten}

We can distinguish predicates with respect to their temporal profile. The traditional classification has four categories:

\begin{itemize}
	\item \emph{accomplishment} predicates
	\item \emph{achievement} predicates
	\item \emph{activity} predicates
	\item \emph{stative} predicates
\end{itemize}

Accomplishment predicates (\emph{build a house}, \emph{cross the street}) describe an event that has a defined beginning and end (\emph{telos}, `goal') and takes some amount of time to finish. Achievement predicates (\emph{reach the summit}, \emph{notice the problem}) also have a \emph{telos} but are conceived of as describing an instantaneous event. Accomplishment predicates and achievement predicates constitute the class of \emph{telic} predicates.

Activity predicates (\emph{run}, \emph{dance}) describe events that are not conceived of as having a defined goal. Stative predicates (\emph{be in New York}, \emph{know French}) describe states that are true of intervals. The difference between activity predicates and stative predicates is often said to turn on whether there is an agent being active in the described event.

\bigskip\noindent\dash Read \citealt[Chapter 1, pp. 1--35]{rothstein:2004:events} \dash

\section{The Progressive}

\noindent\dash Read \citealt{portner:1998:progressive} \dash

\section{Tense in Embedded Clauses}

What happens to the time-sensitivity of the verb in a tenseless clause? Consider ECM complements to verbs of believing:

\ex.\label{ex:rain} John believed it to be raining.

Evidently, there is some kind of dependency of the time reference in the lower clause and the higher clause. The simplest approach in our framework would be to have \emph{believe} pass down its evaluation time to the lower clause and to assume that the lower clause doesn't have a tense operator. Then, whatever time \emph{believe} is being interpreted at would be the same time that the lower verb would be evaluated at.

\ex.\label{ex:believe-pass} $\sv{\text{believe}}^{w,t} = \lambda p_{\type{s,t}}. \lambda x.\ p(w',t),$ for all worlds $w'$ compatible with what $x$ believes in $w$ at $t$.

Together with the rest of the system, we predict that \LLast will be true iff there is a past time \textbf{t} such that it is raining at \textbf{t} in all worlds which conform to what John believes at \textbf{t}, which seems adequate. Unfortunately, it only \emph{seems} adequate.

Consider these four worlds:

\begin{itemize}
	\item[$w_1$] rain at 4am, John awake at 4am
	\item[$w_2$] rain at 4am, John awake at 5am
	\item[$w_3$] rain at 5am, John awake at 4am
	\item[$w_4$] rain at 5am, John awake at 5am
\end{itemize}

Assume that in all four worlds, John wakes up, has no idea what time it is, hears a dripping noise, and says to himself ``it is raining (now)''. Which worlds conform to what John believes at 4am in $w_1$? In which worlds is it raining at 4am? Are the former a subset of the latter? No!

Consider a variant of the story. Everything is the same as above, except that John wakes up, thinks it is 5am and says to himself: ``It was raining at 4am.'' Fact: Sentence \ref{ex:rain} is not a true report of John's beliefs in $w_1$ in this story. Why not? There is a description, viz. \emph{4am}, which in fact picks out the time of John's thinking, and under which he ascribes rain to that time.

Conclusion: Sentence \ref{ex:rain} unambiguously means that there is a past time \textbf{t} such that John at \textbf{t} ascribes rain to \textbf{t} under the description ``now''. We need to capture this but the proposal encapsulated in \ref{ex:believe-pass} doesn't achieve this.

The solution: \emph{believe} (and other attitude verbs, or perhaps the complementizer they select) controls not just the world parameter of its prejacent but also the time parameter.

\ex. $\sv{\text{believe}}^{w,t} = \lambda p_{\type{s,t}}. \lambda x.\ p(w',t'),$ for all worlds $w'$ and $t'$ such that for all that $x$ can tell in $w$ at $t$, $x$ might be located in $w'$ at $t'$.

On this analysis, \ref{ex:rain} means essentially that John located himself at a raining time. This is intuitively correct.

\bigskip\noindent\dash More on tense in tensed complement clauses \dash

% Sequence of tense. Present under past. Tense deletion. Abusch example of past not in the past of anything.

\section*{Supplementary Readings} \label{sec:suppl-read-tense}

\phantomsection \addcontentsline{toc}{section}{Supplemental Readings}

{\setlength{\parindent}{0pt}\setlength{\parskip}{6pt}

A nice and gentle introduction to some of the issues discussed in this chapter comes from Ogihara:

\begin{bibentrylist}
	\item \bibentry{ogihara:2007:tense}. 
\end{bibentrylist}

Partee's seminal paper is a must read:

\begin{bibentrylist}
	\item \bibentry{partee:1973:analogies}. 
\end{bibentrylist}

Musan's work on the pragmatic effects of tense:

\begin{bibentrylist}
	\item \bibentry{musan:1997:lifetime}. 
\end{bibentrylist}

The three essential works on the progressive:

\begin{bibentrylist}
	\item \bibentry{dowty:1977:progressive}.
	\item \bibentry{landman:1992:progressive}.
	\item \bibentry{portner:1998:progressive}.
\end{bibentrylist}

The first chapter of Susan Rothstein's book on lexical aspect gives a nice overview of Aktionsarten/aspectual classes:

\begin{bibentrylist}
	\item \bibentry{rothstein:2004:events}, Chapter 1: ``Verb Classes and Aspectual Classification'', pp. 1--35, available online at \url{http://tinyurl.com/rothstein-aktionsarten}. 
\end{bibentrylist}

Concise statements of some of the issues surrounding dependent tenses:

\begin{bibentrylist}
	\item \bibentry{stechow:1995:proper-tense}.
	\item \bibentry{stechow:2009:tenses}.
\end{bibentrylist}

% chapter tense (end)
