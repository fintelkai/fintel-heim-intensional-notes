\cleartoevenpage
\thispagestyle{empty}
\vspace*{\fill}

\noindent\emph{Hypotheticals, `imaginaries', conditionals, the syntax of
  counter-factuality and contingency may well be the generative centres of human
  speech. [\dots] Language is the main instrument of man's refusal to accept the
  world as it is.}

\medskip
\noindent George Steiner, \emph{After Babel} (1975), p.226 \& p.228
\nocite{steiner-1975-babel}

\vfill\vfill\null

\chapter{Beginnings}\label{cha:beginnings}

\minitoc

\section{Displacement}\label{sec:displacement}

\cite{hockett-1960-logical-considerations} presented a list of \term{design
  features of human language}, which continues to play a role in current
discussions of animal communication and the evolution of language. One of the
design features is \term{displacement}: human language is not restricted to
discourse about the \emph{actual here and now}. %
\note{Hockett cites this passage from \cite[p.30]{childe-1936-man-make-himself}
  to illustrate the utility of displacement:\\[\baselineskip]
  \noindent``[\dots] parents can, with the aid of language, instruct their offspring how to deal
  with situations which cannot conveniently be illustrated by actual concrete
  examples. The child need not wait till a bear attacks the family to learn how
  to avoid it. Instruction by example alone in such a case is liable to be fatal
  to some of the pupils. Language, however, enables the elders to forewarn the
  young of the danger while it is absent, and then demonstrate the appropriate
  course of action.''}%
We use language to speculate about how things might have
been different, about what would happen if we don't find dinner soon; we wonder
what our friend thinks the world is like; we may want to tell our boss what it
would take for us not to resign.

How does natural language untie us from the actual here and now? One degree of
freedom is given by the ability to name entities and refer to them even if they
are not where we are when we speak:

\ex Rahel is in Hamburg. \xe
%
This kind of displacement is not something we will explore here. We'll take it
for granted.

Consider a sentence with no names of absent entities in it:

\ex It is snowing.\qquad [said in Cambridge] \xe
%
On its own, \Last makes a claim about what is happening right now here in
Cambridge. But there are devices at our disposal that can be added to \Last,
resulting in claims about snow in displaced situations. Displacement can occur
in the \term{temporal} dimension: 

\ex At noon yesterday, it was snowing. \xe
%
This sentence makes a claim not about snow now but about snow at noon yesterday,
a different time from now. We will look at temporal semantics in Part 2 of this
book.

\kwn%
\note{The terms \term{modal} and \term{modality} descend from the Latin
  \expression{modus}, ``way'', and are ancient terms pertaining to the way a
  proposition holds, necessarily, contingently, etc. For more on the history,
  see \cite{auwera-aguilar-2015-mood-modality}.}%
\note{If we wanted to be more fanciful, we could call this the Fifth Dimension
  (after the four dimensions of space and time). Take a look at the original
  intro to \emph{The Twilight Zone}:
  \url{https://www.youtube.com/watch?v=vB1Ot9MEOOs}}%
In Part 1 of this book, we will focus on what might be called the \term{modal}
dimension. Here's an example of modal displacement:

\ex \term{counterfactual conditional}\\
If the storm system hadn't been deflected by the jet stream, it would have
been snowing. \xe
%
This sentence makes a claim not about snow in the actual world but about snow in
the world as it would have been if the storm system hadn't been deflected by the
jet stream, a world distinct from the actual one (where the system did not hit
us), a merely \term{possible world}.

Natural language abounds in modal constructions (see
\cite{kratzer-1981-notional}). Here are some other examples:

\lingset{sampleexno=(12)}%
%
\ex \term{Modal Auxiliaries}\\
It may be snowing. \xe

\ex~ \term{Modal Adverbs}\\
Possibly, it will snow tomorrow. \xe

\ex~ \term{Propositional Attitudes}\\
Miriam believes that it is snowing. \xe

\ex~ \term{Evidentials}\\
It appears that it is snowing.\xe

\ex~ \term{Habituals}\\
Ellen smokes. \xe

\ex~ \term{Generics}\\
Bears like honey. \xe

\ex~ \term{Imperatives}\\
Get your snow shovels ready! \xe

\enlargethispage{24pt}
\ex~ \term{Sufficiency and Excess}\\
Linda is old enough to watch The X-Files.\\
Klara is too expensive to hire. \xe

\ex~ \term{Infinitival relatives}\\
I know \choice{an,the} expert to talk to. \xe

\begin{exercise}
  Collect some examples of modal displacement in a language other than English
  (whether spoken by you or someone else you know). These can serve as a
  touchstone of understanding as we proceed.\qed
\end{exercise}

In this chapter, we will put in place the basic framework of \term{intensional
  semantics}, the kind of semantics that models displacement of the point of
evaluation in temporal and modal dimensions. To do this, we will start with one
rather special example of modal displacement:

\ex\label{sherlock}\marginfig[0.5]{holmes.png}%
In the world of Sherlock Holmes, a detective lives at 221B Baker Street.\xe
%
\Last doesn't claim that a detective lives at 221B Baker Street in the actual
world (presumably a false claim), but that in the world as it is described in
the Sherlock Holmes stories of Sir Arthur Conan Doyle, a detective lives at 221B
Baker Street (a true claim, of course). We choose this example rather than one
of the more well-studied displacement constructions because we want to focus
on conceptual and technical matters for now before we get distracted by many
interesting complications.

The questions we want to answer are: %
%\marginfig[0.5]{garp.jpg}%
How does natural language achieve this feat of modal displacement? How do we
manage to make claims about other possible worlds? And why would we want to?

To make displacement possible and compositionally tractable, we need meanings of
natural language expressions, and of sentences in particular, to be displaceable
in the first place. They need to be ``portable'', so to speak, able to make
claims about more than just the actual here and now. And we need other
natural language expressions that take that portable meaning and apply it to
some situation other than the actual here and now. That is what intensionality
is all about.

The basic idea of the account we'll develop is this:

\begin{itemize}
\item expressions are assigned their semantic values relative to a possible
  world;
\item in particular, sentences have truth-values in possible worlds;
\item in the absence of modal displacement, we evaluate sentences with respect
  to the ``actual'' world, the world in which we are speaking;
\item modal displacement changes the world of evaluation;
\item displacement is effected by special operators, whose semantics is our
  primary concern here.
\end{itemize}
%
A terminological note: we will call the sister of the intensional operator its
\term{prejacent}, a useful term introduced by our medieval colleagues.

\section{An intensional semantics in 10 easy steps}\label{sec:10-steps}

\subsection{Laying the foundations} \label{sec:laying-foundations}

\subsubsection{Step 1: Possible worlds} \label{sec:world-parameter}

Our first step is to introduce possible worlds. This is not the place to discuss
the metaphysics of possible worlds in any depth. Instead, we will just start
working with them and see what they can do for us. Basically, a possible world
is a way that things might have been. In the actual world, there are two coffee
mugs on my desk, but there could have been more or less. So, there is a possible
world \dash albeit a rather bizarre one \dash where there are 17 coffee mugs on
my desk. We join Heim \& Kratzer in adducing this quote from
\citet[1f.]{lewis-1986-plurality-worlds}:

\begin{quote}
  \marginfig{lewis.jpg}%
  The world we live in is a very inclusive thing. Every stick and every stone
  you have ever seen is part of it. And so are you and I. And so are the planet
  Earth, the solar system, the entire Milky Way, the remote galaxies we see
  through telescopes, and (if there are such things) all the bits of empty space
  between the stars and galaxies. There is nothing so far away from us as not to
  be part of our world. Anything at any distance at all is to be included.
  Likewise the world is inclusive in time. No long-gone ancient Romans, no
  long-gone pterodactyls, no long-gone primordial clouds of plasma are too far
  in the past, nor are the dead dark stars too far in the future, to be part of
  the same world. \dots
	
  The way things are, at its most inclusive, means the way the entire world is.
  But things might have been different, in ever so many ways. This book of mine
  might have been finished on schedule. Or, had I not been such a commonsensical
  chap, I might be defending not only a plurality of possible worlds, but also a
  plurality of impossible worlds, whereof you speak truly by contradicting
  yourself. Or I might not have existed at all \dash neither myself, nor any
  counterparts of me. Or there might never have been any people. Or the physical
  constants might have had somewhat different values, incompatible with the
  emergence of life. Or there might have been altogether different laws of
  nature; and instead of electrons and quarks, there might have been alien
  particles, without charge or mass or spin but with alien physical properties
  that nothing in this world shares. There are ever so many ways that a world
  might be: and one of these many ways is the way that this world is.
\end{quote}%
%
\clearpage
\note{It's possible that your previous inventory also included pluralities,
  events, and/or degrees. We're just adding to the menagerie now. Questions
  arise about what the limits are and whether the inventory is universal. For
  some discussion, see the manuscript ``A typology of semantic entities''
  \parencite{rett-2019-types}.}%
Previously, our ``metaphysical inventory'' included a domain of entities and a
set of two truth-values and increasingly complex functions between entities,
truth-values, and functions thereof. Now, we will add possible worlds to the
inventory. Let's assume we are given a set $W$, the set of all possible worlds,
which is a vast space since there are so many ways that things might have been
different from the way they are. Each world has as among its parts entities like
you and me and these coffee mugs. Some of them may not exist in other possible
worlds. So, strictly speaking each possible worlds has its own, possibly
distinctive, domain of entities. What we will use in our system, however, will
be the grand union of all these world-specific domains of entities. We will use
$D$ to stand for the set of all possible individuals.

Among the many possible worlds that there are \dash according to Lewis, there is
a veritable plenitude of them \dash is the world as it is described in the
Sherlock Holmes stories by Sir Arthur Conan Doyle. In that world, there is a
famous detective Sherlock Holmes, who lives at 221B Baker Street in London and
has a trusted sidekick named Dr. Watson. Our sentence \expression{In the world
  of Sherlock Holmes, a detective lives at 221B Baker Street} displaces the
claim that a famous detective lives at 221B Baker Street from the actual world
to the world as described in the Sherlock Holmes stories. In other words, the
following holds (until we revise it):

\ex The sentence \expression{In the world of Sherlock Holmes, a detective lives
  at 221B Baker Street} is true in a world w iff the sentence \expression{a
  detective lives at 221B Baker Street} is true in the world as it is described
in the Sherlock Holmes stories. \xe

What this suggests is that we need to make space in our system for having
devices that control in what world a claim is evaluated. This is what we will do
now.

\subsubsection{Step 2: The evaluation world parameter}
\label{sec:eval-world-param}

Recall from H\&K that we were working with a semantic interpretation function
that was relativized to an assignment function $g$, which was needed to take
care of pronouns, traces, variables, etc. From now on, we will relativize the
semantic values in our system to possible worlds as well. What this means is
that from now on, our interpretation function will have two superscripts: a
world $w$ and an assignment $g$: $\svt{\textperiodcentered}^{w,g}$. For a given
expression $\phi$, we call $\sv{\phi}^{w,g}$ the \term{extension} of $\phi$ at $w$
(relative to $g$). The extension of an expression at a world is the central
notion of an intensional semantics. 

\clearpage
\note{Recall from H\&K, pp.22f, that what's inside the interpretation brackets
  is a mention of an object language expression. They make this clear by
  bold-facing all object language expressions inside interpretation brackets. In
  this book, we will follow common practice in the field and not use a special
  typographic distinction, but let it be understood that what is interpreted are
  object language expressions.}%
So, the prejacent embedded in \refx{sherlock} will have its truth-conditions
described as follows:

\ex For any world \(w\) and assignment function \(g\):\\
$\svt{a famous detective lives at 221B Baker Street}^{w,g} = 1$\\
iff a famous detective lives at 221B Baker Street in world $w$. \xe

It is customary to refer to the world for which we are calculating the extension
of a given expression as the \term{evaluation world}. In the absence of any
shifting devices, we would normally evaluate a sentence in the actual world. But
then there are shifting devices such as our \expression{in the world of Sherlock
  Holmes}. We will soon see how they work. But first some more pedestrian steps:
adding lexical entries and composition principles that are formulated relative
to a possible world. This will allow us to derive the truth-conditions as stated
in (\lastx) in a compositional manner.

\subsubsection{Step 3: Lexical entries} \label{sec:lexical-entries}

Among our lexical items, we can distinguish between items which have a
\term{world-dependent} semantic value and those that are world-independent.
Let's start with the entry for \expression{famous}:

\ex For any $w \in W$ and any assignment function $g$:\\
$\svt{famous}^{w,g} = \lambda x \in D.\ x \mbox{ is famous in }
w$.
\xe
%
Of course, ``$\lambda x \in D.\ \dots$'' is short for
``$\lambda x\co x \in D.\ \dots$''. Get used to semanticists condensing their
notation whenever convenient! A further step of condensation is taken below:
``$\lambda x\co x \in D_e.\ \dots$'' becomes ``$\lambda x_e.\ \dots$''.

Always make sure that you actually understand what the notation means. Here, for
example, we are saying that the semantic value of the word \emph{famous} with
respect to a given possible world $w$ and a variable assignment $g$ is that
function that is defined for an argument $x$ only if $x$ is a member of the
domain of individuals and that, if it is defined, yields the truth-value 1 if
and only if $x$ is famous in $w$. \Last does \emph{not} mean that the
function maps $x$ to ``$x$ is famous in $w$'', which would be very weird:
mapping an individual to a meta-language statement!

\note{\Next shows another customary condensation of notation: the universal
  quantification over evaluation worlds and assignment functions is left
  implicit and on the face of it, the parameters seem unbound. But they are.}%
A couple more predicates:

\pex \a
$\svt{detective}^{w,g} = \lambda x \in D.\ x \mbox{ is a detective in } w$. \a
$\svt{lives-at}^{w,g} = \lambda x \in D.\ \lambda y \in D.\ y \mbox{ lives-at }
x \mbox{ in } w $. \xe
%
The set of detectives will obviously differ from world to world, and so will the
set of famous individuals and the set of pairs where the first element lives at
the second element.

\clearpage
Other items have semantic values which do not differ from world to world. The
most important such items are certain ``logical'' expressions, such as
truth-functional connectives and determiners:\note{Again, note the ruthless
  condensation of the notation in (c) and (d): variables are subscripted with
  the type of the domain that their values are constrained to come from.}

\pex
\a $\svt{and}^{w,g} = \lambda u \in D_t.\ \lambda v \in D_t.\ u=v=1
$.
\a $\svt{the}^{w,g} = \lambda f \in D_{\type{e,t}}\co \exists !
x[f(x) = 1]. \mbox{ the } y \mbox{ such that } f(y) = 1$.
\a $\svt{every}^{w,g} = \lambda f_{\type{e,t}}.\ \lambda
h_{\type{e,t}}.\ \forall x_e\co f(x) = 1 \rightarrow h(x) = 1 $.
\a $\svt{a/some}^{w,g} = \lambda f_{\type{e,t}}.\ \lambda
h_{\type{e,t}}.\ \exists x_e\co f(x) = 1\ \&\ h(x) = 1 $.
\xe
%
Note that there is no occurrence of $w$ on the right-hand side of the entries in
(\lastx). That's the tell-tale sign of the world-independence of the semantics of
these items.

We will also assume that proper names have world-independent semantic values,
that is, they refer to the same individual in any possible world.

\pex
\a $\svt{Noam Chomsky}^{w,g}$ = Noam Chomsky. 
\a $\svt{Sherlock Holmes}^{w,g}$ = Sherlock Holmes. 
\a $\svt{221B Baker Street}^{w,g}$ = 221B Baker Street.
\xe

\subsubsection{Step 4: Composition principles} \label{sec:comp-princ}

The old rules of Functional Application, Predicate Modification, and
$\lambda$-Abstraction can be retained almost intact. We just need to modify them
by adding world-superscripts to the interpretation function. For example:

\ex \term{Functional Application (FA)}\\
If $\alpha$ is a branching node and $\{\beta, \gamma\}$ the set of its
daughters, then, for any world $w$ and assignment $g$: if
$\sv{\beta}^{w,g}$ is a function whose domain contains
$\sv{\gamma}^{w,g}$, then
$\sv{\alpha}^{w,g} = \sv{\beta}^{w,g} (\sv{\gamma}^{w,g})$.
\xe
%
The rule simply passes the world parameter down.

\begin{exercise}
  Formulate the new versions of Predicate Modification and
  \(\lambda\)-Abstraction. \qed
\end{exercise}

\subsubsection{Step 5: Truth} \label{sec:truth}

We will want to connect our semantic system to the notion of the \term{truth of
  an utterance}. We first adopt the ``Appropriateness Condition'' from Heim \&
Kratzer (p.243):

\ex \extitle{Appropriateness Condition}\\
A context $c$ is \emph{appropriate} for an LF $\phi$ only if $c$
determines a variable assignment $g_c$ whose domain includes every
index which has a free occurrence in $\phi$. \xe

\kwn
We then intensionalize Heim \& Kratzer's definition of truth and falsity of
utterances:

\ex \extitle{Truth and Falsity Conditions for Utterances}\\
An utterance of a sentence $\phi$ in a context $c$ in a possible world $w$ is
\emph{true} iff $\sv{\phi}^{w,g_c} = 1$ and \emph{false} if $\sv{\phi}^{w,g_c} =
0$.
\xe

\begin{exercise}
  Compute under what conditions an utterance in possible world $w_7$ (which may
  or may not be the one we are all living in) of the sentence \expression{a
    famous detective lives at 221B Baker Street} is true. \qed
\end{exercise}

\subsection{Intensional operators} \label{sec:intens-oper}

\note{Here's an illustration of the basic idea we're trying to implement:\\[12pt]
  
  \begin{tikzpicture}
    \node (w) at (0,0) {$\sv{Op\ \phi}^{w}$};
    \node (=) at (0,-0.5) {$=$};
    \node (Op) at (-1,-1.5) {$\sv{Op}^w$};
    \node (phi) at (1,-1.5) {$\sv{\phi}$};
    \node (?) at (phi.north east) [draw,circle] {};
    \node (Op-circle) [draw,circle,fit= (Op),inner sep=0pt] {};
    \draw [->] (Op-circle) .. controls (0,-0.7) and (0.5,-0.5) ..
    node[midway,below,sloped] {\scriptsize controls} (?.north west);
  \end{tikzpicture}
}%
%
So far we have merely ``redecorated'' the system inherited from Heim \&
Kratzer. We have introduced possible worlds into our inventory, our lexical
entries and our old composition principles. But with the tools we have now, all
we can do so far is to keep track of the world in which we evaluate the semantic
value of an expression, complex or lexical. We will get real mileage once we
introduce \term{intensional operators} which are capable of shifting the world
parameter. We mentioned that there are a number of devices for modal
displacement. As advertised, for now, we will just focus on a very particular
one: the expression \expression{in the world of Sherlock Holmes}. We will
assume, as seems reasonable, that this expression is a sentence-modifier both
syntactically and semantically.

\subsubsection{Step 6: A syncategorematic entry} \label{sec:sync-entry}

We begin with a heuristic step. We want to derive something like the following
truth-conditions for our sentence:

\ex $\sv{\mbox{in the world of Sherlock Holmes,}$\\
  $\mbox{a famous detective lives at 221B Baker Street}}^{w,g} = 1$\\
iff the world $w'$ as it is described in the Sherlock Holmes stories is such
that there exists a famous detective in $w'$ who lives at 221B Baker Street in
$w'$.
\xe

We would get this if in general we had this rule for \expression{in the world of
  Sherlock Holmes}:

\ex For any sentence $\phi$, any world $w$, and any assignment $g$:\\
$\sv{\mbox{in the world of Sherlock Holmes, }\phi}^{w,g} = 1$\\
iff the world $w'$ as it is described in the Sherlock Holmes stories is such
that $\sv{\phi}^{w',g} = 1$.
\xe

This is a so-called \term{syncategorematic} treatment of the meaning of this
expression. Instead of giving an explicit semantic value to the expression, we
specify what effect it has on the meaning of a complex expression that contains
it. In (\lastx), we do not compute the meaning for \expression{in the world of
  Sherlock Holmes, $\phi$} from the combination of the meanings of its parts,
since \expression{in the world of Sherlock Holmes} is not given a separate
meaning, but in effect triggers a special composition principle. This %
\note{The diamond $\Diamond$ symbol for possibility is due to C.I. Lewis, first
  introduced in \cite{lewis-langford-1932-symbolic}, but he made no use of a
  symbol for the dual combination $\neg\Diamond\neg$. The dual symbol $\Box$
  (``Box'') was later devised by F.B. Fitch and first appeared in print in 1946
  in a paper by his doctoral student \citet{barcan-1946-calculus}. See footnote
  425 of \cite{hughes-cresswell-1968-introduction}. Another notation one finds
  is $L$ for necessity and $M$ for possibility, the latter from the German
  \expression{möglich} `possible'.}%
format is very common in modal logic systems, which usually give a
syncategorematic semantics for the two classic modal operators (the necessity
operator $\Box$ and the possibility operator $\Diamond$). When one only has a
few closed class expressions to deal with that may shift the world parameter,
employing syncategorematic entries is a reasonable strategy. But we are facing a
multitude of displacement devices. We will therefore need to make our system
more modular.

We want to give \expression{in the world of Sherlock Holmes} its own meaning
and combine that meaning with that of its prejacent by a general composition
principle. The Fregean slogan we adopted says that all composition is function
application (modulo the need for $\lambda$-abstraction and the possible need for
predicate modification).\note{See Heim \& Kratzer, Section 4.3, pp. 63--72 for a
  reminder about the status of predicate modification.}

What we will want to do is to make (\blastx) be the result of functional
application. But we can immediately see that it cannot be the result of our
usual rule of functional application, since that would feed to \expression{in
  the world of Sherlock Holmes} the semantic value of \expression{a famous
  detective lives in 221B Baker Street} in $w$, which would be a particular
truth-value, $1$ if a famous detective lives at 221B Baker Street in $w$ and $0$
if there doesn't. And whatever the semantics of \expression{in the world of
  Sherlock Holmes} is, it is certainly \emph{not} a truth-functional operator.

\enlargethispage{18pt}We need to feed something else to \expression{in the world of Sherlock
  Holmes}. At the same time, we want the operator to be able to shift the
evaluation world of its prejacent. Can we do this?

\begin{exercise}\enlargethispage{18pt}
  How would you show that \expression{in the world of Sherlock Holmes} is not a
  truth-functional operator? \qed
\end{exercise}

\subsubsection{Step 7: Intensions} \label{sec:intensions}

We will define a richer notion of semantic value, the \term{intension} of an
expression. This will be a function from possible worlds to the extension of the
expression in that world. The intension of a sentence can be applied to any
world and give the truth-value of the sentence in that world. Intensional
operators take the intension of their prejacent as their argument, that is we
will feed the intension of the embedded sentence to the shifting operator. The
operator will use that intension and apply it to the world it wants the
evaluation to happen in. Voilà.

\enlargethispage{24pt}
\note{Just like H\&K, we make no claim that the semantic values that are
  attributed to expressions in our framework fully capture what is informally
  meant by ``meaning''. But certainly, intensions come closer to ``meaning''
  than extensions.}%
Now let's spell that account out. Our system actually provides us with two kinds
of meanings. For any expression $\alpha$, we have $\sv{\alpha}^{w,g}$, the
semantic value of $\alpha$ in $w$, also known as the \term{extension} of
$\alpha$ in $w$. But we can also calculate $\lambda w. \sv{\alpha}^{w,g}$, the
function that assigns to any world $w$ the extension of $\alpha$ in that world.
This is usually called the \term{intension} of $\alpha$. We will sometimes use
an abbreviatory notation for the intension of $\alpha$:%

\clearpage
\ex\note{The notation with the subscripted cent-sign comes from Montague Grammar.
  See \cite[147]{dowty-wall-peters-1981-intro}.}%
$\svcent{\alpha}^g := \lambda w. \sv{\alpha}^{w,g}$ \xe

It should be immediately obvious that since the definition of intension
abstracts over the evaluation world, intensions are not world-dependent.
% \note{Since intensions are by definition not dependent on the choice of a
%   particular world, it makes no sense to put a world-superscript on the
%   intension-brackets. So don't ever write ``$\svcent{\dots}^{w,g}$''; we'll
%   treat that as undefined nonsense.}%
\note{The definition here is simplified, in that it glosses over the fact that
  some expressions, in particular those that contain \term{presupposition
    triggers}, may fail to have an extension in certain worlds. In such a case,
  the intension has no extension to map such a world to. Therefore, the
  intension will have to be a partial function. So, the official, more
  ``pedantic'', definition will have to be as follows:
  $\sv{\alpha}_{\mbox{\tiny\textcent}}^g := \lambda w\co \alpha \in
  \mbox{dom}(\sv{\null}^{w,g}). \sv{\alpha}^{w,g}$.}

Note that strictly speaking, it now makes no sense anymore to speak of
``\emph{the} semantic value'' of an expression $\alpha$. What we have is a
semantic system that allows us to calculate extensions (for a given possible
world $w$) as well as intensions for all (interpretable) expressions. We will
see that when $\alpha$ occurs in a particular bigger tree, it will always be
determinate which of the two ``semantic values'' of $\alpha$ is the one that
enters into the compositional semantics. So, that one \dash whichever one it is,
the extension or the intension of $\alpha$ \dash might then be called
``\emph{the} semantic value of $\alpha$ in the tree $\beta$''.

\note{The Port-Royal logicians distinguished \term{extension} from
  \term{comprehension}. Leibniz preferred the term \term{intension} rather than
  \term{comprehension}. The notion probably goes back even further. See
  \cite{spencer-1971-intension} for some notes on this. The possible worlds
  interpretation is due to \cite{carnap-1947-meaning-necessity}.}%
It should be noted that the terminology of \term{extension} vs. \term{intension}
is time-honored but that the possible worlds interpretation thereof is more
recent. The technical notion we are using is certainly less rich a notion of
meaning than traditionally assumed. (For example, Frege's ``modes of
presentation'' are not obviously captured by this possible worlds implementation
of extension/intension.)

\subsubsection{Step 8: Semantic types and semantic domains}
\label{sec:semantic-types}

If we want to be able to feed the intensions to lexical items like
\expression{in the world of Sherlock Holmes}, we need to have the appropriate
types in our system.

Recall that $W$ is the set of all possible worlds. And recall that $D$ is the
set of all \term{possible individuals} and thus contains all individuals
existing in the actual world \emph{plus} all individuals existing in any of the
merely possible worlds.

We now expand the set of semantic types, to add intensions. Intensions are
functions from possible worlds to all kinds of extensions. So, basically we want
to add for any kind of extension we have in our system, a corresponding kind of
intension, a function from possible worlds to that kind of extension.

We add a new clause, (\nextx c), to the definition of semantic types:\note{This
  is as good a place as any to add a sermon about the notation for types.
  Functional types like \(\type{e,t}\) are written with angled brackets because
of the mathematical connection between functions and ordered pairs. The basic
types like \(e\) and \(t\), however are not written with angled brackets, since
they are not of the type of functions. Please do not make this mistake.}

\pex \extitle{Semantic Types}
\a $e$ and $t$ are semantic types.
\a If $\sigma$ and $\tau$ are semantic types, then $\type{\sigma,\tau}$ is a
semantic type.
\a If $\sigma$ is a semantic type, then $\type{s,\sigma}$ is a semantic type.
\a Nothing else is a semantic type.
\xe

\kwn
We also add a fourth clause to the previous definition of semantic domains:

\pex \extitle{Semantic Domains}
\a $D_{e} = D$, the set of all possible
individuals
\a $D_{t} = \{0,1\}$, the set of truth-values
\a If $\sigma$ and
$\tau$ are semantic types, then $D_{\type{\sigma,\tau}}$ is the set of all
functions from $D_{\sigma}$ to $D_{\tau}$.
\a \term{Intensions}: If $\sigma$ is
a type, then $D_{\type{s,\sigma}}$ is the set of all functions from $W$ to
$D_{\sigma}$.
\xe

\note{Note a curious feature of this set-up: there is no type $s$ and no
  associated domain. This corresponds to the assumption that there are no
  expressions of English that take as their extension a possible world, that is,
  there are no pronouns or names referring to possible worlds. We will actually
  question this assumption in a later chapter. For now, we will stay with this
  more conventional set-up.}%
Clause (d) is the addition to our previous system of types. The functions of the
schematic type $\type{s,\dots}$ are intensions. Here are some examples of
intensions:

\enlargethispage{24pt}
\begin{itemize}
\item The intensions of sentences are of type $\type{s,t}$, functions from
  possible worlds to truth values. These are usually called \term{propositions}.
  Note that if the function is total, then we can see the sentence as picking
  out a set of possible worlds, those in which the sentence is true. More often
  than not, however, propositions will be \term{partial} functions from worlds
  to truth-values, that is functions that fail to map certain possible worlds
  into either truth-value. This will be the case when the sentence contains a
  presupposition trigger, such as \expression{the}. The famous sentence
  \expression{The King of France is bald} has an intension that (at least in the
  analysis sketched in Heim \& Kratzer) is undefined for any world where
  there fails to be a unique King of France.
\item The intensions of one-place predicates are of type
  $\type{s,\type{e,t}}$, functions from worlds to set of individuals. These
  are usually called \term{properties}.
\item The intensions of expressions of type $e$ are of type $\type{s,e}$,
  functions from worlds to individuals. These are usually called
  \term{individual concepts}.
\end{itemize}


\subsubsection{Step 9: A lexical entry for a shifter}
\label{sec:lexic-entry-expr}

We are ready to formulate the lexical entry for \expression{in the world of
  Sherlock Holmes}:\note{This is not yet the final semantics, see Section
  \ref{sec:comm-compl} for complications. One complication we will not even
  start to discuss is that obviously it is not a necessity that there are
  Sherlock Holmes stories in the first place and that the use of this operator
  \emph{presupposes} that they exist; so a more fully explicit semantics would
  need to build in that presuppositional component. Also, note again the
  condensed notation: ``$\lambda p_{\type{s,t}}.\ \dots$'' stands for the fully
  official ``$\lambda p\co p \in D_{\type{s,t}}.\ \dots$''.}

\ex $\svt{in the world of Sherlock Holmes}^{w,g} =$\\
$\lambda p_{\type{s,t}}.\ \mbox{the world } w'
\mbox{ as it is described in the Sherlock Holmes stories}$\\
$\mbox{is such that } p(w') = 1$. \xe

That is, \expression{in the world of Sherlock Holmes} expects as its argument a
function of type $\type{s,t}$, a proposition. It yields the truth-value 1 iff
the proposition is true in the world as it is described in the Sherlock Holmes
stories.

All that's left to do now is to provide \expression{in the world of Sherlock
  Holmes} with a proposition as its argument. This is the job of a new
composition principle.

\subsubsection{Step 10: Intensional Functional Application}
\label{sec:intens-funct-appl}

We add the new rule of Intensional Functional Application.

\ex \term{Intensional Functional Application (IFA)}\\
If $\alpha$ is a branching node and $\{\beta, \gamma\}$ the set of its
daughters, then, for any world $w$ and assignment $g$: if $\sv{\beta}^{w,g}$ is
a function whose domain contains $\svcent{\gamma}^g$, then $\sv{\alpha}^{w,g} =
\sv{\beta}^{w,g} (\svcent{\gamma}^g)$. \xe

This is the crucial move. It makes space for expressions that want to take the
intension of their sister as their argument and do stuff to it. Now, everything
is in place. Given \LLast, the semantic argument of \expression{in the world of
  Sherlock Holmes} will not be a truth-value but a proposition. And thus,
\expression{in the world of Sherlock Holmes} will be able to check the
truth-value of its prejacent in various possible worlds. To see in practice that
we have all we need, please do the following exercise.

\begin{exercise}
  Calculate the conditions under which an utterance in a given possible world
  $w_7$ of the sentence \expression{in the world of the Sherlock Holmes stories,
    a famous detective lives at 221B Baker Street} is true. \qed
\end{exercise}

\begin{exercise}
  What in our system prevents us from computing the extension of
  \expression{Watson is slow}, for example, by applying the intension of
  \expression{slow} to the extension of \expression{Watson}? What in our system
  prevents us from computing the extension of \expression{Watson is slow} by
  applying the intension of \expression{slow} to the intension of
  \expression{Watson}? \qed
\end{exercise}

\begin{exercise}\label{exercise:meta}%
  \note{Please think about this exercise before looking at
    Section~\ref{sec:meta-issues}, which explores this issue.}%
%
  What is wrong with the following equation:

	\ex[belowexskip=6pt] ($\lambda x.\ x \mbox{ is slow in } w$) (Watson) =
  Watson is slow in $w$ ? \xe
\enlargethispage{18pt}%
	[ Hint: there is nothing wrong with the following:

	\ex[aboveexskip=3pt] ($\lambda x.\ x \mbox{ is slow in } w$) (Watson) = 1 iff
  Watson is slow in $w$. ]\qed\xe 
\end{exercise}

\section{Comments and complications} \label{sec:comm-compl}

\subsection{Intensions all the way?} \label{sec:intensions-all-way}

We have seen that to adequately deal with expressions like \expression{in the
  world of Sherlock Holmes}, we need an intensional semantics, one that gives us
access to the extensions of expressions across the multitude of possible worlds.
At the same time, we have kept the semantics for items like \expression{and},
\expression{every}, and \expression{a} unchanged and extensional. This is not
the only way one can set up an intensional semantics. The following exercise
demonstrates this.
\begin{exercise}
	
  Consider the following ``intensional'' meaning for \expression{and}:
	
  \ex
  $\svt{and}^{w,g} = \lambda p_{\type{s,t}}.\ \lambda
  q_{\type{s,t}}.\ p(w) = q(w) = 1$.
  \xe
  %	
  With this semantics, the conjunction \expression{and} would operate on the
  intensions of the two conjoined sentences. In any possible world $w$, the
  complex sentence will be true iff the component propositions are both true of
  that world.
	
  Compute the truth-conditions of the sentence \expression{In the world of
    Sherlock Holmes, Holmes is quick and Watson is slow} both with the
  extensional meaning for \expression{and} given earlier and the intensional
  meaning given here. Is there any difference in the results? \qed
\end{exercise}

\noindent There are then at least two ways one could develop an
intensional system.
\begin{enumerate}
	[(i)] 
\item We could ``generalize to the worst case'' and make the semantics deliver
  intensions as \emph{the} semantic value of an expression. Such systems are
  common in the literature
  \citep[see][]{lewis-1970-general-semantics,cresswell-1973-logics}.
\item We could maintain much of the extensional semantics we have developed so
  far and extend it conservatively so as to account for non-extensional
  contexts.
	
\end{enumerate}
%
We have chosen to pursue (ii) over (i), because it allows us to keep the
semantics of extensional expressions simpler. The philosophy we follow is that
we will only move to the intensional sub-machinery when triggered by an
expression that creates a non-extensional context. As the exercise just showed,
this might be more a matter of taste than a deep scientific decision. We will
turn to questions of expressive power later in this book.

\subsection{Why talk about other worlds?} \label{sec:why-talk-about}

Why would natural language bother having such elaborate mechanisms to talk about
other possible worlds? While having devices for spatial and temporal
displacement (talking about Hamburg or what happened yesterday) seems eminently
reasonable, talking about worlds other than the actual world seems only suitable
for poets and the like. So, why?

The solution to this puzzle lies in a fact that our current semantics of the
shifter \expression{in the world of Sherlock Holmes} does not yet accurately
capture: modal sentences have empirical content, they make \term{contingent}
claims, claims that are true or false depending on the circumstances in the
actual world.

Our example sentence \expression{In the world of Sherlock Holmes, a famous
  detective lives at 221B Baker Street} is true in this world but it could
easily have been false. There is no reason why Sir Arthur Conan Doyle could not
have decided to locate Holmes' abode on Abbey Road.

To see that our semantics does not yet capture this fact, notice that in the
semantics we gave for \expression{in the world of Sherlock Holmes}:

\ex $\svt{in the world of Sherlock Holmes}^{w,g} =$\\
$\lambda p_{\type{s,t}}.\ \mbox{the world } w' \mbox{ as it is described in the
  Sherlock Holmes stories}$\\
$\mbox{is such that } p(w') = 1$.
\xe
%
there is no occurrence of $w$ on the right hand side. This means that the
truth-conditions for sentences with this shifter would be world-independent. In
other words, they are predicted to make non-contingent claims that are either
true no-matter-what or false no-matter-what. This needs to be fixed.

The fix is obvious: what matters to the truth of our sentence is the content of
the Sherlock Holmes stories as they are in the evaluation world. So, we actually
need the following semantics for our shifter:

\ex $\svt{in the world of Sherlock Holmes}^{w,g} =$\\
$\lambda p_{\type{s,t}}.\ \mbox{the world } w' \mbox{ as it is described in the
  Sherlock Holmes stories}$\\
$\mbox{\emph{in} } w \mbox{ is such that } p(w') = 1$. \xe

We see now that sentences with this shifter do make a claim about the evaluation
world: namely, that the Sherlock Holmes stories as they are in the evaluation
world describe a world in which such-and-such is true. So, what is happening is
that although it appears at first as if modal statements concern other possible
worlds and thus couldn't really be very informative, they actually only talk
about \emph{certain} possible worlds, those that stand in some relation to what
is going on at the ground level in the actual world. As a crude analogy,
consider:

\ex My grandmother is sick. \xe
%
At one level this is a claim about my grandmother. But it is also a claim about
me: namely that I have a grandmother who is sick. Thus it is with modal
statements. They talk about possible worlds that stand in a certain relation to
the actual world and thus they make claims about the actual world, albeit
slightly indirectly.

\subsection{The worlds of Sherlock Holmes} \label{sec:worlds-sherl-holm}

So far, we have played along with colloquial usage in talking of \emph{the}
world of Sherlock Holmes. But it is important to realize that this is sloppy
talk. \citet{lewis-1978-fiction} writes:

\begin{quote}
  [I]t will not do to follow ordinary language to the extent of supposing that
  we can somehow single out a single one of the worlds [as the one described by
  the stories]. Is the world of Sherlock Holmes a world where Holmes has an even
  or an odd number of hairs on his head at the moment when he first meets
  Watson? What is Inspector Lestrade's blood type? It is absurd to suppose that
  these questions about the world of Sherlock Holmes have answers. The best
  explanation of that is that the world\emph{s} of Sherlock Holmes are plural,
  and the questions have different answers at different ones.
\end{quote}
%
\note{An equivalent way to phrase this is that we are looking at the worlds \emph{not
    excluded by the (content of the) stories}.}%
The usual move at this point is to talk about the set of worlds
``\term{compatible with} the (content of) Sherlock Holmes stories in $w$''. We
imagine that we ask of each possible world whether what is going on in it is
compatible with the stories as they were written in our world. Worlds where
Holmes lives on Abbey Road are not compatible. Some worlds where he lives at
221B Baker Street are compatible (again not all, because in some such worlds he
is not a famous detective but an obscure violinist). Among the worlds compatible
with the stories are ones where he has an even number of hairs on his head at
the moment when he first meets Watson and there are others where he has an odd
number of hairs at that moment.

What the operator \expression{in the world of Sherlock Holmes} expresses is that
its complement is true throughout the worlds compatible with the stories. In
other words, the operator \emph{universally quantifies} over the compatible
worlds. Our next iteration of the semantics for the operator is therefore this:

\ex $\svt{in the world of Sherlock Holmes}^{w,g} =$\\
$\lambda p_{\type{s,t}}.\ \forall w' \mbox{ compatible with the Sherlock Holmes
  stories \emph{in} } w\!:$\\
$p(w') = 1$. \xe

At a very abstract level, the way we parse sentences of the form \emph{in the
  world of Sherlock Holmes, $\phi$} is that both components, the
\emph{in}-phrase and the prejacent, determine sets of possible worlds and that
the set of possible worlds representing the content of the fiction mentioned in
the \emph{in}-phrase is a subset of the set of possible worlds determined by the
prejacent. We will encounter the same rough structure of relating sets of
possible worlds in other intensional constructions.

This is where we will leave things. There is more to be said about fiction
operators like \expression{in the world of Sherlock Holmes}, but we will just
refer to you to the relevant literature. In particular, one might want to make
sense of Lewis' idea that a special treatment is needed for cases where the
sentence makes a claim about things that are left open by the fiction (no
truth-value, perhaps?). One also needs to figure out how to deal with cases
where the fiction is internally inconsistent. In any case, for our purposes
we're done with this kind of operator.

\subsection{What's next and a general pattern}
\label{sec:next}

With the basic framework of intensional semantics in place, we can now look at a
succession of intensional operators. In particular, we will explore the
semantics of propositional attitude predicates such as \expression{believe} or
\expression{want}, modal auxiliaries such as \expression{must} or
\expression{might}, and conditional sentences.

\note{%
  \(M\ [f(a)]\ (\phi)\)
  \begin{description}
  \item[\(M\):] a quantificational/modal relation between two sets of worlds
    (propositions)
  \item[\(a\):] the anchor of the modal claim
  \item[\(f\):] the flavor function that projects a set of worlds from the
    anchor
  \item[\(\phi\):] the prejacent set of worlds (proposition)
  \end{description}
} %
We will see a general pattern at work. In our Sherlock Holmes example, an
intensional operator can be seen as taking an \emph{anchor} \(a\) (the Sherlock
Holmes stories as they are in actual world) and \emph{project} from it a set of
compatible worlds, this can be encapsulated in a ``\emph{flavor}'' function
\(f\) from anchors to sets of possible worlds; the operator then makes a claim
with a certain quantificational \emph{force} \(M\) about the relation between
the projected set of worlds and the \emph{prejacent} worlds \(\phi\). Indirectly
then, a claim is thereby made about the anchor.

We will look at the non-trivial issues that arise when several intensional
operators interact (modals under attitudes, modals in the consequent of a
conditional, etc.). We will also see that constituents of the prejacent can
sometimes be evaluated with respect to a world that is not the world that the
intensional operator is taking us to (so-called \emph{de re} readings). Further,
we will move from worlds to times and explore the semantics of tense and aspect.
And, for the intrepid, this can all come together by exploring how tense and
aspect interact with attitudes, modality, and conditionals.

\section{*Issues with an informal meta-language}
\label{sec:meta-issues}

Exercise~\ref{exercise:meta} asks what is wrong with writing something like%
\note{Thanks to Magda Kaufmann, Angelika Kratzer, and Ede Zimmermann for
  discussions of the issues explored in this section, which is optional on a
  first pass, as indicated by the star on the section title.}%

\ex ($\lambda x.\ x \mbox{ is slow in } w$) (Watson) = Watson is slow
in $w$.\label{ex:badwatson} \xe

Think about it. On the left hand side of the ``='' sign is a meta-language
expression consisting of a $\lambda$-expression (so some kind of function)
applied to an individual (contributed by the meta-language name ``Watson''). The
function is a function from individuals to truth-values that will deliver the
truth-value 1 iff the individual is slow in world $w$. So, what we have on the
left hand side is the result of a function from individuals to truth-values
applied to an individual. In other words, on the left hand side we have a
truth-value, namely the truth-value 1 if Watson is slow in $w$ and the
truth-value 0 if Watson is not slow in $w$.

Now, what do we have on the right hand side of the ``=''? We have the
meta-language sentence ``Watson is slow in $w$''. That is not nor does it
contribute a truth-value. It is a statement of fact. Truth-values are not the
same as statements of fact.

\kwn The proper thing to do is to write

\ex ($\lambda x.\ x \mbox{ is slow in } w$) (Watson) = 1 iff Watson
is slow in $w$.\label{ex:goodwatson} \xe

There are actually two ways to parse the statement in (\lastx), both legitimate
it appears.

On one parse, the major connective is the meta-language expression ``iff''. On
its left hand side is a meta-language statement (that applying the function to
the individual Watson gives the truth-value 1) and on the right hand side of the
``iff'' we have another meta-language statement (that Watson is slow in $w$).
So, the whole thing says that these two statements are equivalent: (i) that
function applied to that individual gives us the truth-value 1, and (ii) that
Watson is slow in $w$.

\note{Is this weird? It turns out that natural language, not just our
  semi-formal meta-language, has conditionals that seem very similar:
  \expression{I fear [the consequences if we fail].} See
  \cite{lasersohn-1996-adnominal}, \cite{frana-2017-adnominal}, and
  \cite{bluemel-2019-adnominal-conditionals} for some discussion.}%
The other parse is perhaps more conspicuously represented as follows:

\ex ($\lambda x.\ x \mbox{ is slow in } w$) (Watson) = \leftchoice{1 if
  Watson is slow in $w$,0 if Watson is not slow in $w$} \xe
%
Here, the ``='' sign is the major connective. The left hand side is a
meta-language expression that resolves to a truth-value and the right hand side
as well contributes a truth-value: 1 if such and such and 0 if such and such.

H\&K, of course, introduced a convention that allowed meta-language
statements to be used in a place where a truth-value was expected (p.37, (9)):

\begin{quote}
  Read ``$[\lambda \alpha\co \phi.\ \gamma]$'' as either (i) or (ii),
  whichever makes sense.
  \begin{enumerate}[(i)]
  \item ``the function which maps every $\alpha$ such that $\phi$ to
    $\gamma$''
  \item ``the function which maps every $\alpha$ such that $\phi$ to
    1, if $\gamma$, and to 0 otherwise''
  \end{enumerate}
\end{quote}
%
Since it never makes sense to map anything to a meta-language statement, no
ambiguity will ever arise.

\note{This is the approach of \cite{stechow-1991-lecturenotes}.}%
So, one might want to extend this leeway and use it in the case of
\refx{ex:badwatson} as well. We could say that in general, meta-language
statements supply truth-values wherever that makes sense. In that case,
\refx{ex:badwatson} is just shorthand for \refx{ex:goodwatson}.

\newcommand{\nupsis}[1]{\ensuremath{\vdash #1 \dashv}}

Alternatively, %
\note{This is the approach Ede Zimmermann (pc) advocates and has been using
  in his classes.}%
one can introduce a new notation that indicates that a meta-language statement
is being used to contribute a truth-value:

\ex $\nupsis{\alpha}\ = \leftchoice{1 if $\alpha$,0 if otherwise}$ \xe

Lastly, one could abandon the H\&K informal meta-language approach
altogether and introduce a rigidly formalized meta-language.

These lecture notes will proceed to follow H\&K's approach and will not
introduce any further innovations. So, \refx{ex:badwatson} is illicit and only
\refx{ex:goodwatson} is acceptable.


\section{Further readings}

There is considerable overlap between this chapter and Chapter 12 of
\cite{heim-kratzer-1998-semantics}. Here, we approach intensional semantics from
a slightly different angle. It would probably be beneficial if you read H\&K's
Chapter 12 in addition to this chapter and if you did the exercises in there.
Come to think of it, some other ancillary reading is also recommended. You may
want to look at relevant chapters in other textbooks (for example:
\cite[Chapters 5\&6]{dowty-wall-peters-1981-intro}, \cite[Volume II: Intensional
Logic and Logical Grammar]{gamut:91}, \cite[Chapter 5:
Intensionality]{chierchia-mcconnell-ginet-2000-meaning-grammar}, and
\cite{zimmermann-sternefeld-2013-semantics}).

The \emph{Stanford encyclopedia of philosophy} is always a good resource.
\cite{menzel-2016-sep-possible-worlds} is the entry on possible worlds. A couple
of influential philosophical works on the metaphysics and uses of possible
worlds are \cite{kripke:naming:80} and \cite{lewis-1986-plurality-worlds}. An
interesting paper on the origins of the modern possible worlds semantics for
modal logic is \cite{copeland-2002-genesis}.

A must read for students who plan to go on to becoming specialists in semantics,
together with a handbook article putting it in perspective:
\cite{montague-1973-ptq} and \cite{partee-hendriks-1997-hbll}.

To learn more about discourse about fiction, read \cite{lewis-1978-fiction}.
Recent reconsiderations: \cite{bonomi-zucchi-2003-fiction},
\cite{hanley-2004-lewis-fiction}, \cite{proudfoot-2006-fiction}. An interesting
paper that explores the meaning of fictional texts:
\cite{bauer-beck-2014-fiction}. Inconsistencies in fictions and elsewhere are
discussed in: \cite{varzi-1997-inconsistency} and
\cite{lewis-1982-equivocators}.
  
Some other interesting work on stories and pictures and their content:
\cite{ross-1997-media}, \cite{zucchi-2001-tense-fiction},
\cite{blumson-2009-pictures}. More recently, there's been quite a bit of work on
pictorial semantics, see for example
\cite{abusch-rooth-2017-pictorial,greenberg-2018-pictorial,
  maier-bimpikou-2019-pictorial}.

\cite{hockett-altmann-1968-design} is a follow-up to Hockett's original article
on design features. See \cite{emonds-2011-primate-human} for a recent
re-appraisal. If you're interested in whether displacement really is an
exclusive feature of human language and cognition, there are some recent survey
articles: \cite{cheke-clayton-2010-mtt,redshaw-2014-mtt}. Two very recent
articles about displacement in humans and other animals are
\cite{leahy-carey-2020-modal-acquisition} and
\cite{redshaw-suddendorf-2020-timelines}.

\enlargethispage{24pt}
Astonishingly, Lewis' doctrine of the reality of the plurality of possible
worlds is being paralleled (pun absolutely intended) by theoretical physicists
in a number of ways. There is a controversial ``many worlds'' interpretation of
quantum mechanics, for example. Other terms found are the ``multiverse'' and
``parallel universes''. See for starters, Kai's blog entry on a popular book on
the issue, \url{http://kaivonfintel.org/many-worlds/}, MIT physics professor Max
Tegmark's page on the topic, \url{http://space.mit.edu/home/tegmark/crazy.html},
and a Fresh Air interview with physicist Brian Greene about his book \emph{The
  Hidden Reality: Parallel Universes and the Deep Laws of the Cosmos}:
\url{http://www.npr.org/2011/01/24/132932268/a-physicist-explains-why-parallel-universes-may-exist}.

%%% Local Variables:
%%% mode: latex
%%% TeX-master: "ik-book"
%%% End:
